% !TeX document-id = {dc20ae76-71b7-4d65-b662-f71c46d32e15}
% !TeX TXS-program:compile = txs:///pdflatex/[--shell-escape]
% !TeX root = 03_Variablen.tex
\documentclass[accentcolor=3c,landscape,ngerman,presentation,t,usenames,dvipsnames,svgnames,table]{tudabeamer}

% Template-Modifikationen
\addtobeamertemplate{frametitle}{}{\vspace{-1em}} % mehr Platz vor dem Inhalt

% andere global gemeinsame definitionen
%Includes
\usepackage[ngerman]{babel} %Deutsche Silbentrennung
\usepackage[utf8]{inputenc} %Deutsche Umlaute
\usepackage{float}
\usepackage{graphicx}
\usepackage{minted}
\RequirePackage{csquotes}
\RequirePackage{fontawesome5}

\DeclareGraphicsExtensions{.pdf,.png,.jpg}

\makeatletter
\author{Vorkursteam der Fachschaft Informatik}
\let\Author\@author

% dark mode
\ExplSyntaxOn
\IfDarkModeT{
    \cs_if_exist:NT \setbeamercolor {
        \setbeamercolor*{smallrule}{bg=.}
        \setbeamercolor*{normal~text}{bg=\thepagecolor,fg=.}
        \setbeamercolor*{background~canvas}{parent=normal~text}
        \setbeamercolor*{section~in~toc}{parent=normal~text}
        \setbeamercolor*{subsection~in~toc}{parent=normal~text,fg=.}
        \setbeamercolor*{footline}{parent=normal~text}
        \setbeamercolor{block~title~alerted}{fg=white,bg=white!20!\thepagecolor}
        \setbeamercolor*{block~body}{bg=black!70!gray!98!blue}
        \setbeamercolor*{block~body~alerted}{bg=\thepagecolor}
    }
    \cs_if_exist:NT \setbeamertemplate {
        \setbeamertemplate{subsection~in~toc~shaded}[default][50]
    }
}
\ExplSyntaxOff

% macros
\renewcommand{\arraystretch}{1.2} % Höhe einer Tabellenspalte minimal erhöhen
\newcommand{\N}{{\mathbb N}}
\renewcommand{\code}{\inputminted[]{python}}

\IfDarkModeTF{
    \newmintedfile[pythonfile]{python}{
        fontsize=\small,
        style=native,
        linenos=true,
        numberblanklines=true,
        tabsize=4,
        obeytabs=false,
        breaklines=true,
        autogobble=true,
        encoding="utf8",
        showspaces=false,
        xleftmargin=20pt,
        frame=single,
        framesep=5pt,
    }
    \newmintinline{python}{
        style=native,
        encoding="utf8"
    }
    \newmintinline{kotlin}{
        style=native,
        encoding="utf8"
    }


    \definecolor{codegray}{HTML}{eaf1ff}
    \newminted[bashcode]{awk}{
        escapeinside=||,
        fontsize=\small,
        style=native,
        linenos=true,
        numberblanklines=true,
        tabsize=4,
        obeytabs=false,
        breaklines=true,
        autogobble=true,
        encoding="utf8",
        showspaces=false,
        xleftmargin=20pt,
        frame=single,
        framesep=5pt
    }
}{
    \newmintedfile[pythonfile]{python}{
        fontsize=\small,
        style=friendly,
        linenos=true,
        numberblanklines=true,
        tabsize=4,
        obeytabs=false,
        breaklines=true,
        autogobble=true,
        encoding="utf8",
        showspaces=false,
        xleftmargin=20pt,
        frame=single,
        framesep=5pt,
    }
    \newmintinline{python}{
        style=friendly,
        encoding="utf8"
    }
    \newmintinline{kotlin}{
        style=friendly,
        encoding="utf8"
    }

    \definecolor{codegray}{HTML}{eaf1ff}
    \newminted[bashcode]{awk}{
        escapeinside=||,
        fontsize=\small,
        style=friendly,
        linenos=true,
        numberblanklines=true,
        tabsize=4,
        obeytabs=false,
        breaklines=true,
        autogobble=true,
        encoding="utf8",
        showspaces=false,
        xleftmargin=20pt,
        frame=single,
        framesep=5pt
    }
}

\let\origpythonfile\pythonfile
\renewcommand{\pythonfile}[1]{\pythonfileh{#1}{}}
\newcommand{\pythonfileh}[2]{\origpythonfile[#2]{#1}}

\DeclareDocumentCommand{\kotlinfile}{O{} O{} m}{\inputCode[#1]{minted language=kotlin,#2}{#3}}

\newcommand*{\ditto}{\texttt{\char`\"}}

\newcommand{\shellprefix}{\textcolor{TUDa-3a}{\ttfamily\bfseries \$~}}
\DeclareTCBListing{commandshell}{ O{} O{} }{
    colback=\IfDarkModeTF{black}{black!80},
    colupper=white,
    colframe=TUDa-3a,
    listing only,
    % listing options={style=tcblatex,language=sh},
    listing engine=minted,
    minted style=dracula,
    minted options={
        % linenos=true,
        numbersep=3mm,
        texcl=true,
        autogobble,
        escapeinside=@@,
        breaklines,
        highlightcolor=yellow!50!black,
        #1
    },
    #2,
    % before upper={\textcolor{red}{\small\ttfamily\bfseries root \$> }},
    % every listing line={\textcolor{red}{\small\ttfamily\bfseries root \$> }}
}

%Includes
\usepackage{epstopdf}
\usepackage{wrapfig}
\usepackage{tipa}
\usepackage{tikz}
\usetikzlibrary{calc,shapes,arrows}
%tip: use http://l04.scarfboy.com/coding/phonetic-translation?from=ipa&fromtext=%CB%88pa%C9%AA%CE%B8n%CC%A9&to=tipa
%for converting ipa


\graphicspath{ {./media/} }

\def\shortyear#1{\expandafter\shortyearhelper#1}
\def\shortyearhelper#1#2#3#4{#3#4}

\newcount\NextYear
\NextYear = \year
\advance\NextYear by 1

\newcommand\NextYearShort{\shortyear{\the\NextYear}}

% notes
\usepackage{pgfpages}
\setbeamertemplate{note page}[plain]
%\setbeameroption{show notes on second screen}

% macro for change speaker sign
\newcommand{\changespeaker}{
	\begin{tikzpicture}[line width=.6mm, shorten >= 3pt, shorten <= 3pt]

	\coordinate (c1);
	\coordinate[right of=c1] (c2);

	\draw[rectangle, draw=red!80, fill=red!80, align=center, rounded corners] ($(c1.north west)+(0,-0.3)$) rectangle ($(c2.south east)+(0, 0.3)$) {};
	\draw[->,white] (c1)[bend left] to node[auto] {} (c2);
	\draw[->,white] (c2)[bend left] to node[auto] {} (c1);
	\end{tikzpicture}
}

%Listing-Style pyhon
\title[Programmiervorkurs]{Programmiervorkurs Wintersemester \the\year/\NextYearShort}
\subtitle{{\small der Fachschaft Informatik}}
\logo*{\includegraphics{../globalMedia/bildmarke_ohne_rand}}
\institute{Fachschaft Informatik}
\date{Wintersemester \the\year/\NextYearShort}


% macros
\newcommand{\livecoding}{\begin{frame}\frametitle{\insertsectionhead \\  {\small \insertsubsectionhead}}\centering \huge \vskip 2cm\textbf{\textcolor{red}{Live-Coding}}\end{frame}}

%\newcommand{\slidehead}{\frametitle{\insertsectionhead \\ {\small \insertsubsectionhead}}\vspace{3mm}}
\newcommand{\slidehead}{\frametitle{\insertsectionhead} \framesubtitle{\insertsubsectionhead}\vspace{3mm}}
\newcommand{\tocslide}{\begin{frame}[t]\frametitle{Inhaltsverzeichnis}\vspace{3mm}{\small\tableofcontents[subsectionstyle=shaded]}\end{frame}}


% colors
\definecolor{lightpetrol}{RGB}{0,223,194}


\begin{document}

%Deckblatt
\subtitle{Kapitel 3: Daten Zwischenspeichern}
\titlegraphic{
    \begin{columns}
        \begin{column}{10cm}
            \begin{center}
                {\huge Variablen}
            \end{center}
            \vspace{-1mm}
            \begin{figure}
                \centering
                \includeinvertablegraphics[scale=.35]{media/x.png}
                \\	\sffamily \tiny Bild: \href{https://xkcd.com/2309/}{https://xkcd.com/2309/}
            \end{figure}
        \end{column}
    \end{columns}}
\maketitle

\section{Was sind Variablen?}
\begin{frame}[fragile]
    \slidehead
    % Analogie: Fächer; Computer muss wissen, WAS in ein Fach gelegt werden soll.
    % Idee: Kinderspielzeug -> Bauklötze in verschiedenen Formen als Bild?
    \begin{itemize}[<+->]
        \item Speicherplatz für Werte
        \item Eine Box im Speicher, in der man Werte ablegen kann
        \item Variablennamen sind für Menschen besser zu merken als nummerierte Speicherplätze
    \end{itemize}
\end{frame}

\begin{frame}[fragile]
    \slidehead
    % Analogie: Fächer; Computer muss wissen, WAS in ein Fach gelegt werden soll.
    % Idee: Kinderspielzeug -> Bauklötze in verschiedenen Formen als Bild?
    \begin{figure}
        \centering
        \includeinvertablegraphics[scale=.2]{media/wip_variables_title.png}
        \\	\sffamily \tiny Bild: Tristan Schulz
    \end{figure}
\end{frame}

\section{Ablegen von werten in einer Variable}
\subsection{Zuweisung}
\begin{frame}
    \slidehead
    \begin{itemize}[<+->]
        \item Variablen kann ein Wert zugewiesen werden, indem man den Zuweisungsoperator \lstinline{=} verwendet\\
            \pythoninline{<variableName>: <Datentyp> = <Wert>}

    \end{itemize}
    \onslide<+->
    \pythonfile{listings/variables.py}
\end{frame}

\begin{frame}
    \slidehead

    Variablennamen:
    \begin{itemize}[<+->]
        \item sind ein Verweis auf die Stelle an dem der Wert der Variable liegt
        \item \textbf{müssen} mit einem Buchstaben oder Unterstrich beginnen
        \item \textbf{dürfen keine} Punkte, Bindestriche, Sterne oder andere Sonderzeichen enthalten
        \item sollten \textbf{aussagekräftig} sein!
        \item Groß- und Kleinschreibung ist \textbf{wichtig}! ("Case sensitive"):\\ $eineVariable \neq EINEVARIABLE \neq EiNeVaRiAbLe$.
        \item Namenskonvention \pythoninline{snake_case} für Variablen
    \end{itemize}
\end{frame}

\section{Variablen auf der Konsole ausgeben!}
\begin{frame}
    \slidehead
    Um verschiedene Informationen sichtbar für den Nutzer eines Programms zu machen, kann man den Befehl \pythoninline{print} verwenden.
    \pythonfile{listings/printing.py}
    Ausgabe auf der Konsole sieht wie folgt aus!
    \pythonfile{listings/printing_console.sh}
\end{frame}

%TODO: TOLLES
% \subsection{Exkurs: Historik von Print}
% \begin{frame}
%     \slidehead
%     Der Befehl heißt \pythoninline{print}, das
% \end{frame}

\section{Ablegen von werten in einer Variable}
\subsection{Zuweisung}
\livecoding

%\nextvid{Konsoleneingabe}{Variablen Beispiel}

% \subtitle{Kapitel 3: Daten Zwischenspeichern}
% \titlegraphic{
% 	\begin{columns}
% 		\begin{column}{10cm}
% 			\begin{center}
% 				{\huge Konsoleneingabe}
% 			\end{center}
% 			\vspace{-1mm}
% 			\begin{figure}
% 				\centering
% 			\includegraphics[scale=.35]{media/incident\IfDarkModeT{_dark}.png}
% 				\\	\sffamily \tiny Bild: \href{https://xkcd.com/838/}{https://xkcd.com/838/}
% 			\end{figure}
% 		\end{column}
% 	\end{columns}}
% \maketitle
%\subtitle{Konsoleneingabe}


\section{Exkurs: String Templates}
\begin{frame}
    \slidehead

    %TODO: Ich weiß nicht ob wir das hier reinbringen wollen, dass scheint mir etwas suspekt
    \begin{itemize}[<+->]
        \item \textbf{Ziel}: Komplexere ausgaben ermöglichen durch nutzung von Strings (Zeichenketten).
        \item Mithilfe der Funktion \pythoninline{str} und dem Konkatenationsoperator \pythoninline{+} können komplexere Ausdrücke zusammengebaut werden
            \pythonfile{listings/string_no_template.py}
        \item Dies kann man jedoch auch mit einem Template String umsetzen
            \pythonfile{listings/string_template.py}
    \end{itemize}
\end{frame}

\begin{frame}
    \slidehead

    \begin{itemize}[<+->]
        \item Komplexere Ausdrücke funktionieren selbstverständlich auch!
    \end{itemize}
        \pythonfile{listings/string_template_complex.py}
        \pythonfile{listings/string_template_complex_output.sh}
\end{frame}

\section{Exkurs: Konsoleneingabe}
\begin{frame}
    \slidehead

    \begin{itemize}
        \item \textbf{Ziel}: Eingabe in der Konsole \textit{einlesen} und \textit{verarbeiten}
    \end{itemize}

    \begin{block}{Beispiele}
        \begin{itemize}
            \item Zeichenkette einlesen
                \pythonfile[][top=0cm,bottom=0cm]{listings/input.py}
            \item Um andere Typen einzulesen, muss \textit{konvertiert} werden
        \end{itemize}
    \end{block}
\end{frame}

\begin{frame}
    \slidehead

    Beispiele: Eingabekonvertierung
    \pythonfile{listings/string_conversion.py}

    \begin{block}{Hinweis}
        In python kann man sich den Typ eines Ausdrucks mit dem Befehl \pythoninline{type(<Ausdruck>)} ausgeben lassen
    \end{block}
\end{frame}

\livecoding

%\nextvid{Anweisungen und Ausdrücke}{Konsoleneingabe Beispiele}

% \subtitle{Kapitel 3: Daten Zwischenspeichern}
% \titlegraphic{
% 	\begin{columns}
% 		\begin{column}{10cm}
% 			\begin{center}
% 				{\huge Anweisungen und Ausdrücke}
% 			\end{center}
% 			\vspace{-2mm}
% 			\begin{figure}
% 				\centering
% 			\includeinvertablegraphics[scale=.35]{media/sandwich.png}
% 				\\	\sffamily \tiny Bild: \href{https://xkcd.com/149/}{https://xkcd.com/149/}
% 			\end{figure}
% 		\end{column}
% 	\end{columns}}
% \maketitle

%\begin{frame}
%	\vspace{2cm}
%	\begin{center}
%		\huge Anweisungen und Ausdrücke
%	\end{center}
%\end{frame}

% Redaktionelle änderung
% \section{Anweisungen und Ausdrücke}
% \begin{frame}
%     \slidehead
%     \begin{itemize}
%         \item Eine \textbf{Anweisung (Statement)} führt etwas aus
%         \item Ein \textbf{Ausdruck (Expression)} wertet zu einem \textit{Wert} aus
%     \end{itemize}
%     \pause
%     \kotlinfile{listings/anweisungen.kts}
%     \begin{itemize}
%         \item Jede oben gezeigte Zeile enthält \textit{eine Anweisung}
%         \item \kotlininline{1 + 42 + 5} ist ein \textit{Ausdruck}
%             \pause
%         \item Auch die Zahl $1$ ist hier ein Ausdruck, der zu dem Wert $1$ ausgewertet wird
%         \item Das gilt in Zeile 2 auch für den String
%     \end{itemize}
%
% \end{frame}
%
% \subsection{Zusammengesetze Operatoren}
% \begin{frame}[fragile]
%     \slidehead
%
%     \begin{itemize}
%         \item Häufig werden Variablen gelesen und gleichzeitig geschrieben
%     \end{itemize}
%
%     \kotlinfile{listings/zusammengesetzeoperatoren.kts}
%
%     \begin{itemize}
%         \item Es gibt noch weitere:
%             \begin{itemize}
%                 \item \verb+-=+
%                 \item \verb+*=+
%                 \item \verb+/=+
%                 \item \verb+%=+
%             \end{itemize}
%             \pause
%         \item Quiz: Was passiert bei \kotlininline{val} statt \kotlininline{var}?
%     \end{itemize}
% \end{frame}
%
% \livecoding
%
% %\nextvid{If-Anweisung}{Anweisungen und Ausdrücke Beispiel}

\section{Control Flow}
\subsection{If-Anweisung}
\begin{frame}
    \slidehead
    \note{If-Anweisungen sind mächtig. Bisher war alles einspurig. Immer der gleiche Weg von Start bis Ziel. Das kann auch der Taschenrechner. Ab jetzt kann das Programm selber Entscheidungen treffen, welche Codezeilen ausgeführt werden. Berechnungen können nun entscheiden, ob andere Berechnungen gemacht werden.}
    \begin{center}
        \vspace{-0.25cm}
        %{\fontsize{200}{50}\selectfont $\prec$}
        % Photo by pine watt on Unsplash.  Source: https://unsplash.com/photos/biqNmQEjfTA Licence is completely free to do anything: "More precisely, Unsplash grants you an irrevocable, nonexclusive, worldwide copyright license to download, copy, modify, distribute, perform, and use photos from Unsplash for free, including for commercial purposes, without permission from or attributing the photographer or Unsplash."
        \begin{figure}
            \includegraphics[height=0.8\textheight]{media/direction.jpg}
        \end{figure}
    \end{center}

\end{frame}

\begin{frame}[c]
    \slidehead
    % Define block styles
    \tikzstyle{block} = [rectangle, draw, fill=\IfDarkModeTF{TUDa-2b!40!black}{TUDa-2b},
        text width=10em, text centered, rounded corners, minimum height=4em]
    \tikzstyle{line} = [draw, -latex']

    \begin{center}
        %Flowchart:
        \begin{tikzpicture}[node distance = 4cm, auto]
            \node [block] (if) {Führerschein?};
            \node [block, below left of=if] (true) {Autofahren};
            \node [block, below right of=if] (false) {nicht Autofahren};

            \draw [-latex,thick] (if) --  node [left] {$ja$} (true) ;
            \draw [-latex,thick] (if) --  node [right] {$nein$} (false) ;

        \end{tikzpicture}
    \end{center}
\end{frame}

\begin{frame}
    \slidehead
    \visible<2->{
        \vspace{-1ex}
        \begin{itemize}
            \item Programm läuft sequentiell, d.h zeilenweise
            \item If-Anweisungen testen \textit{Bedingungen}
            \item Bei Auswertung zu \pythoninline{False} wird der Block übersprungen, bei \pythoninline{True} wird dieser ausgeführt
        \end{itemize}}
    \pythonfile[][top=0cm,bottom=0cm]{listings/if_anweisung.py}
    \onslide<3->
    \vspace{-1ex}
    \begin{block}{Hinweis}
        \begin{itemize}
            \item<3-> Der Anweisungscode muss eingerückt werden
            \item<4-> Es gibt \textbf{keine} \enquote{If-Schleife}
        \end{itemize}
    \end{block}
\end{frame}

\subsection*{Beispiel}
\begin{frame}[c]
    \slidehead
    \pythonfile[][top=0cm,bottom=0cm]{listings/if_if.py}
    Ausgabe auf der Konsole:
    \plainfile[][top=0cm,bottom=0cm]{listings/if_if_output.sh}
\end{frame}


\section{If-Anweisung}
\subsection{Else-Klausel}

\begin{frame}
    \slidehead

    \begin{itemize}
        \item \pythoninline{else} bezeichnet einen \textit{alternativ auszuführenden Teil} des Codes
        \item Übersetzung: "`Wenn \textbf{DAS} zutrifft, mache \textbf{DIESES}; falls nicht, mache \textbf{JENES}"'
    \end{itemize}

    \onslide<2->
    \vspace{-1ex}
    \pythonfile[][top=0cm,bottom=0cm]{listings/if_bsp2.py}
    Ausgabe auf der Konsole:
    \plainfile[][top=0cm,bottom=0cm]{listings/if_if_output.sh}
    \vspace{-1ex}

\end{frame}

\livecoding
%\nextvid{Else-Klausel}{}

\subsection{Else If-Klausel}

\begin{frame}
    \slidehead
    \begin{itemize}
        \item Manchmal gibt es mehr als zwei Fälle:
            \pythonfile{listings/if_if_if.py}
        \item[$\Rightarrow$] Auch dafür gibt es eine Kurzschreibweise
    \end{itemize}
\end{frame}

\begin{frame}
    \slidehead

    \begin{itemize}
        \item Mit \pythoninline{elif} können mehrere If-Anweisungen aneinandergereiht werden
        \item Es handelt sich um ein \pythoninline{else} gefolgt von einem \pythoninline{if}
        \item Weitere \pythoninline{else if}-Klauseln werden nur dann überprüft, wenn alle vorherigen Klauseln nicht eingetreten sind.
            \pythonfile{listings/if_else_if_example.py}
    \end{itemize}
\end{frame}

\livecoding

%\nextvid{Logische Operatoren}{if und elif Beispiele}

\section{Logische Operatoren}

% \titlegraphic{
% 	\begin{columns}
% 		\begin{column}{4cm}
% 			\vspace{1.5cm}
% 			\begin{center}
% 				{\huge Logische Operatoren}
% 			\end{center}
% 		\end{column}
% 		\begin{column}{4cm}
% 			\vspace{-4mm}
% 			\begin{figure}
% 				\centering
% 			\includeinvertablegraphics[scale=.2]{media/logic_boat.png}
% 				\\	\sffamily \tiny Bild: \href{https://xkcd.com/1134/}{https://xkcd.com/1134/}
% 			\end{figure}
% 		\end{column}
% 	\end{columns}}
% \maketitle

\begin{frame}
    \slidehead
    \begin{itemize}
        \item Logisches \textbf{Nicht} (\pythoninline{not})  $\Rightarrow$ \pythoninline{not Ausdruck1}
        \item Logisches \textbf{Und} (\pythoninline{and}) $\Rightarrow$ \pythoninline{Ausdruck1 and Ausdruck2}
        \item Logisches \textbf{Oder} (\pythoninline{or})  $\Rightarrow$ \pythoninline{Ausdruck1 or Ausdruck2}
    \end{itemize}
    \pythonfile[][top=0cm,bottom=0cm]{listings/logische_operatoren_listing.py}
    \begin{block}{Hinweis}
        \begin{itemize}
            \item Auswertung in der Reihenfolge \pythoninline{not}, \pythoninline{and} und dann \pythoninline{or}
            \item Reihenfolge immer beachten!
                Klammern setzen falls nötig!
        \end{itemize}
    \end{block}
\end{frame}

\livecoding

\section{Quiz}
% TODO: Quiz besser machen mit lösung anzeigen
\begin{frame}[fragile]
    \slidehead
    \begin{itemize}[<+->]
        \item Was darf nicht in einem Variablennamen enthalten sein?
        \item Sind \pythoninline{=} und \pythoninline{==} die gleichen Operatoren?
        \item Darf der Rumpf einer If-Anweisung leer sein?
        \item Darf eine If-Anweisung im Rumpf einer anderen If-Anweisung stehen?
    \end{itemize}
\end{frame}


% \nextvid{Warum Schleifen}{}

\end{document}
