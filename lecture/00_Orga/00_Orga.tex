% !TeX document-id = {6b928c0d-85af-41d1-a21b-e6c7a23a6e8d}
% !TeX TXS-program:compile = txs:///pdflatex/[--shell-escape]
\documentclass[accentcolor=3c,landscape,ngerman,presentation,t,usenames,dvipsnames,svgnames,table]{tudabeamer}

% Template-Modifikationen
\addtobeamertemplate{frametitle}{}{\vspace{-1em}} % mehr Platz vor dem Inhalt

% andere global gemeinsame definitionen
%Includes
\usepackage[ngerman]{babel} %Deutsche Silbentrennung
\usepackage[utf8]{inputenc} %Deutsche Umlaute
\usepackage{float}
\usepackage{graphicx}
\usepackage{minted}
\RequirePackage{csquotes}
\RequirePackage{fontawesome5}

\DeclareGraphicsExtensions{.pdf,.png,.jpg}

\makeatletter
\author{Vorkursteam der Fachschaft Informatik}
\let\Author\@author

% dark mode
\ExplSyntaxOn
\IfDarkModeT{
    \cs_if_exist:NT \setbeamercolor {
        \setbeamercolor*{smallrule}{bg=.}
        \setbeamercolor*{normal~text}{bg=\thepagecolor,fg=.}
        \setbeamercolor*{background~canvas}{parent=normal~text}
        \setbeamercolor*{section~in~toc}{parent=normal~text}
        \setbeamercolor*{subsection~in~toc}{parent=normal~text,fg=.}
        \setbeamercolor*{footline}{parent=normal~text}
        \setbeamercolor{block~title~alerted}{fg=white,bg=white!20!\thepagecolor}
        \setbeamercolor*{block~body}{bg=black!70!gray!98!blue}
        \setbeamercolor*{block~body~alerted}{bg=\thepagecolor}
    }
    \cs_if_exist:NT \setbeamertemplate {
        \setbeamertemplate{subsection~in~toc~shaded}[default][50]
    }
}
\ExplSyntaxOff

% macros
\renewcommand{\arraystretch}{1.2} % Höhe einer Tabellenspalte minimal erhöhen
\newcommand{\N}{{\mathbb N}}
\renewcommand{\code}{\inputminted[]{python}}

\IfDarkModeTF{
    \newmintedfile[pythonfile]{python}{
        fontsize=\small,
        style=native,
        linenos=true,
        numberblanklines=true,
        tabsize=4,
        obeytabs=false,
        breaklines=true,
        autogobble=true,
        encoding="utf8",
        showspaces=false,
        xleftmargin=20pt,
        frame=single,
        framesep=5pt,
    }
    \newmintinline{python}{
        style=native,
        encoding="utf8"
    }
    \newmintinline{kotlin}{
        style=native,
        encoding="utf8"
    }


    \definecolor{codegray}{HTML}{eaf1ff}
    \newminted[bashcode]{awk}{
        escapeinside=||,
        fontsize=\small,
        style=native,
        linenos=true,
        numberblanklines=true,
        tabsize=4,
        obeytabs=false,
        breaklines=true,
        autogobble=true,
        encoding="utf8",
        showspaces=false,
        xleftmargin=20pt,
        frame=single,
        framesep=5pt
    }
}{
    \newmintedfile[pythonfile]{python}{
        fontsize=\small,
        style=friendly,
        linenos=true,
        numberblanklines=true,
        tabsize=4,
        obeytabs=false,
        breaklines=true,
        autogobble=true,
        encoding="utf8",
        showspaces=false,
        xleftmargin=20pt,
        frame=single,
        framesep=5pt,
    }
    \newmintinline{python}{
        style=friendly,
        encoding="utf8"
    }
    \newmintinline{kotlin}{
        style=friendly,
        encoding="utf8"
    }

    \definecolor{codegray}{HTML}{eaf1ff}
    \newminted[bashcode]{awk}{
        escapeinside=||,
        fontsize=\small,
        style=friendly,
        linenos=true,
        numberblanklines=true,
        tabsize=4,
        obeytabs=false,
        breaklines=true,
        autogobble=true,
        encoding="utf8",
        showspaces=false,
        xleftmargin=20pt,
        frame=single,
        framesep=5pt
    }
}

\let\origpythonfile\pythonfile
\renewcommand{\pythonfile}[1]{\pythonfileh{#1}{}}
\newcommand{\pythonfileh}[2]{\origpythonfile[#2]{#1}}

\DeclareDocumentCommand{\kotlinfile}{O{} O{} m}{\inputCode[#1]{minted language=kotlin,#2}{#3}}

\newcommand*{\ditto}{\texttt{\char`\"}}

\newcommand{\shellprefix}{\textcolor{TUDa-3a}{\ttfamily\bfseries \$~}}
\DeclareTCBListing{commandshell}{ O{} O{} }{
    colback=\IfDarkModeTF{black}{black!80},
    colupper=white,
    colframe=TUDa-3a,
    listing only,
    % listing options={style=tcblatex,language=sh},
    listing engine=minted,
    minted style=dracula,
    minted options={
        % linenos=true,
        numbersep=3mm,
        texcl=true,
        autogobble,
        escapeinside=@@,
        breaklines,
        highlightcolor=yellow!50!black,
        #1
    },
    #2,
    % before upper={\textcolor{red}{\small\ttfamily\bfseries root \$> }},
    % every listing line={\textcolor{red}{\small\ttfamily\bfseries root \$> }}
}

%Includes
\usepackage{epstopdf}
\usepackage{wrapfig}
\usepackage{tipa}
\usepackage{tikz}
\usetikzlibrary{calc,shapes,arrows}
%tip: use http://l04.scarfboy.com/coding/phonetic-translation?from=ipa&fromtext=%CB%88pa%C9%AA%CE%B8n%CC%A9&to=tipa
%for converting ipa


\graphicspath{ {./media/} }

\def\shortyear#1{\expandafter\shortyearhelper#1}
\def\shortyearhelper#1#2#3#4{#3#4}

\newcount\NextYear
\NextYear = \year
\advance\NextYear by 1

\newcommand\NextYearShort{\shortyear{\the\NextYear}}

% notes
\usepackage{pgfpages}
\setbeamertemplate{note page}[plain]
%\setbeameroption{show notes on second screen}

% macro for change speaker sign
\newcommand{\changespeaker}{
	\begin{tikzpicture}[line width=.6mm, shorten >= 3pt, shorten <= 3pt]

	\coordinate (c1);
	\coordinate[right of=c1] (c2);

	\draw[rectangle, draw=red!80, fill=red!80, align=center, rounded corners] ($(c1.north west)+(0,-0.3)$) rectangle ($(c2.south east)+(0, 0.3)$) {};
	\draw[->,white] (c1)[bend left] to node[auto] {} (c2);
	\draw[->,white] (c2)[bend left] to node[auto] {} (c1);
	\end{tikzpicture}
}

%Listing-Style pyhon
\title[Programmiervorkurs]{Programmiervorkurs Wintersemester \the\year/\NextYearShort}
\subtitle{{\small der Fachschaft Informatik}}
\logo*{\includegraphics{../globalMedia/bildmarke_ohne_rand}}
\institute{Fachschaft Informatik}
\date{Wintersemester \the\year/\NextYearShort}


% macros
\newcommand{\livecoding}{\begin{frame}\frametitle{\insertsectionhead \\  {\small \insertsubsectionhead}}\centering \huge \vskip 2cm\textbf{\textcolor{red}{Live-Coding}}\end{frame}}

%\newcommand{\slidehead}{\frametitle{\insertsectionhead \\ {\small \insertsubsectionhead}}\vspace{3mm}}
\newcommand{\slidehead}{\frametitle{\insertsectionhead} \framesubtitle{\insertsubsectionhead}\vspace{3mm}}
\newcommand{\tocslide}{\begin{frame}[t]\frametitle{Inhaltsverzeichnis}\vspace{3mm}{\small\tableofcontents[subsectionstyle=shaded]}\end{frame}}


% colors
\definecolor{lightpetrol}{RGB}{0,223,194}


\usepackage{listings}
\usepackage{tikz}

\begin{document}

%Deckblatt
\subtitle{Organisatorisches}
\titlegraphic{
    \begin{columns}
        \begin{column}{6cm}
            \begin{figure}
                \centering
                \includegraphics[scale=0.065]{ws14_15}
                \caption{WS 2013/2014}
            \end{figure}
        \end{column}
        \begin{column}{6cm}
            \begin{figure}
                \centering
                \includegraphics[scale=0.21]{ws16_17}
                \caption{WS 2016/2017}
            \end{figure}
        \end{column}
    \end{columns}}
\maketitle

\section{Veranstalter}
\subsection*{Fachschaft Informatik}
\begin{frame}
    \slidehead
    \begin{itemize}
        \item "`Die Schüler*innenvertretung der Studierenden"'
        \item Ansprechpartner für Studierende bei Fragen zum Studium
        \item Bei Problemen zwischen Dozierenden und Studierenden vermitteln
        \item Den Studienablauf am Fachbereich weiter verbessern
        \item Soziale Interaktion zwischen Studierenden fördern
        \item Erstsemestern den Einstieg ins Studium erleichtern
        \item ...
    \end{itemize}
    \centering
    \vspace{3mm}
    \huge Mehr auf D120.de
\end{frame}

\section{Kontakt}
\begin{frame}
    \slidehead
    \begin{itemize}
        \item \textbf{Mail:} \href{mailto:vorkurs@d120.de}{vorkurs@d120.de}
        \item \textbf{Moodle:}  \href{https://moodle.informatik.tu-darmstadt.de/course/view.php?id=624} {https://moodle.informatik.tu-darmstadt.de}
    \end{itemize}
    \vspace{2.5cm}
    \begin{alertblock}{Hinweise}
        Ihr habt allgemeine Fragen: Schreibt ins Moodle-Forum. \\
        Ihr habt spezielle Fragen an uns: Schreibt uns eine Mail. \\
        Ihr könnt uns natürlich auch in Discord ansprechen.
    \end{alertblock}
\end{frame}

\section{Hygienevorschriften}
\begin{frame}
    \slidehead
    \begin{itemize}
        \item \textbf{Maskenpflicht!}
        \item auf dem gesamten Gelände der TU Darmstadt
        \item ``Die Maskenpflicht gilt seit dem 2. April an der TU weiterhin auf Verkehrsflächen, an Arbeitsplätzen und in Veranstaltungen.'' (Krisenstab der TU Darmstadt)
        \item \href{https://www.tu-darmstadt.de/universitaet/aktuelles_meldungen/corona_vorsorge/}{https://www.tu-darmstadt.de/universitaet/aktuelles\_meldungen/corona\_vorsorge/}
    \end{itemize}
\end{frame}

\section{Ablauf des Vorkurses}
\subsection{Ein üblicher Tag im Vorkurs}
\begin{frame}
    \slidehead
    \begin{itemize}
        \item \textbf{Vormittags Vorlesung (Hybrid)}
            \begin{itemize}
                \item interaktiv
                \item heißt: Fragen stellen
                \item zusätzlicher Stream auf \href{https://youtu.be/2V5YFfyPcr4}{https://youtu.be/2V5YFfyPcr4}
            \end{itemize}
            \pause
        \item \textbf{Mittagspause}
            \begin{itemize}
                \item Pausen sind wichtig
                \item Esst vielleicht ne Kleinigkeit.
                \item Wenn ihr schon eure Athenekarte habt, könnt ihr in die Mensa gehen.
            \end{itemize}
            \pause
        \item \textbf{Nachmittags Übung}
            \begin{itemize}
                \item Gruppenarbeit im C-Pool
                \item Tutor*innen
                \item offenes Ende
            \end{itemize}
    \end{itemize}
\end{frame}

\subsection{Wochenplan}
\begin{frame}
    \slidehead
    \textbf{Allgemeines Programm}
    \begin{itemize}
        \item Dienstag
            \begin{itemize}
                \item Vorlesung um 10:00
                \item Nachmittags: Python installieren und Übung
            \end{itemize}
            \pause
        \item Mittwoch (!)
            \begin{itemize}
                \item Anderer Raum!
                \item Vorlesung um 10:00 (S313|30) (Schloss)
                \item Nachmittags: Übung
            \end{itemize}
            \pause
        \item Donnerstag und Freitag
            \begin{itemize}
                \item Vorlesung um 10:00
                \item Nachmittags: Übung
            \end{itemize}
    \end{itemize}
\end{frame}

\subsection{Stoffplan}
\begin{frame}
    \slidehead
    \begin{itemize}
        \item Erste Schritte im Programmieren?
        \item Was ist Python?
        \item Mein erstes Programm
        \item Datentypen und Operatoren
        \item Schleifen und bedingte Anweisungen
        \item Funktionen \& Rekursion
    \end{itemize}
\end{frame}

\subsection{Voraussetzungen}
\begin{frame}
    \slidehead
    \centering
    \vspace{1.5cm}
    % \tiny Ein internetfähiges Endgerät,\\ansonsten:\\ % <- nur für eine online-Vorkurse
    \Huge Keine
\end{frame}

\subsection{Ziele}
\begin{frame}
    \slidehead
    \begin{itemize}
        \item Vereinfachter Start ins Studium
        \item Grundlegende Konzepte verstehen
        \item Leute kennenlernen

    \end{itemize}
\end{frame}

\subsection{Übungsaufgaben}
\begin{frame}
    \slidehead
    \begin{itemize}
        \item Aufgaben mit unterschiedlichen Schwierigkeitsgrad
        \item Tutor*innen
            %in Discord,
            die euch unterstützen
        \item Helft euch auch gegenseitig, denn so lernt ihr am meisten
        \item Ihr müsst nicht alle Aufgaben lösen
    \end{itemize}
\end{frame}

\section{Moodle}
\begin{frame}
    \slidehead
    \vskip -1.1em
    \begin{columns}
        \begin{column}{.667\linewidth}
            \begin{itemize}
                \item Vorlesungsfolien \& Übungen
                \item Ankündigungen
                \item Fragen, Foren
            \end{itemize}
        \end{column}
        \begin{column}{.3\linewidth}
            \\
            \vspace{-0.2cm}
            \includegraphics[width=.7\linewidth]{media/moodle_qr.png}
        \end{column}
    \end{columns}
    \vspace{-0.2cm}
    \begin{block}{Lernportal Informatik - Vorkurs}
        \vskip 1 pt
        \Huge{Gastschlüssel erhaltet ihr in Discord}\\
        \normalsize
        \vskip 1 pt
        Nähere Informationen zum Zugriff auf die Materialien:\\
        \quad \href{https://d120.de/vorkurs}{https://d120.de/vorkurs}
    \end{block}
\end{frame}


\section{Discord}
\begin{frame}
    \slidehead
    \vskip -1.1em
    \begin{columns}
        \begin{column}{.667\linewidth}
            \begin{itemize}
                \item Wir haben für euch einen Discord Server angelegt!
                \item dort könnt ihr euch untereinander austauschen
                \item dort könnt ihr auch mit erfahrenen Tutor*innen reden
            \end{itemize}
        \end{column}
        \begin{column}{.3\linewidth}
            \\
            \vspace{-0.2cm}
            %\includegraphics[width=.7\linewidth]{media/moodle_qr.png} %TODO QR code für discord link
        \end{column}
    \end{columns}
    \vspace{0.9cm}
    \begin{block}{Discord - Vorkurs}
        \normalsize
        \vskip 1 pt
        Für Discord braucht ihr einen Account!\\
        Ihr könnt Discord hier herunter laden:\\
        \quad \href{https://discord.com/}{https://discord.com/}\\\\
        Wie ihr in Discord kommt, erfahrt ihr auf der Vorkurs Seite.
        Lest euch bitte auf dem Server die Willkommensnachricht durch!
    \end{block}
\end{frame}


\section{YouTube}
\begin{frame}
    \slidehead
    \begin{itemize}
        \item wir streamen für euch die Vorlesungen Live
        \item ihr könnt während dessen fragen auf Discord stellen!
            %		\item Diskutiert auch gerne parallel in Discord.
    \end{itemize}
\end{frame}


\section{Ophase}
\begin{frame}
    \slidehead
    \textbf{Nächste Woche} ist die Ophase (10.10. bis 14.10.2022)! \\
    Nehmt auf jeden Fall daran teil. Es lohnt sich für euch!
\end{frame}

\end{document}
