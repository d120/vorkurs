% !TeX document-id = {6b928c0d-85af-41d1-a21b-e6c7a23a6e8d}
% !TeX TXS-program:compile = txs:///pdflatex/[--shell-escape]
\documentclass[accentcolor=3c,landscape,ngerman,presentation,t,usenames,dvipsnames,svgnames,table]{tudabeamer}

% Template-Modifikationen
\addtobeamertemplate{frametitle}{}{\vspace{-1em}} % mehr Platz vor dem Inhalt

% andere global gemeinsame definitionen
%Includes
\usepackage[ngerman]{babel} %Deutsche Silbentrennung
\usepackage[utf8]{inputenc} %Deutsche Umlaute
\usepackage{float}
\usepackage{graphicx}
\usepackage{minted}
\RequirePackage{csquotes}
\RequirePackage{fontawesome5}

\DeclareGraphicsExtensions{.pdf,.png,.jpg}

\makeatletter
\author{Vorkursteam der Fachschaft Informatik}
\let\Author\@author

% dark mode
\ExplSyntaxOn
\IfDarkModeT{
    \cs_if_exist:NT \setbeamercolor {
        \setbeamercolor*{smallrule}{bg=.}
        \setbeamercolor*{normal~text}{bg=\thepagecolor,fg=.}
        \setbeamercolor*{background~canvas}{parent=normal~text}
        \setbeamercolor*{section~in~toc}{parent=normal~text}
        \setbeamercolor*{subsection~in~toc}{parent=normal~text,fg=.}
        \setbeamercolor*{footline}{parent=normal~text}
        \setbeamercolor{block~title~alerted}{fg=white,bg=white!20!\thepagecolor}
        \setbeamercolor*{block~body}{bg=black!70!gray!98!blue}
        \setbeamercolor*{block~body~alerted}{bg=\thepagecolor}
    }
    \cs_if_exist:NT \setbeamertemplate {
        \setbeamertemplate{subsection~in~toc~shaded}[default][50]
    }
}
\ExplSyntaxOff

% macros
\renewcommand{\arraystretch}{1.2} % Höhe einer Tabellenspalte minimal erhöhen
\newcommand{\N}{{\mathbb N}}
\renewcommand{\code}{\inputminted[]{python}}

\IfDarkModeTF{
    \newmintedfile[pythonfile]{python}{
        fontsize=\small,
        style=native,
        linenos=true,
        numberblanklines=true,
        tabsize=4,
        obeytabs=false,
        breaklines=true,
        autogobble=true,
        encoding="utf8",
        showspaces=false,
        xleftmargin=20pt,
        frame=single,
        framesep=5pt,
    }
    \newmintinline{python}{
        style=native,
        encoding="utf8"
    }
    \newmintinline{kotlin}{
        style=native,
        encoding="utf8"
    }


    \definecolor{codegray}{HTML}{eaf1ff}
    \newminted[bashcode]{awk}{
        escapeinside=||,
        fontsize=\small,
        style=native,
        linenos=true,
        numberblanklines=true,
        tabsize=4,
        obeytabs=false,
        breaklines=true,
        autogobble=true,
        encoding="utf8",
        showspaces=false,
        xleftmargin=20pt,
        frame=single,
        framesep=5pt
    }
}{
    \newmintedfile[pythonfile]{python}{
        fontsize=\small,
        style=friendly,
        linenos=true,
        numberblanklines=true,
        tabsize=4,
        obeytabs=false,
        breaklines=true,
        autogobble=true,
        encoding="utf8",
        showspaces=false,
        xleftmargin=20pt,
        frame=single,
        framesep=5pt,
    }
    \newmintinline{python}{
        style=friendly,
        encoding="utf8"
    }
    \newmintinline{kotlin}{
        style=friendly,
        encoding="utf8"
    }

    \definecolor{codegray}{HTML}{eaf1ff}
    \newminted[bashcode]{awk}{
        escapeinside=||,
        fontsize=\small,
        style=friendly,
        linenos=true,
        numberblanklines=true,
        tabsize=4,
        obeytabs=false,
        breaklines=true,
        autogobble=true,
        encoding="utf8",
        showspaces=false,
        xleftmargin=20pt,
        frame=single,
        framesep=5pt
    }
}

\let\origpythonfile\pythonfile
\renewcommand{\pythonfile}[1]{\pythonfileh{#1}{}}
\newcommand{\pythonfileh}[2]{\origpythonfile[#2]{#1}}

\DeclareDocumentCommand{\kotlinfile}{O{} O{} m}{\inputCode[#1]{minted language=kotlin,#2}{#3}}

\newcommand*{\ditto}{\texttt{\char`\"}}

\newcommand{\shellprefix}{\textcolor{TUDa-3a}{\ttfamily\bfseries \$~}}
\DeclareTCBListing{commandshell}{ O{} O{} }{
    colback=\IfDarkModeTF{black}{black!80},
    colupper=white,
    colframe=TUDa-3a,
    listing only,
    % listing options={style=tcblatex,language=sh},
    listing engine=minted,
    minted style=dracula,
    minted options={
        % linenos=true,
        numbersep=3mm,
        texcl=true,
        autogobble,
        escapeinside=@@,
        breaklines,
        highlightcolor=yellow!50!black,
        #1
    },
    #2,
    % before upper={\textcolor{red}{\small\ttfamily\bfseries root \$> }},
    % every listing line={\textcolor{red}{\small\ttfamily\bfseries root \$> }}
}

%Includes
\usepackage{epstopdf}
\usepackage{wrapfig}
\usepackage{tipa}
\usepackage{tikz}
\usetikzlibrary{calc,shapes,arrows}
%tip: use http://l04.scarfboy.com/coding/phonetic-translation?from=ipa&fromtext=%CB%88pa%C9%AA%CE%B8n%CC%A9&to=tipa
%for converting ipa


\graphicspath{ {./media/} }

\def\shortyear#1{\expandafter\shortyearhelper#1}
\def\shortyearhelper#1#2#3#4{#3#4}

\newcount\NextYear
\NextYear = \year
\advance\NextYear by 1

\newcommand\NextYearShort{\shortyear{\the\NextYear}}

% notes
\usepackage{pgfpages}
\setbeamertemplate{note page}[plain]
%\setbeameroption{show notes on second screen}

% macro for change speaker sign
\newcommand{\changespeaker}{
	\begin{tikzpicture}[line width=.6mm, shorten >= 3pt, shorten <= 3pt]

	\coordinate (c1);
	\coordinate[right of=c1] (c2);

	\draw[rectangle, draw=red!80, fill=red!80, align=center, rounded corners] ($(c1.north west)+(0,-0.3)$) rectangle ($(c2.south east)+(0, 0.3)$) {};
	\draw[->,white] (c1)[bend left] to node[auto] {} (c2);
	\draw[->,white] (c2)[bend left] to node[auto] {} (c1);
	\end{tikzpicture}
}

%Listing-Style pyhon
\title[Programmiervorkurs]{Programmiervorkurs Wintersemester \the\year/\NextYearShort}
\subtitle{{\small der Fachschaft Informatik}}
\logo*{\includegraphics{../globalMedia/bildmarke_ohne_rand}}
\institute{Fachschaft Informatik}
\date{Wintersemester \the\year/\NextYearShort}


% macros
\newcommand{\livecoding}{\begin{frame}\frametitle{\insertsectionhead \\  {\small \insertsubsectionhead}}\centering \huge \vskip 2cm\textbf{\textcolor{red}{Live-Coding}}\end{frame}}

%\newcommand{\slidehead}{\frametitle{\insertsectionhead \\ {\small \insertsubsectionhead}}\vspace{3mm}}
\newcommand{\slidehead}{\frametitle{\insertsectionhead} \framesubtitle{\insertsubsectionhead}\vspace{3mm}}
\newcommand{\tocslide}{\begin{frame}[t]\frametitle{Inhaltsverzeichnis}\vspace{3mm}{\small\tableofcontents[subsectionstyle=shaded]}\end{frame}}


% colors
\definecolor{lightpetrol}{RGB}{0,223,194}


\usepackage{listings}
\usepackage{tikz}
\usepackage{tabularx,booktabs,multirow,multicol,colortbl}
\aboverulesep = 0mm \belowrulesep = 0mm

\def\streamlink{https://youtu.be/dTWa0Tk3C9U}
\def\moodlecourselink{https://moodle.informatik.tu-darmstadt.de/course/view.php?id=1502}

\begin{document}

%Deckblatt
\subtitle{Organisatorisches}
\titlegraphic{
    \begin{columns}
        \begin{column}{6cm}
            \begin{figure}
                \centering
                \includegraphics[scale=0.065]{ws14_15}
                \caption{WS 2013/2014}
            \end{figure}
        \end{column}
        \begin{column}{6cm}
            \begin{figure}
                \centering
                \includegraphics[scale=0.21]{ws16_17}
                \caption{WS 2016/2017}
            \end{figure}
        \end{column}
    \end{columns}}
\maketitle

\section{Veranstalter}
\subsection*{Fachschaft Informatik}
\begin{frame}
    \slidehead
    \begin{itemize}
        \item "`Die Schüler*innenvertretung der Studierenden"'
        \item Ansprechpartner für Studierende bei Fragen zum Studium
        \item Bei Problemen zwischen Dozierenden und Studierenden vermitteln
        \item Den Studienablauf am Fachbereich weiter verbessern
        \item Soziale Interaktion zwischen Studierenden fördern
        \item Erstsemestern den Einstieg ins Studium erleichtern
        \item ...
    \end{itemize}
    \centering
    \vspace{3mm}
    \huge Mehr auf D120.de
\end{frame}

\section{Moodle-Kurs}
\begin{frame}[c]
    \slidehead
    \begin{columns}[c]
        \begin{column}{.6\linewidth}
            \centering
            \begin{itemize}
                \item Vorlesungsfolien \& Übungen
                \item Ankündigungen
                \item Fragen, Foren
            \end{itemize}
        \end{column}%
        \begin{column}{.4\linewidth}
            \begin{figure}
                \centering\mbox{}
                \qrcode[height=2.5cm]{\moodlecourselink}
                \caption{\href{\moodlecourselink}{Link zum Moodle-Kurs}}
            \end{figure}
        \end{column}
    \end{columns}
    \vspace{-1em}
    \begin{block}{Lernportal Informatik - Vorkurs}
        {\Huge Gastschlüssel: PVK-2023}

        \vspace{1em}Nähere Informationen zum Zugriff auf die Materialien:\\
        \href{https://d120.de/vorkurs}{https://d120.de/vorkurs}
    \end{block}
\end{frame}


\section{Discord}
\begin{frame}[c]
    \slidehead
    \begin{columns}[c]
        \begin{column}{.6\linewidth}
            \begin{itemize}
                \item Discord-Server der Fachschaft (D120)
                    % \item zu finden im TU Darmstadt Student Hub
                \item dort könnt ihr euch untereinander austauschen
                \item dort könnt ihr auch mit erfahrenen Tutor*innen reden
            \end{itemize}
        \end{column}%
        \begin{column}{.4\linewidth}
            \begin{figure}
                \centering\mbox{}
                % \qrcode[height=2.5cm]{https://discord.gg/JUCns6f7vT}
                % \caption{\href{https://discord.gg/JUCns6f7vT}{Link zum TU-Darmstadt Student Hub}}
                \qrcode[height=2.5cm]{https://discord.gg/GBj9rvszDT}
                \caption{\href{https://discord.gg/GBj9rvszDT}{Link zum D120-Discord-Server}}
            \end{figure}
        \end{column}
    \end{columns}
    \begin{block}{d120 Discord - Vorkurs}
        Für Discord braucht ihr einen Account!

        Ihr könnt Discord hier herunterladen: \url{https://discord.com/}

        % Wie ihr in Discord kommt, erfahrt ihr auf der Vorkurs Seite.
        Lest euch bitte auf dem Server die Willkommensnachricht durch!
    \end{block}
\end{frame}


\section{YouTube}
\begin{frame}[c]
    \slidehead
    \begin{columns}[c]
        \begin{column}{.5\textwidth}
            \begin{itemize}
                \item wir streamen für euch die Vorlesungen auf YouTube live
                \item ihr könnt während dessen fragen auf Discord stellen!
                    %		\item Diskutiert auch gerne parallel in Discord.
                \item Link zum Live-Stream: \url{\streamlink}
            \end{itemize}
        \end{column}%
        \begin{column}{.5\textwidth}
            \begin{figure}
                \centering\mbox{}
                \qrcode[height=3cm]{\streamlink}
                \caption{\href{\streamlink}{Link zum Vorlesungs-Stream}}
            \end{figure}
        \end{column}
    \end{columns}
\end{frame}


\section{Kontakt}
\begin{frame}
    \slidehead
    \begin{itemize}
        \item \textbf{Mail:} \href{mailto:vorkurs@d120.de}{vorkurs@d120.de}
        \item \textbf{Moodle:}  Foren im \href{\moodlecourselink}{Moodle-Kurs}
    \end{itemize}
    \vspace{\fill}
    \begin{defBox}[title=Hinweise]%
        \vspace{-1em}
        \begin{itemize}
            \item allgemeine Fragen: Schreibt ins Moodle-Forum.
            \item spezielle Fragen an uns: Schreibt uns eine Mail.
        \end{itemize}
        Ihr könnt uns natürlich auch ansprechen.
    \end{defBox}
\end{frame}

\section{Ablauf des Vorkurses}
\subsection{Ein üblicher Tag im Vorkurs}
\begin{frame}
    \slidehead
    \begin{itemize}
        \item \textbf{Vormittags Vorlesung (Hybrid) 9:00 - 12:00}
            \begin{itemize}
                \item Vorlesung im Hörsaal S2|02 C205
                \item Übertragung in C110/C120, falls zu viele Leute da sind
                \item interaktiv
                \item heißt: Fragen stellen
                \item zusätzlicher Stream auf \url{\streamlink}
            \end{itemize}
            \pause
        \item \textbf{Mittagspause 12:00 - 13:00}
            \begin{itemize}
                \item Pausen sind wichtig
                \item Esst vielleicht ne Kleinigkeit.
                \item Wenn ihr schon eure Athenekarte habt, könnt ihr in die Mensa gehen.
            \end{itemize}
            \pause
        \item \textbf{Nachmittags Übung 13:00 - 18:00}
            \begin{itemize}
                \item Gruppenarbeit im C-Pool
                \item Tutor*innen
                \item offenes Ende
            \end{itemize}
    \end{itemize}
\end{frame}

\subsection{Wochenplan}
\begin{frame}
    \slidehead
    % \textbf{Allgemeines Programm}
    \begin{table}[ht!]
        \small\centering
        \def\vlcolor{\cellcolor{brown!15!\thepagecolor}}
        \def\pausecolor{\cellcolor{accentcolor!15!\thepagecolor}}
        \def\excolor{\cellcolor{TUDa-1b!15!\thepagecolor}}
        \rowcolors{2}{\thepagecolor}{fgcolor!10!\thepagecolor}
        \begin{tabularx}{\textwidth}{l>{\centering\arraybackslash}X>{\centering\arraybackslash}X>{\centering\arraybackslash}X>{\centering\arraybackslash}X>{\centering\arraybackslash}X}
            \toprule
            \fatsf{Zeit} & \fatsf{Montag}& \fatsf{Dienstag}& \fatsf{Mittwoch}& \fatsf{Donnerstag}& \fatsf{Freitag}\\
            \midrule
            \phantom{09:00} &  &  &  &  & \\
            \tikzmark{tpos-1} & \vlcolor{} &  & \vlcolor{} & \vlcolor{} & \vlcolor{}\\
            \tikzmark{tpos-2} & \vlcolor{} &  & \vlcolor{} & \vlcolor{} & \vlcolor{}\\
            \tikzmark{tpos-3} & \multirow{-3}{*}{\vlcolor{}\begin{minipage}[t][2cm][t]{\textwidth}
                    \tiny \fatsf{Vorlesung}\\
                    Foliensätze:\\
                    \quad - 00-Orga\\
                    \quad - 01-Erste Schritte\\
                    \quad - 02-Datentypen-Zeichenketten
                \end{minipage}} & & \multirow{-3}{*}{\vlcolor{}\begin{minipage}[t][2cm][t]{\textwidth}
                    \tiny \fatsf{Vorlesung}\\
                    Foliensätze:\\
                    \quad - 03-Variablen + If
                \end{minipage}}  &
            \multirow{-3}{*}{\vlcolor{}\begin{minipage}[t][2cm][t]{\textwidth}
                    \tiny \fatsf{Vorlesung}\\
                    Foliensätze:\\
                    \quad - 04-Schleifen\\
                    \quad - 05-Listen
                \end{minipage}} &
            \multirow{-3}{*}{\vlcolor{}\begin{minipage}[t][2cm][t]{\textwidth}
                    \tiny \fatsf{Vorlesung}\\
                    Foliensätze:\\
                    \quad - 06-Funktionen + Rekursion
                \end{minipage}}\\
            \tikzmark{tpos-4} & \pausecolor{}{\footnotesize Mittagspause} & & \pausecolor{}{\footnotesize Mittagspause} & \pausecolor{}{\footnotesize Mittagspause} & \pausecolor{}{\footnotesize Mittagspause}\\
            \tikzmark{tpos-5} & \excolor{} &  & \excolor{} & \excolor{} & \excolor{}\\
            \tikzmark{tpos-6} & \excolor{} &  & \excolor{} & \excolor{} & \excolor{}\\
            \tikzmark{tpos-7} & \excolor{} &  & \excolor{} & \excolor{} & \excolor{}\\
            \tikzmark{tpos-8} & \excolor{} &  & \excolor{} & \excolor{} & \excolor{}\\
            \tikzmark{tpos-9} &
            \multirow{-4}{*}{\excolor{}\begin{minipage}[t][3cm][t]{\textwidth}
                    \tiny \fatsf{Übungsphase}\\
                    \quad - Kotlin installieren\\
                    \quad - Übungsblatt 1
                \end{minipage}} & \multirow{-9}{*}{\textcolor{TUDa-9b}{Feiertag}} & \multirow{-4}{*}{\excolor{}\begin{minipage}[t][3cm][t]{\textwidth}
                    \tiny \fatsf{Übungsphase}\\
                    % \quad - Kotlin installieren\\
                    \quad - Übungsblatt 2
                \end{minipage}} & \multirow{-4}{*}{\excolor{}\begin{minipage}[t][3cm][t]{\textwidth}
                    \tiny \fatsf{Übungsphase}\\
                    % \quad - Kotlin installieren\\
                    \quad - Übungsblatt 3
                \end{minipage}} & \multirow{-4}{*}{\excolor{}\begin{minipage}[t][3cm][t]{\textwidth}
                    \tiny \fatsf{Übungsphase}\\
                    % \quad - Kotlin installieren\\
                    \quad - Übungsblatt 4\\
                    \quad - Challenge
                \end{minipage}}\\
            \tikzmark{tpos-10} &  &  &  &  & \\
            \bottomrule
        \end{tabularx}
        \caption{Caption}
        \label{tab:newtab}

        \begin{tikzpicture}[remember picture,overlay]
            \foreach \x in {1,...,10} {
                \node[anchor=south west] at ([xshift=-1mm,yshift=.5ex]pic cs:tpos-\x) {\padzeroes{\the\numexpr8+\x\relax}:00};
            }
        \end{tikzpicture}
    \end{table}
    % \begin{itemize}
    %     \item Montag
    %         \begin{itemize}
    %             \item Vorlesung um 9:00
    %             \item Nachmittags: Kotlin installieren und Übung
    %         \end{itemize}
    %         \pause
    %     \item Mittwoch, Donnerstag und Freitag
    %         \begin{itemize}
    %             \item Vorlesung um 9:00
    %             \item Nachmittags: Übung
    %         \end{itemize}
    % \end{itemize}
\end{frame}

% \subsection{Stoffplan}
% \begin{frame}
%     \slidehead
%     \begin{itemize}
%         \item Erste Schritte im Programmieren?
%         \item Was ist Kotlin?
%         \item Mein erstes Programm
%         \item Datentypen und Operatoren
%         \item Schleifen und bedingte Anweisungen
%         \item Funktionen \& Rekursion
%     \end{itemize}
% \end{frame}

\subsection{Voraussetzungen}
\begin{frame}
    \slidehead
    \centering
    \vspace{1.5cm}
    % \tiny Ein internetfähiges Endgerät,\\ansonsten:\\ % <- nur für eine online-Vorkurse
    \Huge Keine
\end{frame}

\subsection{Ziele}
\begin{frame}
    \slidehead
    \begin{itemize}
        \item Vereinfachter Start ins Studium
        \item Grundlegende Konzepte verstehen
        \item Leute kennenlernen

    \end{itemize}
\end{frame}

\subsection{Übungsaufgaben}
\begin{frame}
    \slidehead
    \begin{itemize}
        \item Aufgaben mit unterschiedlichen Schwierigkeitsgraden
        \item Tutor*innen
            %in Discord,
            die euch unterstützen
        \item Helft euch auch gegenseitig, denn so lernt ihr am meisten
        \item Ihr müsst nicht alle Aufgaben lösen
    \end{itemize}
\end{frame}

\section{Ophase}
\begin{frame}
    \slidehead
    \textbf{Nächste Woche} ist die Ophase (9.10. bis 13.10.2023)! \\
    Nehmt auf jeden Fall daran teil. Es lohnt sich für euch!
\end{frame}

\end{document}
