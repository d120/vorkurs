% !TeX document-id = {3bcfab70-4d4f-4fac-ba3e-b78564f2abde}
% !TeX TXS-program:compile = txs:///pdflatex/[--shell-escape]
\documentclass[accentcolor=3c,landscape,ngerman,presentation,t,usenames,dvipsnames,svgnames,table]{tudabeamer}

% Template-Modifikationen
\addtobeamertemplate{frametitle}{}{\vspace{-1em}} % mehr Platz vor dem Inhalt

% andere global gemeinsame definitionen
%Includes
\usepackage[ngerman]{babel} %Deutsche Silbentrennung
\usepackage[utf8]{inputenc} %Deutsche Umlaute
\usepackage{float}
\usepackage{graphicx}
\usepackage{minted}
\RequirePackage{csquotes}
\RequirePackage{fontawesome5}

\DeclareGraphicsExtensions{.pdf,.png,.jpg}

\makeatletter
\author{Vorkursteam der Fachschaft Informatik}
\let\Author\@author

% dark mode
\ExplSyntaxOn
\IfDarkModeT{
    \cs_if_exist:NT \setbeamercolor {
        \setbeamercolor*{smallrule}{bg=.}
        \setbeamercolor*{normal~text}{bg=\thepagecolor,fg=.}
        \setbeamercolor*{background~canvas}{parent=normal~text}
        \setbeamercolor*{section~in~toc}{parent=normal~text}
        \setbeamercolor*{subsection~in~toc}{parent=normal~text,fg=.}
        \setbeamercolor*{footline}{parent=normal~text}
        \setbeamercolor{block~title~alerted}{fg=white,bg=white!20!\thepagecolor}
        \setbeamercolor*{block~body}{bg=black!70!gray!98!blue}
        \setbeamercolor*{block~body~alerted}{bg=\thepagecolor}
    }
    \cs_if_exist:NT \setbeamertemplate {
        \setbeamertemplate{subsection~in~toc~shaded}[default][50]
    }
}
\ExplSyntaxOff

% macros
\renewcommand{\arraystretch}{1.2} % Höhe einer Tabellenspalte minimal erhöhen
\newcommand{\N}{{\mathbb N}}
\renewcommand{\code}{\inputminted[]{python}}

\IfDarkModeTF{
    \newmintedfile[pythonfile]{python}{
        fontsize=\small,
        style=native,
        linenos=true,
        numberblanklines=true,
        tabsize=4,
        obeytabs=false,
        breaklines=true,
        autogobble=true,
        encoding="utf8",
        showspaces=false,
        xleftmargin=20pt,
        frame=single,
        framesep=5pt,
    }
    \newmintinline{python}{
        style=native,
        encoding="utf8"
    }
    \newmintinline{kotlin}{
        style=native,
        encoding="utf8"
    }


    \definecolor{codegray}{HTML}{eaf1ff}
    \newminted[bashcode]{awk}{
        escapeinside=||,
        fontsize=\small,
        style=native,
        linenos=true,
        numberblanklines=true,
        tabsize=4,
        obeytabs=false,
        breaklines=true,
        autogobble=true,
        encoding="utf8",
        showspaces=false,
        xleftmargin=20pt,
        frame=single,
        framesep=5pt
    }
}{
    \newmintedfile[pythonfile]{python}{
        fontsize=\small,
        style=friendly,
        linenos=true,
        numberblanklines=true,
        tabsize=4,
        obeytabs=false,
        breaklines=true,
        autogobble=true,
        encoding="utf8",
        showspaces=false,
        xleftmargin=20pt,
        frame=single,
        framesep=5pt,
    }
    \newmintinline{python}{
        style=friendly,
        encoding="utf8"
    }
    \newmintinline{kotlin}{
        style=friendly,
        encoding="utf8"
    }

    \definecolor{codegray}{HTML}{eaf1ff}
    \newminted[bashcode]{awk}{
        escapeinside=||,
        fontsize=\small,
        style=friendly,
        linenos=true,
        numberblanklines=true,
        tabsize=4,
        obeytabs=false,
        breaklines=true,
        autogobble=true,
        encoding="utf8",
        showspaces=false,
        xleftmargin=20pt,
        frame=single,
        framesep=5pt
    }
}

\let\origpythonfile\pythonfile
\renewcommand{\pythonfile}[1]{\pythonfileh{#1}{}}
\newcommand{\pythonfileh}[2]{\origpythonfile[#2]{#1}}

\DeclareDocumentCommand{\kotlinfile}{O{} O{} m}{\inputCode[#1]{minted language=kotlin,#2}{#3}}

\newcommand*{\ditto}{\texttt{\char`\"}}

\newcommand{\shellprefix}{\textcolor{TUDa-3a}{\ttfamily\bfseries \$~}}
\DeclareTCBListing{commandshell}{ O{} O{} }{
    colback=\IfDarkModeTF{black}{black!80},
    colupper=white,
    colframe=TUDa-3a,
    listing only,
    % listing options={style=tcblatex,language=sh},
    listing engine=minted,
    minted style=dracula,
    minted options={
        % linenos=true,
        numbersep=3mm,
        texcl=true,
        autogobble,
        escapeinside=@@,
        breaklines,
        highlightcolor=yellow!50!black,
        #1
    },
    #2,
    % before upper={\textcolor{red}{\small\ttfamily\bfseries root \$> }},
    % every listing line={\textcolor{red}{\small\ttfamily\bfseries root \$> }}
}

%Includes
\usepackage{epstopdf}
\usepackage{wrapfig}
\usepackage{tipa}
\usepackage{tikz}
\usetikzlibrary{calc,shapes,arrows}
%tip: use http://l04.scarfboy.com/coding/phonetic-translation?from=ipa&fromtext=%CB%88pa%C9%AA%CE%B8n%CC%A9&to=tipa
%for converting ipa


\graphicspath{ {./media/} }

\def\shortyear#1{\expandafter\shortyearhelper#1}
\def\shortyearhelper#1#2#3#4{#3#4}

\newcount\NextYear
\NextYear = \year
\advance\NextYear by 1

\newcommand\NextYearShort{\shortyear{\the\NextYear}}

% notes
\usepackage{pgfpages}
\setbeamertemplate{note page}[plain]
%\setbeameroption{show notes on second screen}

% macro for change speaker sign
\newcommand{\changespeaker}{
	\begin{tikzpicture}[line width=.6mm, shorten >= 3pt, shorten <= 3pt]

	\coordinate (c1);
	\coordinate[right of=c1] (c2);

	\draw[rectangle, draw=red!80, fill=red!80, align=center, rounded corners] ($(c1.north west)+(0,-0.3)$) rectangle ($(c2.south east)+(0, 0.3)$) {};
	\draw[->,white] (c1)[bend left] to node[auto] {} (c2);
	\draw[->,white] (c2)[bend left] to node[auto] {} (c1);
	\end{tikzpicture}
}

%Listing-Style pyhon
\title[Programmiervorkurs]{Programmiervorkurs Wintersemester \the\year/\NextYearShort}
\subtitle{{\small der Fachschaft Informatik}}
\logo*{\includegraphics{../globalMedia/bildmarke_ohne_rand}}
\institute{Fachschaft Informatik}
\date{Wintersemester \the\year/\NextYearShort}


% macros
\newcommand{\livecoding}{\begin{frame}\frametitle{\insertsectionhead \\  {\small \insertsubsectionhead}}\centering \huge \vskip 2cm\textbf{\textcolor{red}{Live-Coding}}\end{frame}}

%\newcommand{\slidehead}{\frametitle{\insertsectionhead \\ {\small \insertsubsectionhead}}\vspace{3mm}}
\newcommand{\slidehead}{\frametitle{\insertsectionhead} \framesubtitle{\insertsubsectionhead}\vspace{3mm}}
\newcommand{\tocslide}{\begin{frame}[t]\frametitle{Inhaltsverzeichnis}\vspace{3mm}{\small\tableofcontents[subsectionstyle=shaded]}\end{frame}}


% colors
\definecolor{lightpetrol}{RGB}{0,223,194}


\begin{document}

%Deckblatt

\section{Schleifen}
\subtitle{Kapitel 4: Befehle wiederholen}
\titlegraphic{
    \begin{columns}
        \begin{column}{4cm}
            \vspace{1.5cm}
            \begin{center}
                {\huge Schleifen}
            \end{center}
        \end{column}
        \begin{column}{4cm}
            \vspace{2mm}
            \begin{figure}
                \centering
                \includeinvertablegraphics[scale=.5]{media/flowchart.png}
                \\	\sffamily \tiny Bild: \href{https://xkcd.com/1195/}{https://xkcd.com/1195/}
            \end{figure}
        \end{column}
    \end{columns}}
\maketitle


\section{Wofür Schleifen?}
\begin{frame}
    \slidehead

    \kotlinfile{listings/why.kts}
\end{frame}

\begin{frame}
    \slidehead

    \kotlinfile{listings/why2.kts}
    \pause
    \begin{block}{Hinweis}
        Sei faul!
    \end{block}
\end{frame}

\begin{frame}
    \slidehead

    \kotlinfile{listings/while.kts}
    \begin{itemize}
        \item Code wird \textit{kürzer}
        \item Code wird \textit{lesbarer}
        \item Anzahl der Ausführungen kann \textit{dynamisch} entschieden werden
    \end{itemize}
\end{frame}

%\nextvid{While-Schleife}{}

%Deckblatt
\section{While-Schleife}
\subtitle{Kapitel 4: Befehle wiederholen}

% \titlegraphic{
% 	\begin{columns}
% 		\begin{column}{4cm}
% 			\vspace{1.5cm}
% 			\begin{center}
% 				{\huge While-Schleife}
% 			\end{center}
% 		\end{column}
% 		\begin{column}{4cm}
% 		\vspace{2mm}
% 			\begin{figure}
% 				\centering
% 			\includeinvertablegraphics[scale=.4]{media/loop.png}
% 				\\	\sffamily \tiny Bild: \href{https://xkcd.com/1411/}{https://xkcd.com/1411/}
% 			\end{figure}
% 		\end{column}
% 	\end{columns}}
% \maketitle


\begin{frame}
    \slidehead

    \begin{itemize}
        \item Syntax:
            \kotlinfile{listings/while-syntax.kts}
        \item Solange die Bedingung zu \kotlininline{true} ausgewertet wird, werden Anweisungen wiederholt
        \item Zeile 1 wird Schleifenkopf genannt
        \item Zeile 2 und folgende werden Schleifenkörper genannt
    \end{itemize}
\end{frame}

\begin{frame}
    \slidehead
    \begin{itemize}
        \item Alle logischen Operatoren können verwendet werden:
            \kotlinfile{listings/while2.kts}
    \end{itemize}
\end{frame}

\livecoding

\subsection{Continue und break}
\begin{frame}
    \slidehead

    \begin{itemize}
        \item \kotlininline{continue} springt zum Anfang der Schleife
        \item \kotlininline{break} bricht die Schleife komplett ab
    \end{itemize}
    \kotlinfile{listings/continue-break.kts}
\end{frame}

\livecoding

%\nextvid{For-Schleife}{While-Schleife Beispiel}

%Deckblatt
\section{For-Schleife}
\subtitle{Kapitel 4: Befehle wiederholen}

% \titlegraphic{
% 	\begin{columns}
% 		\begin{column}{4cm}
% 			\vspace{1.5cm}
% 			\begin{center}
% 				{\huge For-Schleife}
% 			\end{center}
% 		\end{column}
% 		\begin{column}{5cm}
% 		\vspace{2mm}
% 			\begin{figure}
% 				\centering
% 			\includeinvertablegraphics[scale=.35]{media/delicious.png}
% 				\\	\sffamily \tiny Bild: \href{https://xkcd.com/140/}{https://xkcd.com/140/}
% 			\end{figure}
% 		\end{column}
% 	\end{columns}}
% \maketitle

\section{For-Schleife}
\begin{frame}
    \slidehead

    \begin{itemize}
        \vspace{-0.1cm}
        \item Häufig gibt es folgende Schleifenart:
            \kotlinfile{listings/while3.kts}
            \pause
        \item Kurzschreibweise:
            \kotlinfile{listings/forschleife.kts}
    \end{itemize}
    \begin{block}{Frage:}
        Von wo bis wo zählt diese Schleife?
    \end{block}
\end{frame}

\begin{frame}
    \slidehead

    \begin{itemize}
        \item Sinnvoll, wenn eine Variable bis zu einem bestimmten Wert \textit{hochzählen} soll
        \item Syntax:
            \kotlinfile{listings/forschleife.kts}
        \item Beispiel:
            \kotlinfile{listings/forschleife-print.kts}
    \end{itemize}
\end{frame}

\begin{frame}
    \slidehead

    \begin{itemize}
        \item Ein range kann auch bei einer bestimmten Zahl anfangen
            \begin{itemize}
                \item Schreibweise: \kotlininline{startwert until grenzwert}
            \end{itemize}
        \item Oder mit einer festen Schrittweite arbeiten:
            \begin{itemize}
                \item Schreibweise: \kotlininline{startwert until grenzwert step schrittweite}
            \end{itemize}
            Beispiel:
            \kotlinfile{listings/range.kts}
    \end{itemize}
\end{frame}

\subsection{Verschachtelte Schleifen}
\begin{frame}
    \slidehead

    \begin{itemize}
        \item Natürlich lassen sich Schleifen auch \textit{verschachteln}:
            \vspace{0.1cm}
            \kotlinfile{listings/geschachteltfor.kts}
    \end{itemize}

    \begin{block}{Frage: Was wird ausgegeben?}
        \pause
        0000000000\\
        0000000000\\
        ...\\
        0000000000\\
    \end{block}
\end{frame}

\livecoding

\section{Quiz}
\begin{frame}
    \slidehead
    \pause
    \begin{itemize}
        \item Welche Schleifenarten gibt es in Kotlin?\pause
        \item Was \textbf{muss} mit Code in einer Schleife gemacht werden?\pause
        \item Wann ist die Verwendung einer \kotlininline{for}-Schleife sinnvoll?\pause
        \item Ist es möglich, \kotlininline{while}-Schleifen in \kotlininline{for}-Schleifen zu verschachteln?\pause
    \end{itemize}
    \begin{block}{Finde drei Fehler}
        \kotlinfile{listings/FrageStunde_Schleife_ErrorListing.kts}
    \end{block}
    % while in range (Syntaxfehler)
    % : fehlt (Syntaxfehler)
    % schleife wird 10 und nicht 20 mal ausgeführt (Logikfehler)
\end{frame}

\section{Evaluation}
\begin{frame}
    \slidehead
    \begin{columns}
        \begin{column}{6cm}
            \vspace{-0.5cm}
            \begin{figure}
                \centering
                \includeinvertablegraphics[scale=.33]{media/QRCode_TLGAK_VL.png}
                \\  \sffamily \tiny Vorlesungsevaluation \href{http://evaluation.tu-darmstadt.de/evasys/online.php?p=TLGAK}{http://evaluation.tu-darmstadt.de/evasys/online.php?p=TLGAK}
            \end{figure}
        \end{column}
        \begin{column}{6cm}
            \vspace{-0.5cm}
            \begin{figure}
                \centering
                \includeinvertablegraphics[scale=.33]{media/QRCode_Q5ZNR_UE.png}
                \\  \sffamily \tiny Übungsevaluation: \href{http://evaluation.tu-darmstadt.de/evasys/online.php?p=Q5ZNR}{http://evaluation.tu-darmstadt.de/evasys/online.php?p=Q5ZNR}
            \end{figure}
        \end{column}
    \end{columns}
\end{frame}


%\nextvid{Listen}{For-Schleife Beispiele}

\end{document}
