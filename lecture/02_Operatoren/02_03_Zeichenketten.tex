% !TeX document-id = {ecbf6175-be07-4601-b8eb-d3263d2bff07}
% !TeX TXS-program:compile = txs:///pdflatex/[--shell-escape]
\documentclass[accentcolor=3c,landscape,ngerman,presentation,t,usenames,dvipsnames,svgnames,table]{tudabeamer}

% Template-Modifikationen
\addtobeamertemplate{frametitle}{}{\vspace{-1em}} % mehr Platz vor dem Inhalt

% andere global gemeinsame definitionen
%Includes
\usepackage[ngerman]{babel} %Deutsche Silbentrennung
\usepackage[utf8]{inputenc} %Deutsche Umlaute
\usepackage{float}
\usepackage{graphicx}
\usepackage{minted}
\RequirePackage{csquotes}
\RequirePackage{fontawesome5}

\DeclareGraphicsExtensions{.pdf,.png,.jpg}

\makeatletter
\author{Vorkursteam der Fachschaft Informatik}
\let\Author\@author

% dark mode
\ExplSyntaxOn
\IfDarkModeT{
    \cs_if_exist:NT \setbeamercolor {
        \setbeamercolor*{smallrule}{bg=.}
        \setbeamercolor*{normal~text}{bg=\thepagecolor,fg=.}
        \setbeamercolor*{background~canvas}{parent=normal~text}
        \setbeamercolor*{section~in~toc}{parent=normal~text}
        \setbeamercolor*{subsection~in~toc}{parent=normal~text,fg=.}
        \setbeamercolor*{footline}{parent=normal~text}
        \setbeamercolor{block~title~alerted}{fg=white,bg=white!20!\thepagecolor}
        \setbeamercolor*{block~body}{bg=black!70!gray!98!blue}
        \setbeamercolor*{block~body~alerted}{bg=\thepagecolor}
    }
    \cs_if_exist:NT \setbeamertemplate {
        \setbeamertemplate{subsection~in~toc~shaded}[default][50]
    }
}
\ExplSyntaxOff

% macros
\renewcommand{\arraystretch}{1.2} % Höhe einer Tabellenspalte minimal erhöhen
\newcommand{\N}{{\mathbb N}}
\renewcommand{\code}{\inputminted[]{python}}

\IfDarkModeTF{
    \newmintedfile[pythonfile]{python}{
        fontsize=\small,
        style=native,
        linenos=true,
        numberblanklines=true,
        tabsize=4,
        obeytabs=false,
        breaklines=true,
        autogobble=true,
        encoding="utf8",
        showspaces=false,
        xleftmargin=20pt,
        frame=single,
        framesep=5pt,
    }
    \newmintinline{python}{
        style=native,
        encoding="utf8"
    }
    \newmintinline{kotlin}{
        style=native,
        encoding="utf8"
    }


    \definecolor{codegray}{HTML}{eaf1ff}
    \newminted[bashcode]{awk}{
        escapeinside=||,
        fontsize=\small,
        style=native,
        linenos=true,
        numberblanklines=true,
        tabsize=4,
        obeytabs=false,
        breaklines=true,
        autogobble=true,
        encoding="utf8",
        showspaces=false,
        xleftmargin=20pt,
        frame=single,
        framesep=5pt
    }
}{
    \newmintedfile[pythonfile]{python}{
        fontsize=\small,
        style=friendly,
        linenos=true,
        numberblanklines=true,
        tabsize=4,
        obeytabs=false,
        breaklines=true,
        autogobble=true,
        encoding="utf8",
        showspaces=false,
        xleftmargin=20pt,
        frame=single,
        framesep=5pt,
    }
    \newmintinline{python}{
        style=friendly,
        encoding="utf8"
    }
    \newmintinline{kotlin}{
        style=friendly,
        encoding="utf8"
    }

    \definecolor{codegray}{HTML}{eaf1ff}
    \newminted[bashcode]{awk}{
        escapeinside=||,
        fontsize=\small,
        style=friendly,
        linenos=true,
        numberblanklines=true,
        tabsize=4,
        obeytabs=false,
        breaklines=true,
        autogobble=true,
        encoding="utf8",
        showspaces=false,
        xleftmargin=20pt,
        frame=single,
        framesep=5pt
    }
}

\let\origpythonfile\pythonfile
\renewcommand{\pythonfile}[1]{\pythonfileh{#1}{}}
\newcommand{\pythonfileh}[2]{\origpythonfile[#2]{#1}}

\DeclareDocumentCommand{\kotlinfile}{O{} O{} m}{\inputCode[#1]{minted language=kotlin,#2}{#3}}

\newcommand*{\ditto}{\texttt{\char`\"}}

\newcommand{\shellprefix}{\textcolor{TUDa-3a}{\ttfamily\bfseries \$~}}
\DeclareTCBListing{commandshell}{ O{} O{} }{
    colback=\IfDarkModeTF{black}{black!80},
    colupper=white,
    colframe=TUDa-3a,
    listing only,
    % listing options={style=tcblatex,language=sh},
    listing engine=minted,
    minted style=dracula,
    minted options={
        % linenos=true,
        numbersep=3mm,
        texcl=true,
        autogobble,
        escapeinside=@@,
        breaklines,
        highlightcolor=yellow!50!black,
        #1
    },
    #2,
    % before upper={\textcolor{red}{\small\ttfamily\bfseries root \$> }},
    % every listing line={\textcolor{red}{\small\ttfamily\bfseries root \$> }}
}

%Includes
\usepackage{epstopdf}
\usepackage{wrapfig}
\usepackage{tipa}
\usepackage{tikz}
\usetikzlibrary{calc,shapes,arrows}
%tip: use http://l04.scarfboy.com/coding/phonetic-translation?from=ipa&fromtext=%CB%88pa%C9%AA%CE%B8n%CC%A9&to=tipa
%for converting ipa


\graphicspath{ {./media/} }

\def\shortyear#1{\expandafter\shortyearhelper#1}
\def\shortyearhelper#1#2#3#4{#3#4}

\newcount\NextYear
\NextYear = \year
\advance\NextYear by 1

\newcommand\NextYearShort{\shortyear{\the\NextYear}}

% notes
\usepackage{pgfpages}
\setbeamertemplate{note page}[plain]
%\setbeameroption{show notes on second screen}

% macro for change speaker sign
\newcommand{\changespeaker}{
	\begin{tikzpicture}[line width=.6mm, shorten >= 3pt, shorten <= 3pt]

	\coordinate (c1);
	\coordinate[right of=c1] (c2);

	\draw[rectangle, draw=red!80, fill=red!80, align=center, rounded corners] ($(c1.north west)+(0,-0.3)$) rectangle ($(c2.south east)+(0, 0.3)$) {};
	\draw[->,white] (c1)[bend left] to node[auto] {} (c2);
	\draw[->,white] (c2)[bend left] to node[auto] {} (c1);
	\end{tikzpicture}
}

%Listing-Style pyhon
\title[Programmiervorkurs]{Programmiervorkurs Wintersemester \the\year/\NextYearShort}
\subtitle{{\small der Fachschaft Informatik}}
\logo*{\includegraphics{../globalMedia/bildmarke_ohne_rand}}
\institute{Fachschaft Informatik}
\date{Wintersemester \the\year/\NextYearShort}


% macros
\newcommand{\livecoding}{\begin{frame}\frametitle{\insertsectionhead \\  {\small \insertsubsectionhead}}\centering \huge \vskip 2cm\textbf{\textcolor{red}{Live-Coding}}\end{frame}}

%\newcommand{\slidehead}{\frametitle{\insertsectionhead \\ {\small \insertsubsectionhead}}\vspace{3mm}}
\newcommand{\slidehead}{\frametitle{\insertsectionhead} \framesubtitle{\insertsubsectionhead}\vspace{3mm}}
\newcommand{\tocslide}{\begin{frame}[t]\frametitle{Inhaltsverzeichnis}\vspace{3mm}{\small\tableofcontents[subsectionstyle=shaded]}\end{frame}}


% colors
\definecolor{lightpetrol}{RGB}{0,223,194}


\begin{document}

%Deckblatt
\subtitle{Kabitel 2: Spaß mit Daten}
\titlegraphic{
	\begin{columns}
		\begin{column}{10cm}
			\begin{center}
				{\huge Zeichenketten}
			\end{center}
			\vspace{-1mm}
			\begin{figure}
				\centering
			\includegraphics[scale=.35]{media/mu.png}
				\\	\sffamily \tiny Bild: \href{https://xkcd.com/815/}{https://xkcd.com/815}
				\\
				(Ich habe keinen passenden xkcd gefunden. Schlagt gerne welche vor.)
			\end{figure}
		\end{column}
	\end{columns}}
\maketitle

	\subsection{}
	\section{Zeichenketten}

	\begin{frame}
		\slidehead
		\begin{itemize}
			\item Habt ihr vorhin kennengelernt
			\pause
			\item Zeichenketten werden als \textit{Strings} bezeichnet
			\item Ein String kann beliebig viele Zeichen enthalten
			\pause
			\item Strings können mit \textbf{+} konkateniert werden
		\end{itemize}
	\end{frame}


	%Vielleicht später mal nützlich:

	%\begin{frame}
	%	\slidehead
	%	\begin{itemize}
	%		\item String Slicing (Zerschneiden von Text)
	%			\pythonfile{listings/String_Slicing_Listing.py}
	%		\begin{block}{Wichtig}
	%			\begin{itemize}
	%				\item \textbf{[von : bis]} ist \textbf{inklusive} "`von"' und \textbf{exklusive} "`bis"'
	%				\item Mathematisch: Ein Interval $[von, bis)$ bzw. $[von, bis[$
	%				\item Der erste Index beginnt mit der Zahl \textbf{0}
	%			\end{itemize}
	%		\end{block}
	%	\end{itemize}
	%\end{frame}

	%\begin{frame}
	%	\slidehead
	%	\begin{itemize}
	%		\item String Slicing Teil 2
	%			\pythonfile{listings/String_Slicing_Listing2.py}
	%		\begin{block}{Wichtig}
	%			\begin{itemize}
	%				\item \textbf{[: bis]} ist \textbf{exklusive} und \textbf{[von :]} ist \textbf{inklusive} der Grenze
	%				\item Auch hier beginnt der erste Index mit der Nummer \textbf{0}
	%				\item Es können auch beide Grenzen weggelassen werden
	%			\end{itemize}
	%		\end{block}
	%	\end{itemize}
	%\end{frame}

	\subsection{Funktionen und Operatoren}
	\begin{frame}
		\slidehead

		\begin{itemize}
			\item Länge eines Strings
			\pythonfile{listings/String_len_Listing.py}

			\item Umwandeln in Kleinbuchstaben
			\pythonfile{listings/String_lower_Listing.py}

			\item Umwandeln in Großbuchstaben
			\pythonfile{listings/String_upper_Listing.py}
		\end{itemize}
	\end{frame}


	\subsection{Konvertierung}
	\begin{frame}
		\slidehead
		\begin{block}{Warnung}
			Strings sind keine Zahlen
		\end{block}

		\pythonfile{listings/String_is_not_a_number_Listing.py}

		\pause

		\begin{itemize}
			\item Strings müssen zum Rechnen konvertiert werden
			\item \textbf{int(x)} konvertiert x zu int
			\item \textbf{float(x)} konvertiert x zu float
%			\item \textbf{bool(x)} konvertiert x zu boolean  % Das lassen wir weg, weil es durch leere Zeichenketten und Zahlen zu mehr Verwirrung sorgt als es hilfreich ist
			\item \textbf{str(x)} konvertiert x zu String
		\end{itemize}

	\end{frame}

\section{Konvertierung}
\subsection{}
	\begin{frame}
		\slidehead
		\begin{block}{Beispiel}
			\pythonfile{listings/String_conversion_Listing.py}
		\end{block}
	\end{frame}


\nextvid{Variablen}{Zeichenketten Beispiel}

	\end{document}
