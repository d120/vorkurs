% !TeX document-id = {ecbf6175-be07-4601-b8eb-d3263d2bff07}
% !TeX TXS-program:compile = txs:///pdflatex/[--shell-escape]
\documentclass[accentcolor=3c,landscape,ngerman,presentation,t,usenames,dvipsnames,svgnames,table]{tudabeamer}

% Template-Modifikationen
\addtobeamertemplate{frametitle}{}{\vspace{-1em}} % mehr Platz vor dem Inhalt

% andere global gemeinsame definitionen
%Includes
\usepackage[ngerman]{babel} %Deutsche Silbentrennung
\usepackage[utf8]{inputenc} %Deutsche Umlaute
\usepackage{float}
\usepackage{graphicx}
\usepackage{minted}
\RequirePackage{csquotes}
\RequirePackage{fontawesome5}

\DeclareGraphicsExtensions{.pdf,.png,.jpg}

\makeatletter
\author{Vorkursteam der Fachschaft Informatik}
\let\Author\@author

% dark mode
\ExplSyntaxOn
\IfDarkModeT{
    \cs_if_exist:NT \setbeamercolor {
        \setbeamercolor*{smallrule}{bg=.}
        \setbeamercolor*{normal~text}{bg=\thepagecolor,fg=.}
        \setbeamercolor*{background~canvas}{parent=normal~text}
        \setbeamercolor*{section~in~toc}{parent=normal~text}
        \setbeamercolor*{subsection~in~toc}{parent=normal~text,fg=.}
        \setbeamercolor*{footline}{parent=normal~text}
        \setbeamercolor{block~title~alerted}{fg=white,bg=white!20!\thepagecolor}
        \setbeamercolor*{block~body}{bg=black!70!gray!98!blue}
        \setbeamercolor*{block~body~alerted}{bg=\thepagecolor}
    }
    \cs_if_exist:NT \setbeamertemplate {
        \setbeamertemplate{subsection~in~toc~shaded}[default][50]
    }
}
\ExplSyntaxOff

% macros
\renewcommand{\arraystretch}{1.2} % Höhe einer Tabellenspalte minimal erhöhen
\newcommand{\N}{{\mathbb N}}
\renewcommand{\code}{\inputminted[]{python}}

\IfDarkModeTF{
    \newmintedfile[pythonfile]{python}{
        fontsize=\small,
        style=native,
        linenos=true,
        numberblanklines=true,
        tabsize=4,
        obeytabs=false,
        breaklines=true,
        autogobble=true,
        encoding="utf8",
        showspaces=false,
        xleftmargin=20pt,
        frame=single,
        framesep=5pt,
    }
    \newmintinline{python}{
        style=native,
        encoding="utf8"
    }
    \newmintinline{kotlin}{
        style=native,
        encoding="utf8"
    }


    \definecolor{codegray}{HTML}{eaf1ff}
    \newminted[bashcode]{awk}{
        escapeinside=||,
        fontsize=\small,
        style=native,
        linenos=true,
        numberblanklines=true,
        tabsize=4,
        obeytabs=false,
        breaklines=true,
        autogobble=true,
        encoding="utf8",
        showspaces=false,
        xleftmargin=20pt,
        frame=single,
        framesep=5pt
    }
}{
    \newmintedfile[pythonfile]{python}{
        fontsize=\small,
        style=friendly,
        linenos=true,
        numberblanklines=true,
        tabsize=4,
        obeytabs=false,
        breaklines=true,
        autogobble=true,
        encoding="utf8",
        showspaces=false,
        xleftmargin=20pt,
        frame=single,
        framesep=5pt,
    }
    \newmintinline{python}{
        style=friendly,
        encoding="utf8"
    }
    \newmintinline{kotlin}{
        style=friendly,
        encoding="utf8"
    }

    \definecolor{codegray}{HTML}{eaf1ff}
    \newminted[bashcode]{awk}{
        escapeinside=||,
        fontsize=\small,
        style=friendly,
        linenos=true,
        numberblanklines=true,
        tabsize=4,
        obeytabs=false,
        breaklines=true,
        autogobble=true,
        encoding="utf8",
        showspaces=false,
        xleftmargin=20pt,
        frame=single,
        framesep=5pt
    }
}

\let\origpythonfile\pythonfile
\renewcommand{\pythonfile}[1]{\pythonfileh{#1}{}}
\newcommand{\pythonfileh}[2]{\origpythonfile[#2]{#1}}

\DeclareDocumentCommand{\kotlinfile}{O{} O{} m}{\inputCode[#1]{minted language=kotlin,#2}{#3}}

\newcommand*{\ditto}{\texttt{\char`\"}}

\newcommand{\shellprefix}{\textcolor{TUDa-3a}{\ttfamily\bfseries \$~}}
\DeclareTCBListing{commandshell}{ O{} O{} }{
    colback=\IfDarkModeTF{black}{black!80},
    colupper=white,
    colframe=TUDa-3a,
    listing only,
    % listing options={style=tcblatex,language=sh},
    listing engine=minted,
    minted style=dracula,
    minted options={
        % linenos=true,
        numbersep=3mm,
        texcl=true,
        autogobble,
        escapeinside=@@,
        breaklines,
        highlightcolor=yellow!50!black,
        #1
    },
    #2,
    % before upper={\textcolor{red}{\small\ttfamily\bfseries root \$> }},
    % every listing line={\textcolor{red}{\small\ttfamily\bfseries root \$> }}
}

%Includes
\usepackage{epstopdf}
\usepackage{wrapfig}
\usepackage{tipa}
\usepackage{tikz}
\usetikzlibrary{calc,shapes,arrows}
%tip: use http://l04.scarfboy.com/coding/phonetic-translation?from=ipa&fromtext=%CB%88pa%C9%AA%CE%B8n%CC%A9&to=tipa
%for converting ipa


\graphicspath{ {./media/} }

\def\shortyear#1{\expandafter\shortyearhelper#1}
\def\shortyearhelper#1#2#3#4{#3#4}

\newcount\NextYear
\NextYear = \year
\advance\NextYear by 1

\newcommand\NextYearShort{\shortyear{\the\NextYear}}

% notes
\usepackage{pgfpages}
\setbeamertemplate{note page}[plain]
%\setbeameroption{show notes on second screen}

% macro for change speaker sign
\newcommand{\changespeaker}{
	\begin{tikzpicture}[line width=.6mm, shorten >= 3pt, shorten <= 3pt]

	\coordinate (c1);
	\coordinate[right of=c1] (c2);

	\draw[rectangle, draw=red!80, fill=red!80, align=center, rounded corners] ($(c1.north west)+(0,-0.3)$) rectangle ($(c2.south east)+(0, 0.3)$) {};
	\draw[->,white] (c1)[bend left] to node[auto] {} (c2);
	\draw[->,white] (c2)[bend left] to node[auto] {} (c1);
	\end{tikzpicture}
}

%Listing-Style pyhon
\title[Programmiervorkurs]{Programmiervorkurs Wintersemester \the\year/\NextYearShort}
\subtitle{{\small der Fachschaft Informatik}}
\logo*{\includegraphics{../globalMedia/bildmarke_ohne_rand}}
\institute{Fachschaft Informatik}
\date{Wintersemester \the\year/\NextYearShort}


% macros
\newcommand{\livecoding}{\begin{frame}\frametitle{\insertsectionhead \\  {\small \insertsubsectionhead}}\centering \huge \vskip 2cm\textbf{\textcolor{red}{Live-Coding}}\end{frame}}

%\newcommand{\slidehead}{\frametitle{\insertsectionhead \\ {\small \insertsubsectionhead}}\vspace{3mm}}
\newcommand{\slidehead}{\frametitle{\insertsectionhead} \framesubtitle{\insertsubsectionhead}\vspace{3mm}}
\newcommand{\tocslide}{\begin{frame}[t]\frametitle{Inhaltsverzeichnis}\vspace{3mm}{\small\tableofcontents[subsectionstyle=shaded]}\end{frame}}


% colors
\definecolor{lightpetrol}{RGB}{0,223,194}


\begin{document}

%Deckblatt
\subtitle{Kabitel 2: Spaß mit Daten}
\titlegraphic{
	\begin{columns}
		\begin{column}{10cm}
			\vspace{4.32mm}
			\begin{center}
				{\huge Operatoren}
			\end{center}
			\begin{figure}
				\centering
			\includeinvertablegraphics[scale=.35]{media/too_old_for_this_shit.png}
				\\	\sffamily \tiny Bild: \href{https://xkcd.com/447/}{https://xkcd.com/447/}
			\end{figure}
		\end{column}
	\end{columns}}
\maketitle

	\section{Operatoren}
	\begin{frame}
		\slidehead

		\begin{itemize}
			\item Symbol, das den \textbf{Interpreter} zur Ausführung einer Aktion veranlasst
			\item Operatortypen:
		\end{itemize}
		\tikzstyle{rect} = [rectangle, rounded corners, minimum width=3cm, minimum height=1cm,text centered, draw=., fill=\IfDarkModeTF{red!30!black}{red!10}]
		\centering
		\begin{tikzpicture}[node distance=1.5cm]
			\node (1) [rect, text width=5cm, minimum height=1cm] {\textbf{Mathematische Operatoren}};
			\node (2) [rect, below of=1, text width=5cm, minimum height=1cm] {\textbf{Vergleichsoperatoren}};
			\node (3) [rect, below of=2, text width=5cm, minimum height=1cm] {\textbf{Bitoperatoren}};
		\end{tikzpicture}
	\end{frame}

	\subsection{Mathematische Operatoren}
	\begin{frame}
		\slidehead
		\vspace{-0.2cm}
		\tikzstyle{rect} = [rectangle, rounded corners, minimum width=3cm, minimum height=1cm,text centered, draw=., fill=\IfDarkModeTF{red!30!black}{red!10}]
		\vspace{0.05cm}
		\centering
		\begin{tikzpicture}[node distance=1.1cm]
			\node (1) [rect, text width=3cm, minimum height=1cm] {Addition: $+$};
			\node (2) [rect, below of=1, text width=3cm, minimum height=1cm] {Subtraktion: $-$};
			\node (3) [rect, below of=2, text width=3cm, minimum height=1cm] {Multiplikation: $*$};
			\node (4) [rect, node distance=3.5cm, right of=1, text width=3cm, minimum height=1cm] {Division: $/$};
			\node (5) [rect, below of=4, text width=3cm, minimum height=1cm] {Exponentiation: $**$};
			\node (6) [rect, below of=5, text width=3cm, minimum height=1cm] {Modulo: $\%$};
		\end{tikzpicture}
		\pause
		\vspace{-0.1cm}
		\begin{block}{Hinweis}
			\begin{itemize}
				\item Modulo liefert den \textit{Rest einer Division}: $23 \% 4 = 3$
				\item Arithmetische Operationen zwischen Ganzzahlen und Fließkommazahlen resultieren in \textit{Fließkommazahlen}! \pythoninline{1 + 2.0 = 3.0} % Für die*den Programmieranfänger*innen mag es zwar total normal scheinen, dass man Kommazahlen mit Integers addieren kann, aber für den Computer ist das ein wichtiger Unterschied! Jetzt kommt der Moment wo ich alle überzeugen kann, dass der Computer wirklich komplett anders denkt als wir Menschen. Er hat ein komplett anderes Konzept sich 0.1 vorzustellen.

				%Umfrage, glaubt ihr Computer können rechnen?
				\item \pythoninline{0.1 + 0.2 =} \pause \pythoninline{0.30000000000000004} $\Rightarrow$ Fließkommazahlen sind nicht exakt
			\end{itemize}
		\end{block}
	\end{frame}


	\subsection{Vergleichsoperatoren}
	\begin{frame}
		\slidehead

		\begin{itemize}
			\item Vergleichsoperatoren werten zu einem Wert vom Typ \pythoninline{bool} aus
			\item Beispiel:
			\pythonfile{listings/vergleich_Listing.py}
		\end{itemize}
	\end{frame}


	\subsection{Vergleichsoperatoren - Übersicht}
	\begin{frame}
		\slidehead

		\begin{center}
			\vskip -7 pt
			\begingroup
			\fontsize{10pt}{11pt}\selectfont
			\begin{tabular}{|l|c|c|c|}
				\hline
				\textbf{Operator} & \textbf{Symbol} & \textbf{Beispiel} & \textbf{Rückgabewert} \\ \hline
				gleich 					& == 	& 100 == 50; 	& False \\
				&  		& 50 == 50; 	& True \\ \hline
				ungleich 				& != 	& 100 != 50; 	& True \\
				&  		& 50 != 50 		& False \\ \hline
				größer als 				& > 	& 100 > 50; 	& True \\
				&  		& 50 > 50; 		& False \\ \hline
				größer als oder gleich 	& >= 	& 100 >= 50; 	& True \\
				& 		& 50 >= 50 		& True \\ \hline
				kleiner als 			& < 	& 100 < 50; 	& False \\
				& 		& 50 < 50; 		& False \\ \hline
				kleiner als oder gleich & <= 	& 100 <= 50; 	& False \\
				&  		& 50 <= 50; 	& True \\ \hline
			\end{tabular}
			\endgroup
		\end{center}

		\note{
		== geht auch für Strings!
		}
	\end{frame}

	\subsection{Bitoperatoren}
	\begin{frame}
		\slidehead
		Ja, es gibt auch Bitoperatoren ...
	\end{frame}

	\livecoding

	\subsection{Operatorpräzedenz}
	\begin{frame}
		\slidehead

		\pythonfile{listings/Rangfolge_Listing.py}
		\pause
		\pythonfile{listings/Rangfolge_Listing_parenthesis.py}

		\begin{itemize}
			\item Die Operatorpräzedenz gibt die \textbf{Reihenfolge} vor, in der Ausdrücke ausgewertet werden:
			\begin{enumerate}
				\item Arithmetische Ausdrücke
				\item Vergleiche
				\item Logische Ausdrücke (hierzu später mehr)
				\item Zuweisungen
			\end{enumerate}
		\end{itemize}
	\end{frame}

	\nextvid{Zeichenketten}{Operatoren Beispiel}
	\end{document}
