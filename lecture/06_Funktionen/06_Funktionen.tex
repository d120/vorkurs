% !TeX document-id = {d2a98428-ee0e-4def-9f4b-378c042e42ba}
% !TeX TXS-program:compile = txs:///pdflatex/[--shell-escape]
% !TeX root = 06_Funktionen.tex
\documentclass[accentcolor=3c,landscape,ngerman,presentation,t,usenames,dvipsnames,svgnames,table]{tudabeamer}

% Template-Modifikationen
\addtobeamertemplate{frametitle}{}{\vspace{-1em}} % mehr Platz vor dem Inhalt

% andere global gemeinsame definitionen
%Includes
\usepackage[ngerman]{babel} %Deutsche Silbentrennung
\usepackage[utf8]{inputenc} %Deutsche Umlaute
\usepackage{float}
\usepackage{graphicx}
\usepackage{minted}
\RequirePackage{csquotes}
\RequirePackage{fontawesome5}

\DeclareGraphicsExtensions{.pdf,.png,.jpg}

\makeatletter
\author{Vorkursteam der Fachschaft Informatik}
\let\Author\@author

% dark mode
\ExplSyntaxOn
\IfDarkModeT{
    \cs_if_exist:NT \setbeamercolor {
        \setbeamercolor*{smallrule}{bg=.}
        \setbeamercolor*{normal~text}{bg=\thepagecolor,fg=.}
        \setbeamercolor*{background~canvas}{parent=normal~text}
        \setbeamercolor*{section~in~toc}{parent=normal~text}
        \setbeamercolor*{subsection~in~toc}{parent=normal~text,fg=.}
        \setbeamercolor*{footline}{parent=normal~text}
        \setbeamercolor{block~title~alerted}{fg=white,bg=white!20!\thepagecolor}
        \setbeamercolor*{block~body}{bg=black!70!gray!98!blue}
        \setbeamercolor*{block~body~alerted}{bg=\thepagecolor}
    }
    \cs_if_exist:NT \setbeamertemplate {
        \setbeamertemplate{subsection~in~toc~shaded}[default][50]
    }
}
\ExplSyntaxOff

% macros
\renewcommand{\arraystretch}{1.2} % Höhe einer Tabellenspalte minimal erhöhen
\newcommand{\N}{{\mathbb N}}
\renewcommand{\code}{\inputminted[]{python}}

\IfDarkModeTF{
    \newmintedfile[pythonfile]{python}{
        fontsize=\small,
        style=native,
        linenos=true,
        numberblanklines=true,
        tabsize=4,
        obeytabs=false,
        breaklines=true,
        autogobble=true,
        encoding="utf8",
        showspaces=false,
        xleftmargin=20pt,
        frame=single,
        framesep=5pt,
    }
    \newmintinline{python}{
        style=native,
        encoding="utf8"
    }
    \newmintinline{kotlin}{
        style=native,
        encoding="utf8"
    }


    \definecolor{codegray}{HTML}{eaf1ff}
    \newminted[bashcode]{awk}{
        escapeinside=||,
        fontsize=\small,
        style=native,
        linenos=true,
        numberblanklines=true,
        tabsize=4,
        obeytabs=false,
        breaklines=true,
        autogobble=true,
        encoding="utf8",
        showspaces=false,
        xleftmargin=20pt,
        frame=single,
        framesep=5pt
    }
}{
    \newmintedfile[pythonfile]{python}{
        fontsize=\small,
        style=friendly,
        linenos=true,
        numberblanklines=true,
        tabsize=4,
        obeytabs=false,
        breaklines=true,
        autogobble=true,
        encoding="utf8",
        showspaces=false,
        xleftmargin=20pt,
        frame=single,
        framesep=5pt,
    }
    \newmintinline{python}{
        style=friendly,
        encoding="utf8"
    }
    \newmintinline{kotlin}{
        style=friendly,
        encoding="utf8"
    }

    \definecolor{codegray}{HTML}{eaf1ff}
    \newminted[bashcode]{awk}{
        escapeinside=||,
        fontsize=\small,
        style=friendly,
        linenos=true,
        numberblanklines=true,
        tabsize=4,
        obeytabs=false,
        breaklines=true,
        autogobble=true,
        encoding="utf8",
        showspaces=false,
        xleftmargin=20pt,
        frame=single,
        framesep=5pt
    }
}

\let\origpythonfile\pythonfile
\renewcommand{\pythonfile}[1]{\pythonfileh{#1}{}}
\newcommand{\pythonfileh}[2]{\origpythonfile[#2]{#1}}

\DeclareDocumentCommand{\kotlinfile}{O{} O{} m}{\inputCode[#1]{minted language=kotlin,#2}{#3}}

\newcommand*{\ditto}{\texttt{\char`\"}}

\newcommand{\shellprefix}{\textcolor{TUDa-3a}{\ttfamily\bfseries \$~}}
\DeclareTCBListing{commandshell}{ O{} O{} }{
    colback=\IfDarkModeTF{black}{black!80},
    colupper=white,
    colframe=TUDa-3a,
    listing only,
    % listing options={style=tcblatex,language=sh},
    listing engine=minted,
    minted style=dracula,
    minted options={
        % linenos=true,
        numbersep=3mm,
        texcl=true,
        autogobble,
        escapeinside=@@,
        breaklines,
        highlightcolor=yellow!50!black,
        #1
    },
    #2,
    % before upper={\textcolor{red}{\small\ttfamily\bfseries root \$> }},
    % every listing line={\textcolor{red}{\small\ttfamily\bfseries root \$> }}
}

%Includes
\usepackage{epstopdf}
\usepackage{wrapfig}
\usepackage{tipa}
\usepackage{tikz}
\usetikzlibrary{calc,shapes,arrows}
%tip: use http://l04.scarfboy.com/coding/phonetic-translation?from=ipa&fromtext=%CB%88pa%C9%AA%CE%B8n%CC%A9&to=tipa
%for converting ipa


\graphicspath{ {./media/} }

\def\shortyear#1{\expandafter\shortyearhelper#1}
\def\shortyearhelper#1#2#3#4{#3#4}

\newcount\NextYear
\NextYear = \year
\advance\NextYear by 1

\newcommand\NextYearShort{\shortyear{\the\NextYear}}

% notes
\usepackage{pgfpages}
\setbeamertemplate{note page}[plain]
%\setbeameroption{show notes on second screen}

% macro for change speaker sign
\newcommand{\changespeaker}{
	\begin{tikzpicture}[line width=.6mm, shorten >= 3pt, shorten <= 3pt]

	\coordinate (c1);
	\coordinate[right of=c1] (c2);

	\draw[rectangle, draw=red!80, fill=red!80, align=center, rounded corners] ($(c1.north west)+(0,-0.3)$) rectangle ($(c2.south east)+(0, 0.3)$) {};
	\draw[->,white] (c1)[bend left] to node[auto] {} (c2);
	\draw[->,white] (c2)[bend left] to node[auto] {} (c1);
	\end{tikzpicture}
}

%Listing-Style pyhon
\title[Programmiervorkurs]{Programmiervorkurs Wintersemester \the\year/\NextYearShort}
\subtitle{{\small der Fachschaft Informatik}}
\logo*{\includegraphics{../globalMedia/bildmarke_ohne_rand}}
\institute{Fachschaft Informatik}
\date{Wintersemester \the\year/\NextYearShort}


% macros
\newcommand{\livecoding}{\begin{frame}\frametitle{\insertsectionhead \\  {\small \insertsubsectionhead}}\centering \huge \vskip 2cm\textbf{\textcolor{red}{Live-Coding}}\end{frame}}

%\newcommand{\slidehead}{\frametitle{\insertsectionhead \\ {\small \insertsubsectionhead}}\vspace{3mm}}
\newcommand{\slidehead}{\frametitle{\insertsectionhead} \framesubtitle{\insertsubsectionhead}\vspace{3mm}}
\newcommand{\tocslide}{\begin{frame}[t]\frametitle{Inhaltsverzeichnis}\vspace{3mm}{\small\tableofcontents[subsectionstyle=shaded]}\end{frame}}


% colors
\definecolor{lightpetrol}{RGB}{0,223,194}



\begin{document}
%Deckblatt
\subtitle{Kapitel 6: wie Funktionen funktionieren}
\titlegraphic{
    \vbox to \height {%
        \vfill
        \begin{columns}
            \begin{column}{10cm}
                \begin{center}
                    {\huge Funktionen}
                \end{center}
                \begin{figure}
                    \centering
                    \includegraphics[scale=.4]{media/proofs\IfDarkModeT{-dark}.png}
                    \caption{\url{https://xkcd.com/1724/}}
                \end{figure}
            \end{column}
        \end{columns}
        \vfill
    }
}
\maketitle

\section{Schlechtes Beispiel}
\begin{frame}[c]
    \slidehead
    \pythonfile{listings/beispiel_schlecht.py}
    \small (Namen generiert auf www.listofrandomnames.com)
\end{frame}


\section{Gutes Beispiel}
\begin{frame}
    \slidehead
    \pythonfile{listings/beispiel_gut.py}
\end{frame}

\section{Idee von Funktionen}
\begin{frame}
    \slidehead

    \begin{itemize}
        \item Funktionalität \textbf{kapseln}: An einer Stelle zusammengefasst
        \item Weniger \textbf{Redundanz}: Keinen Code doppelt schreiben
        \item Einmal Code schreiben, danach nur an diese Stelle verweisen
    \end{itemize}
    \vspace{1cm}
    \begin{block}{Hinweis}
        Einige Funktionen haben wir schon kennengelernt, z.B. \pythoninline{print("Hello World")}
    \end{block}
\end{frame}

\subtitle{Kapitel 6: wie Funktionen funktionieren}

\section{Woraus besteht eine Funktion?}
\begin{frame}[c]
    \slidehead
    \begin{itemize}[<+->]
        \item Syntax: \pythonfile{listings/fun_syntax.py}
        \item Beispiel: \pythonfile{listings/fun_syntax_concrete.py}
    \end{itemize}
\end{frame}

\subsection{Return-Statement}
\begin{frame}
    \slidehead
    \begin{itemize}
        \item Funktionen können Rückgabewerte haben
        \item Der Rückgabetyp kann im Methodenkopf stehen
        \item Dazu wird \pythoninline{return} gefolgt vom Rückgabewert geschrieben
            \pause
        \item Syntax \pythonfile{listings/fun_syntax_return.py}
    \end{itemize}
\end{frame}

\livecoding

\subtitle{Kapitel 6: Wie Funktionen funktionieren}

\section{Fallbeispiel: Geometrie}
\begin{frame}[c]
    \slidehead

    \begin{figure}
        \centering
        %Hinweis: dieses Bild wurde von uns erstellt und kann ohne Referenz verwendet werden.
        %Viele Grüße
        %Kevin
        \includegraphics[width=0.8\textheight]{geometrie\IfDarkModeT{-dark}.png}
    \end{figure}

\end{frame}

\subsection{Flächeninhalt eines Kreises}
\begin{frame}
    \slidehead
    \pythonfile{listings/bsp_geometrie_schlecht.py}

    \begin{block}{Problem}
        Immer, wenn irgendwo der Flächeninhalt gebraucht wird, muss er neu geschrieben werden $\Rightarrow$ Viel Code und viele mögliche Fehlerquellen
    \end{block}
\end{frame}

\begin{frame}
    \slidehead
    \vspace{-1ex}
    \begin{onlyenv}<1>
        \pythonfile{listings/bsp_geometrie_1.py}
    \end{onlyenv}

    \begin{onlyenv}<2>
        \pythonfile{listings/bsp_geometrie_2.py}
    \end{onlyenv}
    \pause
    In diesem Beispiel, ist \pythoninline{r} ein \textbf{Parameter} und \pythoninline{radius} das \textbf{Argument} für \pythoninline{r}
\end{frame}

\subsection{Wiederverwendbarkeit}
\begin{frame}
    \slidehead
    \vspace{-1ex}
    \pythonfile[][top=0cm,bottom=0cm]{listings/reusability.py}
    \vspace{-1ex}
    \begin{block}{Hinweis}
        Nun kann die Funktion auch in anderen Funktionen verwendet werden.
        Bei Änderungen muss nur eine Stelle geändert werden.
    \end{block}
\end{frame}

\section{Vorteile von Funktionen}
\begin{frame}
    \slidehead

    \begin{itemize}
        \item \textbf{Wiederverwendung} von Code $\Rightarrow$ Weniger Code schreiben
        \item \textbf{Redundanz} im Code verringern $\Rightarrow$ Veränderungen nur an einer Stelle
        \item \textbf{Blackbox}-Prinzip $\Rightarrow$ Details verstecken
    \end{itemize}
\end{frame}

\livecoding

%Deckblatt
\subtitle{Kapitel 6: wie Funktionen funktionieren}

\section{Scope}
\subsection{Lokale Variablen}
\begin{frame}[fragile]
    \slidehead
    \vspace{-1em}
    \pythonfile[][top=0cm,bottom=0cm]{listings/scope_1.py}
    \pause
    \vspace{-1em}
    %\begin{noindent}
    \begin{commandshell}[fontsize=\footnotesize][minted language=text,top=0cm,bottom=0cm]
        Traceback (most recent call last):
        File "scope_1.py", line 5, in <module>
            print(variable)
                ^^^^^^^^
        NameError: name 'variable' is not defined. Did you mean: 'callable'?
    \end{commandshell}
    %\end{noindent}
    \vspace{-1em}
    \vfill
    \begin{block}{Hinweis:}
        Variablen, die innerhalb einer Funktion erstellt werden, werden gelöscht, sobald die Funktion endet.
        Diese Variablen heißen \textbf{lokale Variablen}.
    \end{block}
\end{frame}

\subsection{Globale Variablen}
\begin{frame}
    \slidehead
    \pythonfile{listings/scope_2.py}
    \begin{itemize}
        \item Variablen die außerhalb von Funktionen erstellt wurden, werden als \textbf{globale Variablen} bezeichnet
        \item Auf globale Variablen kann auch in Funktionen zugegriffen werden
    \end{itemize}
\end{frame}

\begin{frame}
    \slidehead
    \pythonfile{listings/scope_3.py}
    \begin{itemize}
        \item Die globale Variable \pythoninline{zahl} bleibt unverändert
        \item Es wird eine lokale Variable mit dem \textit{gleichen} Namen erstellt
    \end{itemize}
\end{frame}

\livecoding

\subtitle{Kapitel 6: wie Funktionen funktionieren}

\section{Rekursion}
\begin{frame}[t]
    \slidehead

    \vskip -18 pt
    \begin{columns}
        \begin{column}{4.5cm}
            \vskip 5 pt
            \begin{itemize}
                \item Funktionen können sich selbst aufrufen
            \end{itemize}
        \end{column}

        \begin{column}{7cm}
            \begin{figure}
                \IfFileExists{\jobname_tmp.pdf}{
                    \includegraphics[width=\textwidth,page=22]{\jobname_tmp.pdf}
                }{}
            \end{figure}
        \end{column}
    \end{columns}
\end{frame}

\begin{frame}[t]
    \slidehead

    \begin{itemize}
        \item Funktionen können sich mehrfach selbst aufrufen
    \end{itemize}

    \begin{columns}
        \begin{column}{5cm}
            \begin{figure}
                \IfFileExists{\jobname_tmp.pdf}{
                    \includegraphics[width=\textwidth,page=23]{\jobname_tmp.pdf}
                }{}
            \end{figure}
        \end{column}

        \begin{column}{5cm}
            \begin{figure}
                \IfFileExists{\jobname_tmp.pdf}{
                    \includegraphics[width=\textwidth,page=23]{\jobname_tmp.pdf}
                }{}
            \end{figure}
        \end{column}
    \end{columns}
\end{frame}

\subsection*{Beispiel}
\begin{frame}
    \slidehead

    \begin{itemize}
        \item Fibonacci-Zahlen: \texttt{1, 1, 2, 3, 5, 8, 13, 8+13 = 21, ...}
            \begin{onlyenv}<1>
                \pythonfile{listings/bsp_recursion_1.py}
            \end{onlyenv}

            \begin{onlyenv}<2>
                \pythonfile{listings/bsp_recursion_2.py}
            \end{onlyenv}

        \item<2> Rekursionsanker sorgt für den Abbruch
    \end{itemize}

\end{frame}

\subsection{Rekursionsbaum}

\begin{frame}[c]
    \slidehead
    \centering
    \begin{forest}
        % if you think this tree code is unreadable, just look at the previous one from Kevin
        % Fixed by: Ruben
        for tree={
            draw,
            very thick,
            edge={-Latex,thick},
            minimum size=.75cm,
            l+=5mm,
            s sep+=2cm,
            if n=1{ % honor the fancy edge style of the old tree
                edge path={
                    \noexpand\path[\forestoption{edge}]
                    (!u.south west) -- (.north east)\forestoption{edge label};
                },
            }{
                edge path={
                    \noexpand\path[\forestoption{edge}]
                    (!u.south east) -- (.north west)\forestoption{edge label};
                },
            },
        }
        [f(4)
            [f(3)
                [f(2)
                    [f(1)]
                    [f(0)]
                ]
                [f(1)]
            ]
            [f(2)
                [f(1)]
                [f(0)]
            ]
        ]
    \end{forest}
\end{frame}

\subsection{Rekursive Summe in Kotlin}
\begin{frame}
    \slidehead

    \begin{itemize}
        \item Aufgabe: Zahlen von 1 bis 4 zusammenzählen: \texttt{sum(4)} \pause
        \item \texttt{= 1 + 2 + 3 + 4} \pause
        \item \texttt{= (1 + 2 + 3) + 4} \pause
        \item \texttt{= sum(3) + 4}
    \end{itemize}
    \vspace{1em}
    \pause
    \pythonfile{listings/recursiveSum.py}
\end{frame}

\livecoding


\subsection{}

\section{Quiz}
\begin{frame}
    \slidehead
    \begin{itemize}
        \item Was bedeutet Kapseln?
            \pause
        \item Was ist ein Parameter?
            Was ist ein Argument?
            Wie untescheiden sie sich?
            \pause
        \item Darf ich meine Funktion \pythoninline{while} nennen?
            \pause
        \item Braucht eine Funktion ein \pythoninline{return}-Statement?
            \pause
        \item Solltet ihr alle zur Ophase gehen?
    \end{itemize}
\end{frame}

\section{Fragen und Sonstiges}
\begin{frame}
    \slidehead
    \vspace{1.8cm}
    \centering
    \huge Noch Fragen?
\end{frame}

\section{}
\subsection{}

\begin{frame}
    \slidehead
    \vspace{1.8cm}
    \centering
    \huge Danke fürs Teilnehmen und Viel Erfolg im Studium.
\end{frame}

\end{document}
