% !TeX document-id = {5c9b7af7-cc8b-495c-a61c-cd548e104791}
% !TeX TXS-program:compile = txs:///pdflatex/[--shell-escape]
\documentclass[accentcolor=3c,landscape,ngerman,presentation,t,usenames,dvipsnames,svgnames,table]{tudabeamer}

% Template-Modifikationen
\addtobeamertemplate{frametitle}{}{\vspace{-1em}} % mehr Platz vor dem Inhalt

% andere global gemeinsame definitionen
%Includes
\usepackage[ngerman]{babel} %Deutsche Silbentrennung
\usepackage[utf8]{inputenc} %Deutsche Umlaute
\usepackage{float}
\usepackage{graphicx}
\usepackage{minted}
\RequirePackage{csquotes}
\RequirePackage{fontawesome5}

\DeclareGraphicsExtensions{.pdf,.png,.jpg}

\makeatletter
\author{Vorkursteam der Fachschaft Informatik}
\let\Author\@author

% dark mode
\ExplSyntaxOn
\IfDarkModeT{
    \cs_if_exist:NT \setbeamercolor {
        \setbeamercolor*{smallrule}{bg=.}
        \setbeamercolor*{normal~text}{bg=\thepagecolor,fg=.}
        \setbeamercolor*{background~canvas}{parent=normal~text}
        \setbeamercolor*{section~in~toc}{parent=normal~text}
        \setbeamercolor*{subsection~in~toc}{parent=normal~text,fg=.}
        \setbeamercolor*{footline}{parent=normal~text}
        \setbeamercolor{block~title~alerted}{fg=white,bg=white!20!\thepagecolor}
        \setbeamercolor*{block~body}{bg=black!70!gray!98!blue}
        \setbeamercolor*{block~body~alerted}{bg=\thepagecolor}
    }
    \cs_if_exist:NT \setbeamertemplate {
        \setbeamertemplate{subsection~in~toc~shaded}[default][50]
    }
}
\ExplSyntaxOff

% macros
\renewcommand{\arraystretch}{1.2} % Höhe einer Tabellenspalte minimal erhöhen
\newcommand{\N}{{\mathbb N}}
\renewcommand{\code}{\inputminted[]{python}}

\IfDarkModeTF{
    \newmintedfile[pythonfile]{python}{
        fontsize=\small,
        style=native,
        linenos=true,
        numberblanklines=true,
        tabsize=4,
        obeytabs=false,
        breaklines=true,
        autogobble=true,
        encoding="utf8",
        showspaces=false,
        xleftmargin=20pt,
        frame=single,
        framesep=5pt,
    }
    \newmintinline{python}{
        style=native,
        encoding="utf8"
    }
    \newmintinline{kotlin}{
        style=native,
        encoding="utf8"
    }


    \definecolor{codegray}{HTML}{eaf1ff}
    \newminted[bashcode]{awk}{
        escapeinside=||,
        fontsize=\small,
        style=native,
        linenos=true,
        numberblanklines=true,
        tabsize=4,
        obeytabs=false,
        breaklines=true,
        autogobble=true,
        encoding="utf8",
        showspaces=false,
        xleftmargin=20pt,
        frame=single,
        framesep=5pt
    }
}{
    \newmintedfile[pythonfile]{python}{
        fontsize=\small,
        style=friendly,
        linenos=true,
        numberblanklines=true,
        tabsize=4,
        obeytabs=false,
        breaklines=true,
        autogobble=true,
        encoding="utf8",
        showspaces=false,
        xleftmargin=20pt,
        frame=single,
        framesep=5pt,
    }
    \newmintinline{python}{
        style=friendly,
        encoding="utf8"
    }
    \newmintinline{kotlin}{
        style=friendly,
        encoding="utf8"
    }

    \definecolor{codegray}{HTML}{eaf1ff}
    \newminted[bashcode]{awk}{
        escapeinside=||,
        fontsize=\small,
        style=friendly,
        linenos=true,
        numberblanklines=true,
        tabsize=4,
        obeytabs=false,
        breaklines=true,
        autogobble=true,
        encoding="utf8",
        showspaces=false,
        xleftmargin=20pt,
        frame=single,
        framesep=5pt
    }
}

\let\origpythonfile\pythonfile
\renewcommand{\pythonfile}[1]{\pythonfileh{#1}{}}
\newcommand{\pythonfileh}[2]{\origpythonfile[#2]{#1}}

\DeclareDocumentCommand{\kotlinfile}{O{} O{} m}{\inputCode[#1]{minted language=kotlin,#2}{#3}}

\newcommand*{\ditto}{\texttt{\char`\"}}

\newcommand{\shellprefix}{\textcolor{TUDa-3a}{\ttfamily\bfseries \$~}}
\DeclareTCBListing{commandshell}{ O{} O{} }{
    colback=\IfDarkModeTF{black}{black!80},
    colupper=white,
    colframe=TUDa-3a,
    listing only,
    % listing options={style=tcblatex,language=sh},
    listing engine=minted,
    minted style=dracula,
    minted options={
        % linenos=true,
        numbersep=3mm,
        texcl=true,
        autogobble,
        escapeinside=@@,
        breaklines,
        highlightcolor=yellow!50!black,
        #1
    },
    #2,
    % before upper={\textcolor{red}{\small\ttfamily\bfseries root \$> }},
    % every listing line={\textcolor{red}{\small\ttfamily\bfseries root \$> }}
}

%Includes
\usepackage{epstopdf}
\usepackage{wrapfig}
\usepackage{tipa}
\usepackage{tikz}
\usetikzlibrary{calc,shapes,arrows}
%tip: use http://l04.scarfboy.com/coding/phonetic-translation?from=ipa&fromtext=%CB%88pa%C9%AA%CE%B8n%CC%A9&to=tipa
%for converting ipa


\graphicspath{ {./media/} }

\def\shortyear#1{\expandafter\shortyearhelper#1}
\def\shortyearhelper#1#2#3#4{#3#4}

\newcount\NextYear
\NextYear = \year
\advance\NextYear by 1

\newcommand\NextYearShort{\shortyear{\the\NextYear}}

% notes
\usepackage{pgfpages}
\setbeamertemplate{note page}[plain]
%\setbeameroption{show notes on second screen}

% macro for change speaker sign
\newcommand{\changespeaker}{
	\begin{tikzpicture}[line width=.6mm, shorten >= 3pt, shorten <= 3pt]

	\coordinate (c1);
	\coordinate[right of=c1] (c2);

	\draw[rectangle, draw=red!80, fill=red!80, align=center, rounded corners] ($(c1.north west)+(0,-0.3)$) rectangle ($(c2.south east)+(0, 0.3)$) {};
	\draw[->,white] (c1)[bend left] to node[auto] {} (c2);
	\draw[->,white] (c2)[bend left] to node[auto] {} (c1);
	\end{tikzpicture}
}

%Listing-Style pyhon
\title[Programmiervorkurs]{Programmiervorkurs Wintersemester \the\year/\NextYearShort}
\subtitle{{\small der Fachschaft Informatik}}
\logo*{\includegraphics{../globalMedia/bildmarke_ohne_rand}}
\institute{Fachschaft Informatik}
\date{Wintersemester \the\year/\NextYearShort}


% macros
\newcommand{\livecoding}{\begin{frame}\frametitle{\insertsectionhead \\  {\small \insertsubsectionhead}}\centering \huge \vskip 2cm\textbf{\textcolor{red}{Live-Coding}}\end{frame}}

%\newcommand{\slidehead}{\frametitle{\insertsectionhead \\ {\small \insertsubsectionhead}}\vspace{3mm}}
\newcommand{\slidehead}{\frametitle{\insertsectionhead} \framesubtitle{\insertsubsectionhead}\vspace{3mm}}
\newcommand{\tocslide}{\begin{frame}[t]\frametitle{Inhaltsverzeichnis}\vspace{3mm}{\small\tableofcontents[subsectionstyle=shaded]}\end{frame}}


% colors
\definecolor{lightpetrol}{RGB}{0,223,194}


\begin{document}

%Deckblatt
\subtitle{Kapitel 1: Erste Schritte}
\titlegraphic{
    \begin{columns}
        \begin{column}{6cm}
            \vspace{4.32mm}
            \begin{center}
                {\huge Mit einem Computer Reden}
            \end{center}
            \vspace{-5.0mm}
            \begin{figure}
                \centering
                \includeinvertablegraphics[scale=0.3]{code_quality.png}
                \\	\sffamily \tiny Bild: \href{https://xkcd.com/1513/}{https://xkcd.com/1513/}
            \end{figure}
        \end{column}
        \begin{column}{3.5cm}
            \vspace{-0.3mm}
            \begin{figure}
                \centering
                \includeinvertablegraphics[scale=0.2]{good_code.png}
                \\	\sffamily \tiny Bild: \href{https://xkcd.com/844/}{https://xkcd.com/844/}
            \end{figure}
        \end{column}
    \end{columns}}
\maketitle

% Bevor es losgeht die Vortragenden erwähnen!

% Um den Vorkurs zu beginnen muss ich irgendwie die Zuschauer für das Programmieren inspirieren. Es muss sie überzeugen, dass Programmieren geil ist, das die Vortragenden sympathisch sind und dass eigentlich alles super ist. Dafür mache ich eine Einführung, die Programmieren als abstraktes Gedankenexperiment einführt.
% Man stelle sich eine außerirdische Lebensform vor, welche eine komplett andere Art von Intelligenz als wir Menschen hat. Wie kann man mit dieser Lebensform kommunizieren? Wir haben verschiedene Informationen über die Lebensform: Sie ist exakt, vergisst keine Details, ist sehr fleißig, ist sehr schnell aber ist nicht eigenständig. Welche Sprache kann man benutzen, bei der sowohl wir Menschen, als auch diese Lebensform etwas versteht? Englisch würde zwar uns Menschen gefallen, aber Englisch hat zu viele Wörter, deren genaue Bedeutung unklar ist. "Mehrdeutigkeit" ist etwas mit der die Lebensform überhaupt nicht umgehen kann. Die Lebensform könnte eine Sprache vorschlagen, die nur aus Nullen und Einsen besteht, und ein so begrenztes Vokabular hat, dass jeder Begriff eine exakte Bedeutung hat. Damit kann der Mensch jedoch schlecht umgehen, weil er sich Zahlen schlecht merken kann und keine Intuition aufbaut was die Begriffe bedeuten. Lösung: Ein Mittelweg: Eine künstlich erfundene Sprache, die eine sehr begrenzte Anzahl Begriffe auf Englisch hat, die man sich gut merken kann und gleichzeitig eine exakt definierte Bedeutung hat, so dass das Wesen sich es auch merken kann.

\section{Warum mit Computer kommunizieren?}
\begin{frame}
    \slidehead
    \vspace{3mm}
    \begin{itemize}
        \pause
        \item Computer sind rein-mathematische Maschinen
            \pause
        \item Computer führen einfache Arbeitsschritte extrem \textbf{schnell} aus
            \pause
        \item Programme sind aus vielen Arbeitsschritten zusammengesetzt
            \pause
            \pythonfile{listings/Programmieren.txt}
    \end{itemize}
\end{frame}

\section{Was tun wir hier eigentlich?}
\begin{frame}
    \slidehead
    \begin{itemize}
        \item Programmieren mit Kotlin \textcolor{gray}{[\textprimstress k\textopeno t\textscripta l\textsci n]}
            \pause
        \item Kotlin?
            \pause
        \item[] $\Rightarrow$ Programmiersprache!
        \item[] $\Rightarrow$ Verständlicher für Menschen
        \item[] $\Rightarrow$ Oft verwendet
    \end{itemize}

    \pause
    \vspace{0.75cm}

    \tikzstyle{rect} = [rectangle, rounded corners, minimum width=3cm, minimum height=1cm,text centered, draw=., fill=\IfDarkModeTF{red!30!black}{red!10}]
    \centering
    \begin{tikzpicture}[node distance=5cm]
        \node (syntax) [rect, text width=4cm, minimum height=2cm] {\textbf{Syntax}\\ Wortschatz};
        \node (grammatik) [rect, right of=syntax, text width=4cm, minimum height=2cm] {\textbf{Grammatik}\\ Regeln um Befehle zusammenzusetzen};
    \end{tikzpicture}

\end{frame}

\begin{comment}
    \section{Das erste Programm}
    \subsection{Wie fange ich an?}
    \begin{frame}
        \slidehead
        \begin{itemize}
            \item Befehle bestehen aus Text
            \item Mit einem einfachen Texteditor schreibt man Befehle in eine Datei
            \item Anschließend interpretiert der Computer die Befehle
        \end{itemize}
    \end{frame}
\end{comment}

\section{Compiler?}
\begin{frame}[t]
    \slidehead
    \begin{columns}[T]
        \begin{column}{2cm}
            \begin{tikzpicture}

                \node[] (mensch) at (0,0) {\includeinvertablegraphics[height=1.3cm]{media/wesen_notebook.png}};

                \visible<2->{
                    \node[fill=lightpetrol,draw, inner sep=5pt,rounded rectangle] (interpreter) at (0,-1.8) {Compiler};
                    \draw[->,line width=.6mm] (mensch.south) -- (interpreter.north);
                }
                \visible<3->{

                    \node[fill=\IfDarkModeTF{.!30!\thepagecolor}{lightgray},draw, inner sep=5pt, rounded rectangle] (computer) at (0,-4) {Computer};

                    \draw[->,line width=.6mm] (interpreter.south) -- (computer.north);
                }
            \end{tikzpicture}
        \end{column}
        \begin{column}{10cm}
            \begin{itemize}
                \item[]
                \item \textbf{Der Mensch} beschreibt die Aufgabe in einer Programmiersprache $\Rightarrow$ Quelltext
                \item[]
                \item[]
                    \visible<2->{
                \item \textbf{Der Compiler} kompiliert den Quelltext zu Maschinenbefehle
                    }
                \item[]
                \item[]
                    \visible<3->{
                \item \textbf{Der Computer} führt die Maschinenbefehle aus
                    }
            \end{itemize}
        \end{column}
    \end{columns}
\end{frame}

%\nextvid{Erstes Programm}{}

\subtitle{Kapitel 1: Erste Schritte}

%Deckblatt
% \titlegraphic{
% 	\begin{columns}
% 		\begin{column}{5.5cm}
% 			\begin{tikzpicture}
% 				\draw (0, 0) node[inner sep=0] {\includeinvertablegraphics[scale=0.265]{python.png}};
% 				% ich habe eine Mail an xkcd geschrieben, es ist ok das wir die klammern einfügen :)
% 				\draw (-1.95, -2.33) node {\tiny (};
% 				\draw (-1.02, -2.33) node {\tiny )};
% 			\end{tikzpicture}
% 			\\	\sffamily \tiny Bild: \href{https://xkcd.com/353/}{https://xkcd.com/353/ [edit: Kammern eingefügt für Python 3]}
% 		\end{column}
% 		\begin{column}{5cm}
% 			\begin{center}
% 				\vspace{-3.5cm}
% 				{\huge Erstes Program}
% 				\end{center}
% 			\end{column}
% 		\end{columns}
% 	}
% \maketitle

\section{Erstes Program}
\begin{frame}[fragile]
    % Beispiel, um zu zeigen wie so ein Befehl aussieht (und ein Kommentar)
    \slidehead
    \hskip .8cm
    \vspace{.5cm}

    \kotlinfile[escapeinside=\$\$,texcomments]{listings/helloworld.kts}
    
    \begin{tikzpicture}[remember picture,overlay]
        \draw<2->[decorate,decoration={brace,amplitude=6pt}] ([yshift=1em]pic cs:commentstart) -- ([yshift=1em]pic cs:commentend) node [midway,yshift=15pt,xshift=3.4cm] {Kommentar: wird vom Computer \textbf{ignoriert}, ist für Menschen \textbf{nützlich}};
        \draw<3->[decorate,decoration={mirror,brace,amplitude=6pt}] ([yshift=-1ex]pic cs:commandstart) -- ([yshift=-1ex]pic cs:commandend) node [midway,yshift=-15pt] {Befehl};
        \draw<4->[decorate,decoration={mirror,brace,amplitude=6pt}] ([yshift=-1ex]pic cs:textstart) -- ([yshift=-1ex]pic cs:textend) node [midway,yshift=-15pt] {Text};
    \end{tikzpicture}

    \onslide<5->
    \vspace{0.25cm}
    \begin{block}{Wichtig}
        \begin{itemize}
            \item Text muss in \pythoninline{" "} oder in \pythoninline{' '} stehen
            \item Verwende \textbf{aussagekräftige} Kommentare \footnotesize (dein späteres Ich wird es dir danken)!
        \end{itemize}
    \end{block}
    \note{
        " und ' stehen nicht für Auslassungszeichen (Kopiere den Text von der Zeile oberhalb), sondern stehen für wörtliche Rede.
    }
\end{frame}


\subsection{Textausgabe}
\begin{frame}
    % Beispiel wie kleinkariert man mit diesem Computer reden muss - unintuitiv für Menschen, dass man nicht einfach "Enter" drücken kann, sondern so kryptisch "\n" schreiben muss
    \slidehead
    \begin{itemize}
        \item Mithilfe des Befehls \pythoninline{print()} kann Text auf der Konsole ausgeben werden
    \end{itemize}
    \pause
    \begin{columns}
        \begin{column}{6cm}
            \pythonfile{listings/print_Listing.py}
        \end{column}
        \begin{column}{7cm}
            \pythonfile{listings/printSingleLine_Listing.py}
        \end{column}
    \end{columns}
    \vspace{0.25cm}
    \begin{block}{Merke}
        \pythoninline{print()} erzeugt einen Zeilenumbruch nach der Ausgabe
    \end{block}
\end{frame}

\subsection{Steuerzeichen}
\begin{frame}
    \slidehead

    \begin{table}[htbp]
        \begin{tabular}{|l|l|}
            \hline
            \textbf{Symbol}         & \textbf{Wirkung}                                               \\ \hline
            $\backslash$n           & Zeilenumbruch                                                  \\ \hline
            $\backslash$\"          & Doppeltes Anführungszeichen  (nur in \pythoninline{" "} nötig) \\ \hline
            $\backslash$'           & Einfaches Anführungszeichen (nur in \pythoninline{' '} nötig)  \\ \hline
            $\backslash \backslash$ & Backslash                                                      \\ \hline
        \end{tabular}
        \label{}
    \end{table}
    \pause
    \pythonfile{listings/Steuerzeichen_Listing.py}
\end{frame}

\section{Dateisystem}
\begin{frame}
    \slidehead
    \begin{itemize}
        \item Computer benutzen Dateisysteme, um ihre Dateien zu organisieren
        \item Computer benutzen Dateien um Daten (Bilder, Texte, Programme, ...) permanent zu speichern
        \item Dabei werden die Dateien oft in einer Ordnerstruktur abgelegt
        \item Dateien haben einen eindeutigen Namen innerhalb ihres Ordners
        \item Dateien haben meist eine Dateiendung, welche signalisiert, um welche Art von Datei es sich handelt. (z.B: .png, .txt, .doc, .kts, .exe)
    \end{itemize}
\end{frame}

\section{Vom Quelltext zum fertigen Programm}
\begin{frame}
    \slidehead

    \begin{columns}[T]
        \begin{column}{.4\textwidth}
            \begin{tikzpicture}
                \node[] (mensch) at (0,0) {\includeinvertablegraphics[height=1.3cm]{media/wesen_notebook.png}};

                \node[fill=lightpetrol,draw, inner sep=5pt,rounded rectangle] (file) at (0,-1.8) {helloworld.kts};
                \draw[->,line width=.6mm] (mensch.south) -- (file.north);

                \node[fill=lightpetrol,draw, inner sep=5pt,rounded rectangle] (python) at (0,-4) {\texttt{kotlin helloworld.kts}};
                \draw[->,line width=.6mm] (file.south) -- (python.north);
            \end{tikzpicture}
        \end{column}
        \begin{column}{.6\textwidth}
            \begin{itemize}
                \item Ausführen: \texttt{kotlin dateiName.kts}
                \item Kotlin-Programme laufen überall, wo Kotlin installiert ist
            \end{itemize}
            \begin{block}{Hinweis}
                Kotlin-Script-Quelltext mit der Dateiendung \texttt{.kts} kennzeichnen und sinnvoll benennen! Leerzeichen sind problematisch.
            \end{block}
        \end{column}
    \end{columns}
\end{frame}

\section{Ein Python-Programm schreiben}
\begin{frame}
    \slidehead

    \begin{center}
        \vskip -10 pt
        \includegraphics[scale=0.25]{HelloWorldImEditor\IfDarkModeT{_dark}}
    \end{center}

    \begin{itemize}
        \item Programm wird in einem herkömmlichen Editor geschrieben
        \item hier ein paar Editor Beispiele: \texttt{gedit}, \texttt{atom}, \texttt{kate}
    \end{itemize}
    \begin{alertblock}{Wichtig}
        Verwendet \textbf{keine} Textverarbeitung wie LibreOffice, Pages oder MS Word!
    \end{alertblock}
\end{frame}

% \begin{frame}
% 	\slidehead
% 		Zusammenarbeiten hilft beim Lernen, leider ist das aktuell aber etwas schwierig.\\
% 		Hier sind einige Möglichkeiten wie ihr das trotzdem tun könnt (alles freiwillig, natürlich):
% 	\vspace{0.5cm}
%
% 	\begin{itemize}
% 		\item Atom plugin \textbf{Teletype}:\\ \href{https://teletype.atom.io/}{\textbf{https://teletype.atom.io/}}\\(Benötigt GitHub account)
% 		\item Alternativ gibt es online-Services wie:
% 		\begin{itemize}
% 			\item \href{https://codecollab.io/}{https://codecollab.io/} (Wählt Python\textbf{3} aus!)
% 			\item \href{https://repl.it/}{https://repl.it/}
% 		\end{itemize}
% 		Bei beiden gilt, wenn ihr euch keinen Account erstellt, solltet ihr euren Code im Anschluss auf eurem Rechner herunter laden bzw. kopieren!
% 	\end{itemize}
% 	Ihr könnt euch während dessen natürlich auf Discord unterhalten.
% \end{frame}

\section{Ein Python-Programm ausführen}
\begin{frame}[fragile]
    \slidehead
    \begin{bashcode}
        $ python helloworld.py
            Hello World!
        $
    \end{bashcode}

    \begin{itemize}
        \item Programm mit \texttt{python dateiName.py} starten
    \end{itemize}
\end{frame}

\subsection{Interaktiver Interpreter}
\begin{frame}[fragile]
    \slidehead

    \begin{bashcode}
        $ python
        |\texttt{Python 3.10.7 (main, Sep  6 2022, 21:22:27)}|
        |\texttt{[GCC 12.2.0] on linux}|
        |\texttt{Type {\ditto}help{\ditto}, {\ditto}copyright{\ditto}, {\ditto}credits{\ditto} or {\ditto}license{\ditto} for more information.}|
        >>> print("Hello World!")
        Hello World!
        >>>
    \end{bashcode}

    %|\texttt{Python 3.7.0 (default, Jul 15 2018, 10:44:58)}|
    %|\texttt{[GCC 8.1.1 20180531] on linux}|
    %|\texttt{Type {\ditto}help{\ditto}, {\ditto}copyright{\ditto}, {\ditto}credits{\ditto} or %{\ditto}license{\ditto} for more information.}|

    \vskip -.5em

    \begin{itemize}
        \item Der Python-Interpreter hat einen interaktiven Modus
        \item Dieser kann mit dem Konsolenbefehl \texttt{python} gestartet werden
        \item Anschließend können Python-Befehle eingegeben werden
        \item[]  %leerzeile
        \item \textbf{Strg+D} (außer Windows) oder der Befehl \pythoninline{quit()} beendet den Interpreter wieder
    \end{itemize}
    \begin{alertblock}{Wichtig}
        Der Python-Interpreter ist \textbf{kein} Editor!
    \end{alertblock}
\end{frame}

\livecoding
\subsection{}


%\nextvid{Fehler}{Beispiel: Hello World}

\subtitle{Kapitel 1: Erste Schritte}

%Deckblatt
% \titlegraphic{
% 	\begin{columns}
% 		\begin{column}{3cm}
% 			\begin{center}
% 				\vspace{2cm}
% 				{\huge Fehler}
% 			\end{center}
% 		\end{column}
% 		\begin{column}{5cm}
% 			\\
% 			\vspace{1cm}
% 			\includeinvertablegraphics[scale=0.4]{wisdom_of_the_ancients.png}
% 			\\	\sffamily \tiny Bild: \href{https://xkcd.com/979/}{https://xkcd.com/979/}
% 		\end{column}
% 	\end{columns}}
% \maketitle

\subsection{Fehlerarten}
\begin{frame}
    \slidehead
    \begin{itemize}
        \item \textbf{Lexikalische Fehler}
            \begin{itemize}
                \item Beispielsweise Tippfehler
                    % ACHTUNG: print ist bewusst falsch geschrieben!
                \item \pythoninline{prit("Lexikalischer Fehler")}
            \end{itemize}
        \item \textbf{Syntaktische Fehler}
            \begin{itemize}
                \item Falsche Klammern
                \item Anführungszeichen nicht geschlossen
            \end{itemize}
        \item \textbf{Semantische Fehler}
            \begin{itemize}
                \item Division durch 0
            \end{itemize}
    \end{itemize}
\end{frame}

\section{Fehlerbeispiele}
\subsection{Lexikalischer Fehler}
\begin{frame}
    \slidehead
    \begin{itemize}
        \item \texttt{test.py}:
            \pythonfile{listings/lexError_Listing.py}
            \pause
            \pythonfile{listings/lexError_Listing.txt}
            \pause
        \item Fehler in \texttt{test.py}
            \pause
        \item In \texttt{line 1}, also \textbf{Zeile 1}
            \pause
        \item Der Name \texttt{prit} existiert nicht
            \pause
        \item Es war \texttt{print} gemeint (das weiß der Interpreter natürlich nicht)
    \end{itemize}
\end{frame}

\subsection{Syntaktischer Fehler}

\begin{frame}
    \slidehead
    \begin{itemize}
        \item \texttt{zweiterTest.py}:
            \pythonfile{listings/otherSynError_Listing.py}
            \pause
            \pythonfile{listings/otherSynError_Listing.txt}
            \pause
        \item Fehler in \texttt{zweiterTest.py}
            \pause
        \item in \texttt{line 1} fehlt eine \texttt{(}
        \item der Interpreter versteht hier nicht was gemeint war, da das kein richtiger Python-Code ist.
    \end{itemize}
\end{frame}

\begin{frame}
    \slidehead
    \begin{itemize}
        \item \texttt{andererTest.py}:
            \pythonfile{listings/eofError_Listing.py}
            \pause
            \pythonfile{listings/eofError_Listing.txt}
            \pause
        \item Fehler in \texttt{andererTest.py}
            \pause
        \item In \texttt{line 3} stimmt hier nicht ganz, aber in \texttt{line 1} fehlt eine \texttt{)}
            \pause
        \item Der Interpreter sucht weiterhin nach \texttt{)}, aber die Datei ist in \textbf{Zeile 3} zu Ende
            \pause
        \item Deshalb \texttt{unexpected EOF}: \textbf{EOF} steht für \textbf{E}nd \textbf{O}f \textbf{F}ile
    \end{itemize}
\end{frame}

\subsection{}
\livecoding

%\nextvid{Datentypen}{}

\end{document}
