% !TeX document-id = {5c9b7af7-cc8b-495c-a61c-cd548e104791}
% !TeX TXS-program:compile = txs:///pdflatex/[--shell-escape]
% !TeX root = 01_Einfuehrung.tex
\documentclass[accentcolor=3c,landscape,ngerman,presentation,t,usenames,dvipsnames,svgnames,table]{tudabeamer}

% Template-Modifikationen
\addtobeamertemplate{frametitle}{}{\vspace{-1em}} % mehr Platz vor dem Inhalt

% andere global gemeinsame definitionen
%Includes
\usepackage[ngerman]{babel} %Deutsche Silbentrennung
\usepackage[utf8]{inputenc} %Deutsche Umlaute
\usepackage{float}
\usepackage{graphicx}
\usepackage{minted}
\RequirePackage{csquotes}
\RequirePackage{fontawesome5}

\DeclareGraphicsExtensions{.pdf,.png,.jpg}

\makeatletter
\author{Vorkursteam der Fachschaft Informatik}
\let\Author\@author

% dark mode
\ExplSyntaxOn
\IfDarkModeT{
    \cs_if_exist:NT \setbeamercolor {
        \setbeamercolor*{smallrule}{bg=.}
        \setbeamercolor*{normal~text}{bg=\thepagecolor,fg=.}
        \setbeamercolor*{background~canvas}{parent=normal~text}
        \setbeamercolor*{section~in~toc}{parent=normal~text}
        \setbeamercolor*{subsection~in~toc}{parent=normal~text,fg=.}
        \setbeamercolor*{footline}{parent=normal~text}
        \setbeamercolor{block~title~alerted}{fg=white,bg=white!20!\thepagecolor}
        \setbeamercolor*{block~body}{bg=black!70!gray!98!blue}
        \setbeamercolor*{block~body~alerted}{bg=\thepagecolor}
    }
    \cs_if_exist:NT \setbeamertemplate {
        \setbeamertemplate{subsection~in~toc~shaded}[default][50]
    }
}
\ExplSyntaxOff

% macros
\renewcommand{\arraystretch}{1.2} % Höhe einer Tabellenspalte minimal erhöhen
\newcommand{\N}{{\mathbb N}}
\renewcommand{\code}{\inputminted[]{python}}

\IfDarkModeTF{
    \newmintedfile[pythonfile]{python}{
        fontsize=\small,
        style=native,
        linenos=true,
        numberblanklines=true,
        tabsize=4,
        obeytabs=false,
        breaklines=true,
        autogobble=true,
        encoding="utf8",
        showspaces=false,
        xleftmargin=20pt,
        frame=single,
        framesep=5pt,
    }
    \newmintinline{python}{
        style=native,
        encoding="utf8"
    }
    \newmintinline{kotlin}{
        style=native,
        encoding="utf8"
    }


    \definecolor{codegray}{HTML}{eaf1ff}
    \newminted[bashcode]{awk}{
        escapeinside=||,
        fontsize=\small,
        style=native,
        linenos=true,
        numberblanklines=true,
        tabsize=4,
        obeytabs=false,
        breaklines=true,
        autogobble=true,
        encoding="utf8",
        showspaces=false,
        xleftmargin=20pt,
        frame=single,
        framesep=5pt
    }
}{
    \newmintedfile[pythonfile]{python}{
        fontsize=\small,
        style=friendly,
        linenos=true,
        numberblanklines=true,
        tabsize=4,
        obeytabs=false,
        breaklines=true,
        autogobble=true,
        encoding="utf8",
        showspaces=false,
        xleftmargin=20pt,
        frame=single,
        framesep=5pt,
    }
    \newmintinline{python}{
        style=friendly,
        encoding="utf8"
    }
    \newmintinline{kotlin}{
        style=friendly,
        encoding="utf8"
    }

    \definecolor{codegray}{HTML}{eaf1ff}
    \newminted[bashcode]{awk}{
        escapeinside=||,
        fontsize=\small,
        style=friendly,
        linenos=true,
        numberblanklines=true,
        tabsize=4,
        obeytabs=false,
        breaklines=true,
        autogobble=true,
        encoding="utf8",
        showspaces=false,
        xleftmargin=20pt,
        frame=single,
        framesep=5pt
    }
}

\let\origpythonfile\pythonfile
\renewcommand{\pythonfile}[1]{\pythonfileh{#1}{}}
\newcommand{\pythonfileh}[2]{\origpythonfile[#2]{#1}}

\DeclareDocumentCommand{\kotlinfile}{O{} O{} m}{\inputCode[#1]{minted language=kotlin,#2}{#3}}

\newcommand*{\ditto}{\texttt{\char`\"}}

\newcommand{\shellprefix}{\textcolor{TUDa-3a}{\ttfamily\bfseries \$~}}
\DeclareTCBListing{commandshell}{ O{} O{} }{
    colback=\IfDarkModeTF{black}{black!80},
    colupper=white,
    colframe=TUDa-3a,
    listing only,
    % listing options={style=tcblatex,language=sh},
    listing engine=minted,
    minted style=dracula,
    minted options={
        % linenos=true,
        numbersep=3mm,
        texcl=true,
        autogobble,
        escapeinside=@@,
        breaklines,
        highlightcolor=yellow!50!black,
        #1
    },
    #2,
    % before upper={\textcolor{red}{\small\ttfamily\bfseries root \$> }},
    % every listing line={\textcolor{red}{\small\ttfamily\bfseries root \$> }}
}

%Includes
\usepackage{epstopdf}
\usepackage{wrapfig}
\usepackage{tipa}
\usepackage{tikz}
\usetikzlibrary{calc,shapes,arrows}
%tip: use http://l04.scarfboy.com/coding/phonetic-translation?from=ipa&fromtext=%CB%88pa%C9%AA%CE%B8n%CC%A9&to=tipa
%for converting ipa


\graphicspath{ {./media/} }

\def\shortyear#1{\expandafter\shortyearhelper#1}
\def\shortyearhelper#1#2#3#4{#3#4}

\newcount\NextYear
\NextYear = \year
\advance\NextYear by 1

\newcommand\NextYearShort{\shortyear{\the\NextYear}}

% notes
\usepackage{pgfpages}
\setbeamertemplate{note page}[plain]
%\setbeameroption{show notes on second screen}

% macro for change speaker sign
\newcommand{\changespeaker}{
	\begin{tikzpicture}[line width=.6mm, shorten >= 3pt, shorten <= 3pt]

	\coordinate (c1);
	\coordinate[right of=c1] (c2);

	\draw[rectangle, draw=red!80, fill=red!80, align=center, rounded corners] ($(c1.north west)+(0,-0.3)$) rectangle ($(c2.south east)+(0, 0.3)$) {};
	\draw[->,white] (c1)[bend left] to node[auto] {} (c2);
	\draw[->,white] (c2)[bend left] to node[auto] {} (c1);
	\end{tikzpicture}
}

%Listing-Style pyhon
\title[Programmiervorkurs]{Programmiervorkurs Wintersemester \the\year/\NextYearShort}
\subtitle{{\small der Fachschaft Informatik}}
\logo*{\includegraphics{../globalMedia/bildmarke_ohne_rand}}
\institute{Fachschaft Informatik}
\date{Wintersemester \the\year/\NextYearShort}


% macros
\newcommand{\livecoding}{\begin{frame}\frametitle{\insertsectionhead \\  {\small \insertsubsectionhead}}\centering \huge \vskip 2cm\textbf{\textcolor{red}{Live-Coding}}\end{frame}}

%\newcommand{\slidehead}{\frametitle{\insertsectionhead \\ {\small \insertsubsectionhead}}\vspace{3mm}}
\newcommand{\slidehead}{\frametitle{\insertsectionhead} \framesubtitle{\insertsubsectionhead}\vspace{3mm}}
\newcommand{\tocslide}{\begin{frame}[t]\frametitle{Inhaltsverzeichnis}\vspace{3mm}{\small\tableofcontents[subsectionstyle=shaded]}\end{frame}}


% colors
\definecolor{lightpetrol}{RGB}{0,223,194}


% Für den Vortragenden:
%
% Bevor es losgeht die Vortragenden erwähnen!
%
% Um den Vorkurs zu beginnen muss man irgendwie die Zuschauer für das Programmieren inspirieren.
% Es muss sie überzeugen, dass Programmieren geil ist, das die Vortragenden sympathisch sind und dass eigentlich alles super ist.
%
% Warum nutzt man also überhaupt Computer, rechnen lernt man schließlich schon in der Schule.
% Und für alles andere, wie Notizen, Kalender etc., hat man Zettel und Stift.
%
% - Rechner sind exakt.
% - Rechner sind schnell, sehr schnell.
% - Die Ergebnisse der Rechenschritte (Programm später erwähnen) sind reproduzierbar.
% - TODO: mehr
%
% Man möchte den Zuschauern klarmachen, dass ein Rechner keine natürliche Sprache kann (auch nicht via ChatGPT!),
% und man eine Programmiersprache braucht, die sowohl Mensch, als auch Rechner spricht.
% Dafür mache ich eine Einführung, die Programmieren als abstraktes Gedankenexperiment einführt:
% Man stelle sich eine außerirdische Lebensform vor, welche eine komplett andere Art von Intelligenz als wir Menschen hat.
% Wie kann man mit dieser Lebensform kommunizieren?
% Wir haben verschiedene Informationen über die Lebensform: Sie ist exakt, vergisst keine Details, ist sehr fleißig, ist sehr schnell aber ist nicht eigenständig.
% Welche Sprache kann man benutzen, bei der sowohl wir Menschen, als auch diese Lebensform etwas versteht?
% Englisch würde zwar uns Menschen gefallen, aber Englisch hat zu viele Wörter, deren genaue Bedeutung unklar ist.
% "Mehrdeutigkeit" ist etwas mit der die Lebensform überhaupt nicht umgehen kann.
% Die Lebensform könnte eine Sprache vorschlagen, die nur aus Nullen und Einsen besteht, und ein so begrenztes Vokabular hat, dass jeder Begriff eine exakte Bedeutung hat.
% Damit kann der Mensch jedoch schlecht umgehen, weil er sich Zahlen schlecht merken kann und keine Intuition aufbaut was die Begriffe bedeuten.
% Lösung: Ein Mittelweg: Eine künstlich erfundene Sprache, die eine sehr begrenzte Anzahl Begriffe auf Englisch hat,
% die man sich gut merken kann und gleichzeitig eine exakt definierte Bedeutung hat, so dass das Wesen sich es auch merken kann.
%
% Unsere Programmiersprache heißt Python.
% Wir folgen ihrem Formalismus, um unseren Computern Befehle bzw. Programme zu vermitteln und diese ausführen zu lassen.
% Die Wahl ist willkürlich und es gibt nicht die *eine* Programmiersprache.
% Dies soll dem Publikum verdeutlicht werden.

\begin{document}

%Deckblatt
\titlegraphic{
    \begin{columns}[b]
        \begin{column}{.4\textwidth}
            \centering
            \begin{figure}
                \centering
                \includeinvertablegraphics[height=.55\textheight]{computer1917.jpg}
                \caption*{Ein Rechenzentrum, 1917}
            \end{figure}
        \end{column}%
        \begin{column}{.4\textwidth}
            \centering
            \begin{figure}
                \centering
                \includeinvertablegraphics[height=.55\textheight]{Z3_Deutsches_Museum.JPG}
                \caption*{Der Z3 von Konrad Zuse}
            \end{figure}
        \end{column}
    \end{columns}}
\maketitle

\section{Noch etwas Historie}
\begin{frame}
    \slidehead
    \vspace{3mm}
    \begin{columns}[b]
        \begin{column}{.3\textwidth}
            \begin{figure}
                \centering
                \includeinvertablegraphics[width=\textwidth]{punchcards.png}
                \caption*{Ein Lochkartenleser}
            \end{figure}
        \end{column}%
        \begin{column}{.4\textwidth}
            \centering
            \begin{figure}
                \centering
                \includeinvertablegraphics[height=.58\textheight]{apollo_code.png}
                \caption*{\footnotesize Margaret Hamilton und ihr \\ Programm der Apollo 11 Mission}
            \end{figure}
        \end{column}
    \end{columns}
\end{frame}

\section{Kapitel 1: Mit einem Computer Reden}
\begin{frame}
    \slidehead
    \begin{itemize}[<+->]
        \item Computer sind rein-mathematische Maschinen
        \item Sie sprechen keine natürlichen Sprachen
        \item Sie rechnen lediglich Arbeitsschritt für Arbeitsschritt
        \item \dots dabei sind sie jedoch extrem \textbf{schnell}
            \vspace{3em}
            % Im Sinne: Das kann man gar nicht groß genug schreiben
        \item \dots also sie sind wirklich \textbf{\Huge schnell}
    \end{itemize}
\end{frame}

\section{Die Sprachen der Computer}
\begin{frame}
    \slidehead
    \begin{itemize}[<+->]
        \item Aber wie schreiben wir die Arbeitsschritte auf?
        \item Anweisung für Anweisung gemäß einer fest definierten Schreibweise.
        \item Eine Liste von Arbeitsschritten - ähnlich wie ein Kochrezept - nennen wir ein \emph{Programm}
        \item Die \emph{Programmiersprache} gibt uns die Regeln, wie unsere Arbeitsschritte aussehen
        \item \emph{Python} ist eine solche Programmiersprache
            % An der Stelle eventuell Dschungel der Programmiersprachen von Ullmann erwähnen,
            % falls das wieder stattfindet - ansonsten auf YouTube
        \item Es gibt aber noch unzählige Andere
        \item Insbesondere gibt es nicht die \emph{eine} Programmiersprache
    \end{itemize}
\end{frame}

\section{Python}
\subsection{Die Sprache des Vorkurses}
\begin{frame}
    \slidehead
    \begin{itemize}[<+->]
        \item In Python bestehen Programme aus Text
        \item Dieser Text wird auch \emph{Quelltext} (eng. Source Code) genannt
        \item Man kann Python-Programme direkt in Python eingeben
        \item Alternativ kann man das Programm in einer Datei speichern
        \item Diese Datei kann dann an Python übergeben werden
        \item So kann man das selbe Programm noch Jahre später nutzen
    \end{itemize}
\end{frame}

\livecoding

\section{Exkurs: Ordner und Dateien}
\begin{frame}
    \slidehead
    \begin{itemize}[<+->]
        \item Computer benutzen \emph{Dateisysteme}, um ihre \emph{Dateien} zu organisieren
        \item In Dateien sind Daten (Bilder, Texte, Programme, ...) permanent gespeichert
        \item Dabei werden die Dateien oft in \emph{Ordnern} zusammengefasst
        \item Dateien haben einen eindeutigen Namen innerhalb ihres Ordners
        \item Dateien haben meist eine Dateiendung, welche signalisiert, um welche Art von Datei es sich handelt, z.B: \pythoninline{.png, .txt, .doc, .kts, .exe}
    \end{itemize}
\end{frame}

\end{document}
