% !TeX document-id = {5c9b7af7-cc8b-495c-a61c-cd548e104791}
% !TeX TXS-program:compile = txs:///pdflatex/[--shell-escape]
\documentclass[accentcolor=3c,landscape,ngerman,presentation,t,usenames,dvipsnames,svgnames,table]{tudabeamer}

% Template-Modifikationen
\addtobeamertemplate{frametitle}{}{\vspace{-1em}} % mehr Platz vor dem Inhalt

% andere global gemeinsame definitionen
%Includes
\usepackage[ngerman]{babel} %Deutsche Silbentrennung
\usepackage[utf8]{inputenc} %Deutsche Umlaute
\usepackage{float}
\usepackage{graphicx}
\usepackage{minted}
\RequirePackage{csquotes}
\RequirePackage{fontawesome5}

\DeclareGraphicsExtensions{.pdf,.png,.jpg}

\makeatletter
\author{Vorkursteam der Fachschaft Informatik}
\let\Author\@author

% dark mode
\ExplSyntaxOn
\IfDarkModeT{
    \cs_if_exist:NT \setbeamercolor {
        \setbeamercolor*{smallrule}{bg=.}
        \setbeamercolor*{normal~text}{bg=\thepagecolor,fg=.}
        \setbeamercolor*{background~canvas}{parent=normal~text}
        \setbeamercolor*{section~in~toc}{parent=normal~text}
        \setbeamercolor*{subsection~in~toc}{parent=normal~text,fg=.}
        \setbeamercolor*{footline}{parent=normal~text}
        \setbeamercolor{block~title~alerted}{fg=white,bg=white!20!\thepagecolor}
        \setbeamercolor*{block~body}{bg=black!70!gray!98!blue}
        \setbeamercolor*{block~body~alerted}{bg=\thepagecolor}
    }
    \cs_if_exist:NT \setbeamertemplate {
        \setbeamertemplate{subsection~in~toc~shaded}[default][50]
    }
}
\ExplSyntaxOff

% macros
\renewcommand{\arraystretch}{1.2} % Höhe einer Tabellenspalte minimal erhöhen
\newcommand{\N}{{\mathbb N}}
\renewcommand{\code}{\inputminted[]{python}}

\IfDarkModeTF{
    \newmintedfile[pythonfile]{python}{
        fontsize=\small,
        style=native,
        linenos=true,
        numberblanklines=true,
        tabsize=4,
        obeytabs=false,
        breaklines=true,
        autogobble=true,
        encoding="utf8",
        showspaces=false,
        xleftmargin=20pt,
        frame=single,
        framesep=5pt,
    }
    \newmintinline{python}{
        style=native,
        encoding="utf8"
    }
    \newmintinline{kotlin}{
        style=native,
        encoding="utf8"
    }


    \definecolor{codegray}{HTML}{eaf1ff}
    \newminted[bashcode]{awk}{
        escapeinside=||,
        fontsize=\small,
        style=native,
        linenos=true,
        numberblanklines=true,
        tabsize=4,
        obeytabs=false,
        breaklines=true,
        autogobble=true,
        encoding="utf8",
        showspaces=false,
        xleftmargin=20pt,
        frame=single,
        framesep=5pt
    }
}{
    \newmintedfile[pythonfile]{python}{
        fontsize=\small,
        style=friendly,
        linenos=true,
        numberblanklines=true,
        tabsize=4,
        obeytabs=false,
        breaklines=true,
        autogobble=true,
        encoding="utf8",
        showspaces=false,
        xleftmargin=20pt,
        frame=single,
        framesep=5pt,
    }
    \newmintinline{python}{
        style=friendly,
        encoding="utf8"
    }
    \newmintinline{kotlin}{
        style=friendly,
        encoding="utf8"
    }

    \definecolor{codegray}{HTML}{eaf1ff}
    \newminted[bashcode]{awk}{
        escapeinside=||,
        fontsize=\small,
        style=friendly,
        linenos=true,
        numberblanklines=true,
        tabsize=4,
        obeytabs=false,
        breaklines=true,
        autogobble=true,
        encoding="utf8",
        showspaces=false,
        xleftmargin=20pt,
        frame=single,
        framesep=5pt
    }
}

\let\origpythonfile\pythonfile
\renewcommand{\pythonfile}[1]{\pythonfileh{#1}{}}
\newcommand{\pythonfileh}[2]{\origpythonfile[#2]{#1}}

\DeclareDocumentCommand{\kotlinfile}{O{} O{} m}{\inputCode[#1]{minted language=kotlin,#2}{#3}}

\newcommand*{\ditto}{\texttt{\char`\"}}

\newcommand{\shellprefix}{\textcolor{TUDa-3a}{\ttfamily\bfseries \$~}}
\DeclareTCBListing{commandshell}{ O{} O{} }{
    colback=\IfDarkModeTF{black}{black!80},
    colupper=white,
    colframe=TUDa-3a,
    listing only,
    % listing options={style=tcblatex,language=sh},
    listing engine=minted,
    minted style=dracula,
    minted options={
        % linenos=true,
        numbersep=3mm,
        texcl=true,
        autogobble,
        escapeinside=@@,
        breaklines,
        highlightcolor=yellow!50!black,
        #1
    },
    #2,
    % before upper={\textcolor{red}{\small\ttfamily\bfseries root \$> }},
    % every listing line={\textcolor{red}{\small\ttfamily\bfseries root \$> }}
}

%Includes
\usepackage{epstopdf}
\usepackage{wrapfig}
\usepackage{tipa}
\usepackage{tikz}
\usetikzlibrary{calc,shapes,arrows}
%tip: use http://l04.scarfboy.com/coding/phonetic-translation?from=ipa&fromtext=%CB%88pa%C9%AA%CE%B8n%CC%A9&to=tipa
%for converting ipa


\graphicspath{ {./media/} }

\def\shortyear#1{\expandafter\shortyearhelper#1}
\def\shortyearhelper#1#2#3#4{#3#4}

\newcount\NextYear
\NextYear = \year
\advance\NextYear by 1

\newcommand\NextYearShort{\shortyear{\the\NextYear}}

% notes
\usepackage{pgfpages}
\setbeamertemplate{note page}[plain]
%\setbeameroption{show notes on second screen}

% macro for change speaker sign
\newcommand{\changespeaker}{
	\begin{tikzpicture}[line width=.6mm, shorten >= 3pt, shorten <= 3pt]

	\coordinate (c1);
	\coordinate[right of=c1] (c2);

	\draw[rectangle, draw=red!80, fill=red!80, align=center, rounded corners] ($(c1.north west)+(0,-0.3)$) rectangle ($(c2.south east)+(0, 0.3)$) {};
	\draw[->,white] (c1)[bend left] to node[auto] {} (c2);
	\draw[->,white] (c2)[bend left] to node[auto] {} (c1);
	\end{tikzpicture}
}

%Listing-Style pyhon
\title[Programmiervorkurs]{Programmiervorkurs Wintersemester \the\year/\NextYearShort}
\subtitle{{\small der Fachschaft Informatik}}
\logo*{\includegraphics{../globalMedia/bildmarke_ohne_rand}}
\institute{Fachschaft Informatik}
\date{Wintersemester \the\year/\NextYearShort}


% macros
\newcommand{\livecoding}{\begin{frame}\frametitle{\insertsectionhead \\  {\small \insertsubsectionhead}}\centering \huge \vskip 2cm\textbf{\textcolor{red}{Live-Coding}}\end{frame}}

%\newcommand{\slidehead}{\frametitle{\insertsectionhead \\ {\small \insertsubsectionhead}}\vspace{3mm}}
\newcommand{\slidehead}{\frametitle{\insertsectionhead} \framesubtitle{\insertsubsectionhead}\vspace{3mm}}
\newcommand{\tocslide}{\begin{frame}[t]\frametitle{Inhaltsverzeichnis}\vspace{3mm}{\small\tableofcontents[subsectionstyle=shaded]}\end{frame}}


% colors
\definecolor{lightpetrol}{RGB}{0,223,194}


\begin{document}

%Deckblatt
\subtitle{Kabitel 1: Erste Schritte}
\titlegraphic{
	\begin{columns}
		\begin{column}{5.5cm}
			\begin{tikzpicture}
				\draw (0, 0) node[inner sep=0] {\includegraphics[scale=0.265]{python.png}};
				% ich habe eine Mail an xkcd geschrieben, es ist ok das wir die klammern einfügen :)
				\draw (-1.95, -2.33) node {\tiny (};
				\draw (-1.02, -2.33) node {\tiny )};
			\end{tikzpicture}
			\\	\sffamily \tiny Bild: \href{https://xkcd.com/353/}{https://xkcd.com/353/ [edit: Kammern eingefügt für Python 3]}
		\end{column}
		\begin{column}{5cm}
			\begin{center}
				\vspace{-3.5cm}
				{\huge Erstes Program}
				\end{center}
			\end{column}
		\end{columns}
	}
\maketitle

\subsection{Erstes Program}
\begin{frame}
	% Beispiel, um zu zeigen wie so ein Befehl aussieht (und ein Kommentar)
	\slidehead
	\hskip .8cm
	\vspace{.5cm}
	\pythonfile{listings/HelloWorld.py}

	\pause

	\vspace{-2.55cm}
	\hskip .8cm
	\begin{tikzpicture}
		\draw [decorate,decoration={brace,amplitude=6pt}]
		(0,0.5) -- (3.7,0.5) node [black,midway,yshift=15pt,xshift=3.4cm] {Kommentar: wird vom Computer \textbf{ignoriert}, ist für Menschen \textbf{nützlich}};
	\end{tikzpicture}

	\pause

	\vspace{1.8cm}
	\hskip .8cm
	\begin{minipage}[b]{0.4\textwidth}
		\begin{tikzpicture}
			\draw [decorate,decoration={mirror,brace,amplitude=6pt}]
			(0,0.5) -- (1,0.5) node [black,midway,yshift=-15pt] {Befehl};
		\end{tikzpicture}
	\end{minipage}
	\pause
	\hskip -4.8cm
	\begin{minipage}[b]{0.4\textwidth}
		\begin{tikzpicture}
			\draw [decorate,decoration={mirror,brace,amplitude=6pt}]
			(1.1,0.5) -- (3.35,0.5) node [black,midway,yshift=-15pt] {Text};
		\end{tikzpicture}
	\end{minipage}

	\pause
	\vspace{0.25cm}
	\begin{block}{Wichtig}
		\begin{itemize}
			\item Text muss in \pythoninline{" "} oder in \pythoninline{' '} stehen
			\item Verwende \textbf{aussagekräftige} Kommentare \footnotesize (dein späteres Ich wird es dir danken)!
		\end{itemize}
	\end{block}
	\note{
		" und ' stehen nicht für Auslassungszeichen (Kopiere den Text von der Zeile oberhalb), sondern stehen für wörtliche Rede.
	}
\end{frame}


\subsection{Textausgabe}
\begin{frame}
	% Beispiel wie kleinkariert man mit diesem Computer reden muss - unintuitiv für Menschen, dass man nicht einfach "Enter" drücken kann, sondern so kyptisch "\n" schreiben muss
	\slidehead
	\begin{itemize}
		\item Mithilfe des Befehls \pythoninline{print()} kann Text auf der Konsole ausgeben werden
	\end{itemize}
	\begin{columns}
		\begin{column}{6cm}
			\pythonfile{listings/print_Listing.py}
		\end{column}
		\begin{column}{7cm}
			\pythonfile{listings/printSingleLine_Listing.py}
		\end{column}
	\end{columns}
	\vspace{0.25cm}
	\begin{block}{Merke}
		\pythoninline{print()} erzeugt einen Zeilenumbruch nach der Ausgabe
	\end{block}
\end{frame}

\subsection{Steuerzeichen}
\begin{frame}
	\slidehead

	\begin{table}[htbp]
		\begin{tabular}{|l|l|}
			\hline
			\textbf{Symbol} & \textbf{Wirkung} \\ \hline
			$\backslash$n & Zeilenumbruch \\ \hline
			$\backslash$\" & Doppeltes Anführungszeichen  (nur in \pythoninline{" "} nötig) \\ \hline
			$\backslash$' & Einfaches Anführungszeichen (nur in \pythoninline{' '} nötig) \\ \hline
			$\backslash \backslash$ & Backslash \\ \hline
		\end{tabular}
		\label{}
	\end{table}

	\pythonfile{listings/Steuerzeichen_Listing.py}
\end{frame}

\section{Vom Quelltext zum fertigen Programm}
\begin{frame}
	\slidehead

	\begin{columns}[T]
		\begin{column}{.4\textwidth}
			\begin{tikzpicture}
				\node[] (mensch) at (0,0) {\includegraphics[height=1.3cm]{media/wesen_notebook.png}};

				\node[fill=lightpetrol,draw, inner sep=5pt,rounded rectangle] (file) at (0,-1.8) {helloworld.py};
				\draw[->,line width=.6mm] (mensch.south) -- (file.north);

				\node[fill=lightpetrol,draw, inner sep=5pt,rounded rectangle] (python) at (0,-4) {\texttt{python3 helloworld.py}};
				\draw[->,line width=.6mm] (file.south) -- (python.north);
			\end{tikzpicture}
		\end{column}
		\begin{column}{.6\textwidth}
			\begin{itemize}
				\item Ausführen: \texttt{python3 dateiName.py}
				\item Python-Programme laufen überall, wo Python installiert ist
			\end{itemize}
			\vspace{2cm}
			\begin{block}{Hinweis}
				Python-Quelltext mit der Dateiendung \texttt{.py} kennzeichnen und sinnvoll benennen! Leerzeichen sind problematisch.
			\end{block}
		\end{column}
	\end{columns}
\end{frame}

\section{Ein Python-Programm schreiben}
\begin{frame}
	\slidehead

	\begin{center}
		\vskip -10 pt
		\includegraphics[scale=0.25]{HelloWorldImEditor}
	\end{center}

	\begin{itemize}
		\item Programm wird in einem herkömmlichen Editor geschrieben
		\item hier ein paar Editor Beispiele: \texttt{gedit}, \texttt{atom}, \texttt{kate}
	\end{itemize}
	\begin{alertblock}{Wichtig}
		Verwendet \textbf{keine} Textverarbeitung wie LibreOffice, Pages oder MS Word!
	\end{alertblock}
\end{frame}

\begin{frame}
	\slidehead
		Zusammen arbeiten hilft beim lernen, leider ist das aktuell aber etwas schwierig.\\
		Hier sind einige Möglichkeiten wie ihr das trotzdem könnt (alles freiwillig natürlich):
	\vspace{0.5cm}

	\begin{itemize}
		\item Atom plugin \textbf{Teletype}:\\ \href{https://teletype.atom.io/}{\textbf{https://teletype.atom.io/}}\\(Benötigt GitHub account)
		\item Alternativ gibt es online services wie:
		\begin{itemize}
			\item \href{https://codecollab.io/}{https://codecollab.io/} (Wählt Python\textbf{3} aus!)
			\item \href{https://repl.it/}{https://repl.it/}
		\end{itemize}
		Bei beiden gilt, wenn ihr euch keinen Account erstellt, solltet ihr euren code im Anschluss auf eurem Rechner herunter laden bzw. kopieren!
	\end{itemize}
	Ihr könnt euch während dessen natürlich auf Discord unterhalten.
\end{frame}

\section{Ein Python-Programm ausführen}
\begin{frame}[fragile]
	\slidehead
	\begin{bashcode}
		$ python3 helloworld.py
		Hello World!
		$
	\end{bashcode}

	\begin{itemize}
		\item Programm mit \texttt{python3 dateiName.py} starten
	\end{itemize}
\end{frame}

\subsection{Interaktiver Interpreter}
\begin{frame}[fragile]
	\slidehead

	\begin{bashcode}
		$ python3
		|\texttt{Python 3.7.0 (default, Jul 15 2018, 10:44:58)}|
		|\texttt{[GCC 8.1.1 20180531] on linux}|
		|\texttt{Type {\ditto}help{\ditto}, {\ditto}copyright{\ditto}, {\ditto}credits{\ditto} or {\ditto}license{\ditto} for more information.}|
		>>> print("Hello World!")
		Hello World!
		>>>
	\end{bashcode}

	\vskip -.5em

	\begin{itemize}
		\item Der Python-Interpreter hat einen interaktiven Modus
		\item Dieser kann mit dem Konsolenbefehl \texttt{python3} gestartet werden
		\item Anschließend können Python-Befehle eingegeben werden
		\item[]  %leerzeile
		\item \textbf{Strg+D} (außer Windows) oder der Befehl \pythoninline{quit()} beendet den Interpreter wieder
	\end{itemize}
\end{frame}

\livecoding
\subsection{}


\nextvid{Fehler}{Beispiel: Hello World}

\end{document}
