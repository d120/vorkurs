\RequirePackage[minted_workaround,caption_workaround,boxarc]{algo-common}

%include guard:
\ifdefined\GlobalAlreadyIncluded
    \expandafter\endinput
\fi

\gdef\GlobalAlreadyIncluded{}
%

\documentclass[accentcolor=3b,landscape,ngerman,presentation,t,usenames,dvipsnames,svgnames,table, aspectratio=169,colorframetitle=true,design=2008]{tudabeamer}

% Template-Modifikationen
\addtobeamertemplate{frametitle}{}{\vspace{-1em}} % mehr Platz vor dem Inhalt

% andere global gemeinsame definitionen
%Includes
\usepackage[ngerman]{babel} %Deutsche Silbentrennung
\usepackage[utf8]{inputenc} %Deutsche Umlaute
\usepackage{float}
\usepackage{graphicx}
\usepackage{minted}
\RequirePackage{csquotes}
\RequirePackage{fontawesome5}

\DeclareGraphicsExtensions{.pdf,.png,.jpg}

\makeatletter
\author{Vorkursteam der Fachschaft Informatik}
\let\Author\@author

% dark mode
\ExplSyntaxOn
\IfDarkModeT{
    \cs_if_exist:NT \setbeamercolor {
        \setbeamercolor*{smallrule}{bg=.}
        \setbeamercolor*{normal~text}{bg=\thepagecolor,fg=.}
        \setbeamercolor*{background~canvas}{parent=normal~text}
        \setbeamercolor*{section~in~toc}{parent=normal~text}
        \setbeamercolor*{subsection~in~toc}{parent=normal~text,fg=.}
        \setbeamercolor*{footline}{parent=normal~text}
        \setbeamercolor{block~title~alerted}{fg=white,bg=white!20!\thepagecolor}
        \setbeamercolor*{block~body}{bg=black!70!gray!98!blue}
        \setbeamercolor*{block~body~alerted}{bg=\thepagecolor}
    }
    \cs_if_exist:NT \setbeamertemplate {
        \setbeamertemplate{subsection~in~toc~shaded}[default][50]
    }
}
\ExplSyntaxOff

% macros
\renewcommand{\arraystretch}{1.2} % Höhe einer Tabellenspalte minimal erhöhen
\newcommand{\N}{{\mathbb N}}
\renewcommand{\code}{\inputminted[]{python}}

\IfDarkModeTF{
    \newmintedfile[pythonfile]{python}{
        fontsize=\small,
        style=native,
        linenos=true,
        numberblanklines=true,
        tabsize=4,
        obeytabs=false,
        breaklines=true,
        autogobble=true,
        encoding="utf8",
        showspaces=false,
        xleftmargin=20pt,
        frame=single,
        framesep=5pt,
    }
    \newmintinline{python}{
        style=native,
        encoding="utf8"
    }
    \newmintinline{kotlin}{
        style=native,
        encoding="utf8"
    }


    \definecolor{codegray}{HTML}{eaf1ff}
    \newminted[bashcode]{awk}{
        escapeinside=||,
        fontsize=\small,
        style=native,
        linenos=true,
        numberblanklines=true,
        tabsize=4,
        obeytabs=false,
        breaklines=true,
        autogobble=true,
        encoding="utf8",
        showspaces=false,
        xleftmargin=20pt,
        frame=single,
        framesep=5pt
    }
}{
    \newmintedfile[pythonfile]{python}{
        fontsize=\small,
        style=friendly,
        linenos=true,
        numberblanklines=true,
        tabsize=4,
        obeytabs=false,
        breaklines=true,
        autogobble=true,
        encoding="utf8",
        showspaces=false,
        xleftmargin=20pt,
        frame=single,
        framesep=5pt,
    }
    \newmintinline{python}{
        style=friendly,
        encoding="utf8"
    }
    \newmintinline{kotlin}{
        style=friendly,
        encoding="utf8"
    }

    \definecolor{codegray}{HTML}{eaf1ff}
    \newminted[bashcode]{awk}{
        escapeinside=||,
        fontsize=\small,
        style=friendly,
        linenos=true,
        numberblanklines=true,
        tabsize=4,
        obeytabs=false,
        breaklines=true,
        autogobble=true,
        encoding="utf8",
        showspaces=false,
        xleftmargin=20pt,
        frame=single,
        framesep=5pt
    }
}

\let\origpythonfile\pythonfile
\renewcommand{\pythonfile}[1]{\pythonfileh{#1}{}}
\newcommand{\pythonfileh}[2]{\origpythonfile[#2]{#1}}

\DeclareDocumentCommand{\kotlinfile}{O{} O{} m}{\inputCode[#1]{minted language=kotlin,#2}{#3}}

\newcommand*{\ditto}{\texttt{\char`\"}}

\newcommand{\shellprefix}{\textcolor{TUDa-3a}{\ttfamily\bfseries \$~}}
\DeclareTCBListing{commandshell}{ O{} O{} }{
    colback=\IfDarkModeTF{black}{black!80},
    colupper=white,
    colframe=TUDa-3a,
    listing only,
    % listing options={style=tcblatex,language=sh},
    listing engine=minted,
    minted style=dracula,
    minted options={
        % linenos=true,
        numbersep=3mm,
        texcl=true,
        autogobble,
        escapeinside=@@,
        breaklines,
        highlightcolor=yellow!50!black,
        #1
    },
    #2,
    % before upper={\textcolor{red}{\small\ttfamily\bfseries root \$> }},
    % every listing line={\textcolor{red}{\small\ttfamily\bfseries root \$> }}
}


%Includes
\usepackage{svg}
\usepackage{epstopdf}
\usepackage{wrapfig}
\usepackage{tipa}
\usepackage{tikz}
\usetikzlibrary{3d, angles, animations, arrows, arrows.meta, arrows.spaced, automata, babel, backgrounds, bending, calc, calendar, chains, circuits.ee.IEC, circuits.logic.CDH, circuits.logic.IEC, circuits.logic.US, datavisualization, datavisualization.formats.functions, datavisualization.polar, decorations, decorations.footprints, decorations.fractals, decorations.markings, decorations.pathmorphing, decorations.pathreplacing, decorations.shapes, decorations.text, er, external, fadings, fit, fixedpointarithmetic, folding, fpu, graphs, graphs.standard, intersections, lindenmayersystems, math, matrix, patterns, patterns.meta, perspective, petri, plotmarks, positioning, quotes, rdf, scopes, shadings, shadows, shadows.blur, shapes, shapes.arrows, shapes.callouts, shapes.gates.logic.IEC, shapes.gates.logic.US, shapes.geometric, shapes.misc, shapes.multipart, shapes.symbols, spy, svg.path, through, tikzmark, topaths, trees, turtle, views}
%tip: use http://l04.scarfboy.com/coding/phonetic-translation?from=ipa&fromtext=%CB%88pa%C9%AA%CE%B8n%CC%A9&to=tipa
%for converting ipa

\graphicspath{ {./media/} }

\def\shortyear#1{\expandafter\shortyearhelper#1}
\def\shortyearhelper#1#2#3#4{#3#4}

\newcount\NextYear
\NextYear = \year
\advance\NextYear by 1

\newcommand\NextYearShort{\shortyear{\the\NextYear}}

% notes
\usepackage{pgfpages}
\setbeamertemplate{note page}[plain]
%\setbeameroption{show notes on second screen}

% macro for change speaker sign
\newcommand{\changespeaker}{
    \begin{tikzpicture}[line width=.6mm, shorten >= 3pt, shorten <= 3pt]

        \coordinate (c1);
        \coordinate[right of=c1] (c2);

        \draw[rectangle, draw=red!80, fill=red!80, align=center, rounded corners] ($(c1.north west)+(0,-0.3)$) rectangle ($(c2.south east)+(0, 0.3)$) {};
        \draw[->,white] (c1)[bend left] to node[auto] {} (c2);
        \draw[->,white] (c2)[bend left] to node[auto] {} (c1);
    \end{tikzpicture}
}

%Listing-Style pyhon
\title[Programmiervorkurs]{Programmiervorkurs Wintersemester \the\year/\NextYearShort}
\subtitle{{\small der Fachschaft Informatik}}
\logo*{
    \IfDarkModeTF{
        \includesvg{../globalMedia/logo-dark.svg}
    }{
        \includesvg{../globalMedia/logo.svg}
    }
}
\institute{Fachschaft Informatik}
\date{Wintersemester \the\year/\NextYearShort}


% macros
\newcommand{\livecoding}{
    %\ifdefined\StreamSlides
    \begin{frame}[c]
        \slidehead
        \centering\huge\textbf{\textcolor{TUDa-9b}{Live-Coding}}
    \end{frame}
    %\fi
}

%\newcommand{\slidehead}{\frametitle{\insertsectionhead \\ {\small \insertsubsectionhead}}\vspace{3mm}}
\newcommand{\slidehead}{\frametitle{\insertsectionhead} \framesubtitle{\insertsubsectionhead}\vspace{3mm}}
\newcommand{\tocslide}{\begin{frame}[t]\frametitle{Inhaltsverzeichnis}\vspace{3mm}{\small\tableofcontents[subsectionstyle=shaded]}\end{frame}}

\newcommand{\nextvid}[2]{
    \ifdefined\StreamSlides
    \else
        \section{Nächstes Video}
        \begin{frame}[t]
            \slidehead
            \begin{block}{Nächstes Video}
                \vspace{0.5cm}
                #1
                \vspace{0.5cm}
            \end{block}
            \ifx\hfuzz#2\hfuzz
                \vspace{2.5cm}
                \else
                {\begin{block}{Bonus Video}
                        \vspace{0.5cm}
                        #2
                        \vspace{0.5cm}
                    \end{block}}
            \fi
            Danke fürs Zuschauen!\\
            Links zu den Folien und Quellen sind in der Videobeschreibung.
        \end{frame}
    \fi
}


\usepackage{verbatim}
\usetikzlibrary{decorations.pathreplacing}
\usetikzlibrary{shapes.misc}


% colors
\definecolor{lightpetrol}{RGB}{\IfDarkModeTF{0,136,119}{0,223,194}}

\RequirePackage{qrcode}

\DeclareDocumentCommand{\urlslide}{om}{
    \begin{frame}[c]
        \slidehead{}
        \begin{figure}
            \centering
            \qrcode[height=3cm]{#2}
            \caption{\IfNoValueF{#1}{#1\\}\url{#2}}
        \end{figure}
    \end{frame}
}

% Center Captions
\captionsetup[figure]{justification=centering}
\captionsetup[subfigure]{justification=centering}
\captionsetup[table]{justification=centering}
