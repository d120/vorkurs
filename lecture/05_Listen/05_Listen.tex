% !TeX document-id = {ce0a7a56-eb5e-4d6d-848f-46fb4968468e}
% !TeX TXS-program:compile = txs:///pdflatex/[--shell-escape]
\documentclass[accentcolor=3c,landscape,ngerman,presentation,t,usenames,dvipsnames,svgnames,table]{tudabeamer}

% Template-Modifikationen
\addtobeamertemplate{frametitle}{}{\vspace{-1em}} % mehr Platz vor dem Inhalt

% andere global gemeinsame definitionen
%Includes
\usepackage[ngerman]{babel} %Deutsche Silbentrennung
\usepackage[utf8]{inputenc} %Deutsche Umlaute
\usepackage{float}
\usepackage{graphicx}
\usepackage{minted}
\RequirePackage{csquotes}
\RequirePackage{fontawesome5}

\DeclareGraphicsExtensions{.pdf,.png,.jpg}

\makeatletter
\author{Vorkursteam der Fachschaft Informatik}
\let\Author\@author

% dark mode
\ExplSyntaxOn
\IfDarkModeT{
    \cs_if_exist:NT \setbeamercolor {
        \setbeamercolor*{smallrule}{bg=.}
        \setbeamercolor*{normal~text}{bg=\thepagecolor,fg=.}
        \setbeamercolor*{background~canvas}{parent=normal~text}
        \setbeamercolor*{section~in~toc}{parent=normal~text}
        \setbeamercolor*{subsection~in~toc}{parent=normal~text,fg=.}
        \setbeamercolor*{footline}{parent=normal~text}
        \setbeamercolor{block~title~alerted}{fg=white,bg=white!20!\thepagecolor}
        \setbeamercolor*{block~body}{bg=black!70!gray!98!blue}
        \setbeamercolor*{block~body~alerted}{bg=\thepagecolor}
    }
    \cs_if_exist:NT \setbeamertemplate {
        \setbeamertemplate{subsection~in~toc~shaded}[default][50]
    }
}
\ExplSyntaxOff

% macros
\renewcommand{\arraystretch}{1.2} % Höhe einer Tabellenspalte minimal erhöhen
\newcommand{\N}{{\mathbb N}}
\renewcommand{\code}{\inputminted[]{python}}

\IfDarkModeTF{
    \newmintedfile[pythonfile]{python}{
        fontsize=\small,
        style=native,
        linenos=true,
        numberblanklines=true,
        tabsize=4,
        obeytabs=false,
        breaklines=true,
        autogobble=true,
        encoding="utf8",
        showspaces=false,
        xleftmargin=20pt,
        frame=single,
        framesep=5pt,
    }
    \newmintinline{python}{
        style=native,
        encoding="utf8"
    }
    \newmintinline{kotlin}{
        style=native,
        encoding="utf8"
    }


    \definecolor{codegray}{HTML}{eaf1ff}
    \newminted[bashcode]{awk}{
        escapeinside=||,
        fontsize=\small,
        style=native,
        linenos=true,
        numberblanklines=true,
        tabsize=4,
        obeytabs=false,
        breaklines=true,
        autogobble=true,
        encoding="utf8",
        showspaces=false,
        xleftmargin=20pt,
        frame=single,
        framesep=5pt
    }
}{
    \newmintedfile[pythonfile]{python}{
        fontsize=\small,
        style=friendly,
        linenos=true,
        numberblanklines=true,
        tabsize=4,
        obeytabs=false,
        breaklines=true,
        autogobble=true,
        encoding="utf8",
        showspaces=false,
        xleftmargin=20pt,
        frame=single,
        framesep=5pt,
    }
    \newmintinline{python}{
        style=friendly,
        encoding="utf8"
    }
    \newmintinline{kotlin}{
        style=friendly,
        encoding="utf8"
    }

    \definecolor{codegray}{HTML}{eaf1ff}
    \newminted[bashcode]{awk}{
        escapeinside=||,
        fontsize=\small,
        style=friendly,
        linenos=true,
        numberblanklines=true,
        tabsize=4,
        obeytabs=false,
        breaklines=true,
        autogobble=true,
        encoding="utf8",
        showspaces=false,
        xleftmargin=20pt,
        frame=single,
        framesep=5pt
    }
}

\let\origpythonfile\pythonfile
\renewcommand{\pythonfile}[1]{\pythonfileh{#1}{}}
\newcommand{\pythonfileh}[2]{\origpythonfile[#2]{#1}}

\DeclareDocumentCommand{\kotlinfile}{O{} O{} m}{\inputCode[#1]{minted language=kotlin,#2}{#3}}

\newcommand*{\ditto}{\texttt{\char`\"}}

\newcommand{\shellprefix}{\textcolor{TUDa-3a}{\ttfamily\bfseries \$~}}
\DeclareTCBListing{commandshell}{ O{} O{} }{
    colback=\IfDarkModeTF{black}{black!80},
    colupper=white,
    colframe=TUDa-3a,
    listing only,
    % listing options={style=tcblatex,language=sh},
    listing engine=minted,
    minted style=dracula,
    minted options={
        % linenos=true,
        numbersep=3mm,
        texcl=true,
        autogobble,
        escapeinside=@@,
        breaklines,
        highlightcolor=yellow!50!black,
        #1
    },
    #2,
    % before upper={\textcolor{red}{\small\ttfamily\bfseries root \$> }},
    % every listing line={\textcolor{red}{\small\ttfamily\bfseries root \$> }}
}

%Includes
\usepackage{epstopdf}
\usepackage{wrapfig}
\usepackage{tipa}
\usepackage{tikz}
\usetikzlibrary{calc,shapes,arrows}
%tip: use http://l04.scarfboy.com/coding/phonetic-translation?from=ipa&fromtext=%CB%88pa%C9%AA%CE%B8n%CC%A9&to=tipa
%for converting ipa


\graphicspath{ {./media/} }

\def\shortyear#1{\expandafter\shortyearhelper#1}
\def\shortyearhelper#1#2#3#4{#3#4}

\newcount\NextYear
\NextYear = \year
\advance\NextYear by 1

\newcommand\NextYearShort{\shortyear{\the\NextYear}}

% notes
\usepackage{pgfpages}
\setbeamertemplate{note page}[plain]
%\setbeameroption{show notes on second screen}

% macro for change speaker sign
\newcommand{\changespeaker}{
	\begin{tikzpicture}[line width=.6mm, shorten >= 3pt, shorten <= 3pt]

	\coordinate (c1);
	\coordinate[right of=c1] (c2);

	\draw[rectangle, draw=red!80, fill=red!80, align=center, rounded corners] ($(c1.north west)+(0,-0.3)$) rectangle ($(c2.south east)+(0, 0.3)$) {};
	\draw[->,white] (c1)[bend left] to node[auto] {} (c2);
	\draw[->,white] (c2)[bend left] to node[auto] {} (c1);
	\end{tikzpicture}
}

%Listing-Style pyhon
\title[Programmiervorkurs]{Programmiervorkurs Wintersemester \the\year/\NextYearShort}
\subtitle{{\small der Fachschaft Informatik}}
\logo*{\includegraphics{../globalMedia/bildmarke_ohne_rand}}
\institute{Fachschaft Informatik}
\date{Wintersemester \the\year/\NextYearShort}


% macros
\newcommand{\livecoding}{\begin{frame}\frametitle{\insertsectionhead \\  {\small \insertsubsectionhead}}\centering \huge \vskip 2cm\textbf{\textcolor{red}{Live-Coding}}\end{frame}}

%\newcommand{\slidehead}{\frametitle{\insertsectionhead \\ {\small \insertsubsectionhead}}\vspace{3mm}}
\newcommand{\slidehead}{\frametitle{\insertsectionhead} \framesubtitle{\insertsubsectionhead}\vspace{3mm}}
\newcommand{\tocslide}{\begin{frame}[t]\frametitle{Inhaltsverzeichnis}\vspace{3mm}{\small\tableofcontents[subsectionstyle=shaded]}\end{frame}}


% colors
\definecolor{lightpetrol}{RGB}{0,223,194}


\begin{document}

%Deckblatt
\titlegraphic*{\includegraphics{media/cern.jpg}}
% Photo by Samuel Zeller on Unsplash.  Source: https://unsplash.com/photos/JuFcQxgCXwA Licence is completely free to do anything: "More precisely, Unsplash grants you an irrevocable, nonexclusive, worldwide copyright license to download, copy, modify, distribute, perform, and use photos from Unsplash for free, including for commercial purposes, without permission from or attributing the photographer or Unsplash."

\subtitle{Kapitel 5: ganz viele Variablen in einer}
\titlegraphic{
    \begin{columns}
        \begin{column}{4cm}
            \vspace{1.5cm}
            \begin{center}
                {\huge Listen}
            \end{center}
        \end{column}
        \begin{column}{4cm}
            \vspace{-4mm}
            \begin{figure}
                \centering
                \includegraphics[scale=.28]{media/tattoo_ideas\IfDarkModeT{_dark}.png}
                \\	\sffamily \tiny Bild: \href{https://xkcd.com/2255/}{https://xkcd.com/2255/}\label{fig:figure-xkcd-2255}
            \end{figure}
        \end{column}
    \end{columns}}
\maketitle


\section{Listen?}\label{sec:listen?}
\begin{frame}
    \slidehead
    \kotlinfile{listings/no_list.kts}
\end{frame}

\section{Adressierung}\label{sec:adressierung}
\begin{frame}
    \slidehead

    \begin{itemize}
        \item Liste: Kann mehrere Werte aufnehmen
        \item Metapher: Schrank mit vielen Schubladen
    \end{itemize}

    %list graphic
    %its beautiful
    %best regard
    %Kevin O.
    \begin{figure}[!h]
        \centering

        \begin{tikzpicture}
            \newcount\foo
            \foo=10
            \loop
            \draw[draw=., ultra thick] (11+\foo*0.5,0) rectangle ++(0.5,0.5);
            \advance \foo -1
            \node at (11.75+\foo*0.5, 0.75) {\the\foo};
            \ifnum \foo>0
            \repeat

            \draw [->, ultra thick] (12.5,-0.25) -- (11.5,-0.25);
            \node at (14, -0.25) {Listenlänge ist 10};
            \draw [->, ultra thick] (15.5,-0.25) -- (16.5,-0.25);

            \draw [-, ultra thick] (11.5,0.75) -- (11.2,0.75);
            \node at (10.5, 0.75) {Indizes};

            \draw [-, ultra thick] (16.25,0.25) -- (17.0,1);
            \node at (17.5, 1.25) {Element an Index 9};

        \end{tikzpicture}\label{fig:figure}

    \end{figure}
    \pause
    Beispiel:\\
    \kotlininline{eineListe = listOf(1, 2, 5, 9)}
    \begin{itemize}
        \item Der Wert 1 steht an Index 0
        \item Der Wert 2 steht an Index 1
        \item Der Wert 5 steht an Index 2 ...
        \item Die Liste hat eine Länge von 4 (via \kotlininline{eineListe.size})
    \end{itemize}
\end{frame}

% TODO: Put this somewhere
%\begin{frame}
%    \slidehead
%    \kotlinfile{listings/no_list_for.kts}
%\end{frame}

\section{Syntax}\label{sec:syntax}
\begin{frame}[c]
    \slidehead
    \kotlinfile[escapeinside=\$\$,texcomments]{listings/list_syntax_immutable_intro.kts}

    \begin{tikzpicture}[remember picture,overlay]
        \draw<2->[decorate,decoration={brace,amplitude=6pt}] ([yshift=1.25em]pic cs:typestart) -- ([yshift=1.25em]pic cs:typeend) node [midway,yshift=15pt] {Elemententyp};
        \draw<3->[decorate,decoration={brace,amplitude=6pt,mirror}] ([yshift=-0.75em]pic cs:elementsstart) -- ([yshift=-0.75em]pic cs:elementsend) node [midway,yshift=-15pt] {Elemente};
    \end{tikzpicture}
\end{frame}

\begin{frame}
    \slidehead
    \kotlinfile{listings/list_syntax_immutable_basic.kts}
\end{frame}

\begin{frame}
    \slidehead
    \kotlinfile[escapeinside=\$\$,texcomments]{listings/list_syntax_immutable_no_type.kts}

    \begin{tikzpicture}[remember picture,overlay]
        \node[draw, inner sep=0.75em, fill=accentcolor!50!\thepagecolor, rounded corners=3pt, above right=3.5em and -6em of pic cs:notype] (node) {Typ nicht nötig, wenn es Elemente gibt};
        \draw[thick, -Latex] (node) -- ([yshift=0.5em]pic cs:notype);
    \end{tikzpicture}
\end{frame}

\begin{frame}
    \slidehead
    \kotlinfile{listings/list_syntax_immutable_access.kts}
\end{frame}

\begin{frame}
    \slidehead
    \kotlinfile{listings/list_syntax_immutable_ends.kts}
\end{frame}

\livecoding

\section{Iterieren}\label{sec:iterieren}
\subsection{Zugriffsfehler}\label{subsec:zugriffsfehler}
\begin{frame}[fragile]
    \slidehead
    \kotlinfile{listings/out_of_bounds_code.kts}
    \pause
    \begin{commandshell}[fontsize=\footnotesize][minted language=text,top=0cm,bottom=0cm]
        java.lang.ArrayIndexOutOfBoundsException: Index 5 out of bounds for length 5
        at java.base/java.util.Arrays$ArrayList.get(Arrays.java:4165)
        at Foo.<init>(foo.kts:4)
    \end{commandshell}
    Bedeutung:
    \begin{itemize}
        \item In Zeile 4 der Datei \texttt{foo.kts} wurde ein Index angefordert, der nicht innerhalb der Liste liegt
    \end{itemize}
\end{frame}

\subsection{Mit For-Schleife}
\begin{frame}
    \slidehead
    \kotlinfile{listings/list_iteration.kts}
    \pause
    \begin{block}{Hinweis}
        Alles, was \enquote{iterable} ist, kann nach dem \kotlininline{in} (auch \kotlininline{String}!)
    \end{block}
\end{frame}

\livecoding

\section{In-Operator als Abfrage}\label{sec:in-operator-als-abfrage}

\begin{frame}
    \slidehead
    Der \kotlininline{in}-Operator kann auch überprüfen, ob sich ein Wert in der Liste befindet
    \pause
    \kotlinfile{listings/in_operator_list.kts}
    \pause
    \begin{block}{Hinweis}
        Alles, was \enquote{iterable} ist, kann nach dem \kotlininline{in} (auch \kotlininline{String}!)
    \end{block}
    \pause
\end{frame}

\begin{frame}
    \slidehead
    \kotlinfile{listings/in_operator_string.kts}
    \pause
    \vspace{2em}
    \centering
    \enquote{Das F in TU Darmstadt steht für Freizeit}
\end{frame}

\livecoding

\section{Destructuring}\label{sec:destructuring}
\begin{frame}
    \slidehead
    Mit destructuring können die ersten $n$ Werte aus einer Liste direkt in Variablen gespeichert werden
    \pause
    \kotlinfile[][top=0cm,bottom=0cm]{listings/destructure_list.kts}
    \pause
    \begin{block}{Hinweis}
        Das geht auch mit einem \kotlininline{String} (fast)
    \end{block}
\end{frame}

\begin{frame}
    \slidehead
    \kotlinfile[][top=0cm,bottom=0cm]{listings/destructure_string.kts}
\end{frame}

\livecoding

\section{Mutability}
\begin{frame}
    \slidehead
    \begin{itemize}[<+->]
        \item In Kotlin, ist fast alles \enquote{immutable-by-default}
        \item Die Liste, die von \kotlininline{listOf()} erstellt wird, kann nicht modifiziert werden
        \item Mit \kotlininline{mutableListOf()} kann eine Liste erstellt werden, die modifizierbar ist
    \end{itemize}
    \onslide<4->
    \kotlinfile{listings/list_mutability.kts}
    \begin{block}{Frage am Rande}
        Wie unterscheidet sich die \enquote{nicht-Veränderbarkeit} von \kotlininline{val} und \kotlininline{listOf()}?
    \end{block}
\end{frame}

\livecoding

\section{Listen-Slicing}
\begin{frame}
    \slidehead
    \begin{itemize}
        \item Listen können mittels Slicing zugeschnitten werden
            \pause
        \item Dazu wird \kotlininline{.slice(a until b)} hinter die Liste geschrieben
        \item Dabei ist \textbf{a} der Anfangsindex, und \textbf{b} der Endindex
            \pause
        \item Wichtig: Der Wert an Stelle \textbf{b} wird nicht mit kopiert
        \item Dies funktioniert auch mit \kotlininline{String}!
    \end{itemize}
    \pause

    \begin{block}{Beispiel:}
        \only<4>{\kotlinfile{listings/list_slice_no_solution.kts}}
        \only<5>{\kotlinfile{listings/list_slice.kts}}
    \end{block}
\end{frame}

\begin{frame}
    \slidehead
    Es ist möglich einzelne Indizes anzugeben
    \kotlinfile{listings/list_slice_list.kts}
\end{frame}

\livecoding

\section{Ranges}
\subsection{Deep Dive}
\begin{frame}
    \slidehead
    Ranges haben einen Start- und Endwert
    \begin{itemize}
        \item<1-> \kotlininline{a until b} ist ein Range von \kotlininline{a} bis \kotlininline{b - 1} inklusive
        \item<2-> \kotlininline{a .. b} ist ein Range von \kotlininline{a} bis \kotlininline{b} inklusive
        \item<3-> \kotlininline{a downTo b} ist ein Range von \kotlininline{a} bis \kotlininline{b} inklusive, falls \kotlininline{a > b}
        \item<4-> \kotlininline{a .. b step c} ist ein Range von \kotlininline{a} bis \kotlininline{b} inklusive, in sprüngen von \kotlininline{c}
    \end{itemize}
    \onslide<5->
    Ranges können auch zusammengesetzt werden
    \begin{itemize}
        \item<6-> \kotlininline{(a .. b) + (c .. d)}
        \item<7-> \kotlininline{(a .. d) - (b .. c)}
    \end{itemize}
    \onslide<8->
    Ranges können in eine Liste umgewandelt werden
    \begin{itemize}
        \item<9-> \kotlininline{(a .. b).toList()}
        \item<9-> \kotlininline{(a .. b).toMutableList()}
    \end{itemize}
\end{frame}

\begin{frame}
    \slidehead
    Beispiel: Gebe die Bustaben von \enquote{a} bis \enquote{c} und von \enquote{x} bis \enquote{z} aus
    \pause
    \kotlinfile{listings/range_char_example_plus.kts}
\end{frame}

\begin{frame}
    \slidehead
    Beispiel: Gebe alle Bustaben aus, außer die von \enquote{e} bis \enquote{m}
    \pause
    \kotlinfile{listings/range_char_example_minus.kts}
\end{frame}

\livecoding

\section{Listen kopieren}
\begin{frame}
    \slidehead
    \kotlinfile{listings/listcopy1.kts}
\end{frame}

\subsection{Analyse}
\begin{frame}
    \slidehead
    \vspace{2ex}
    \centering
    \begin{tikzpicture}
        \node[draw, thick] at (-1,0) {\kotlininline{meineListe}};
        \node[draw, thick] at (1,0) {\kotlininline{kopie}};

        \node[draw, thick, fill=lightpetrol] at (0,2) {Speicherbereich};

        \draw [->, ultra thick] (-1, 0.3) -- (-0.1,1.7);
        \draw [->, ultra thick] (1, 0.3) -- (0.1,1.7);
    \end{tikzpicture}

    \vspace{2ex}
    \begin{itemize}
        \item Bei der Zuweisung mit \textbf{=} wurde nur die Zieladresse
            kopiert
        \item Deshalb wird die gleiche Liste referenziert
        \item Eine Änderung von \kotlininline{kopie} bewirkt eine Änderung in \kotlininline{meineListe} (und umgekehrt)
    \end{itemize}
\end{frame}


\begin{frame}
    \slidehead
    \kotlinfile{listings/listcopy2.kts}
\end{frame}

\subsection*{Analyse}
\begin{frame}
    \slidehead

    \vspace{2ex}
    \centering
    \begin{tikzpicture}
        \node[draw, thick] at (-2,0) {\kotlininline{meineListe}};
        \node[draw, thick] at (2,0) {\kotlininline{kopie}};


        \node[draw, thick, fill=lightpetrol] at (-2,2) {Speicherbereich};

        \node[draw, thick, fill=lightpetrol] at (2,2) {Speicherbereich 2};

        \draw [->, ultra thick] (-2, 0.25) -- (-2,1.75);
        \draw [->, ultra thick] (2, 0.25)  -- (2,1.75);
    \end{tikzpicture}
    \vspace{2ex}

    \begin{itemize}
        \item \kotlininline{kopie} liegt nun in einem neuen, unabhängigen Speicherbereich
    \end{itemize}
\end{frame}

\livecoding

%\nextvid{Mehrdimensionale Listen}{Listen kopieren Beispiel}

%Deckblatt
\subtitle{Kapitel 5: ganz viele Variablen in einer}

% \titlegraphic{
% 	\begin{columns}
% 		\begin{column}{10cm}
% 			\vspace{1.5cm}
% 			\begin{center}
% 				{\huge Mehrdimensionale Listen}
% 			\end{center}
% 		\end{column}
% 	\end{columns}}
% \subtitle{Listen}
% \maketitle

\section{Mehrdimensionale Listen}
\begin{frame}
    \newcount\foo
    \slidehead
    \begin{columns}[T]
        \begin{column}{4cm}
            \centering
            \begin{tikzpicture}
                \foo=2
                \loop
                \advance \foo -1
                \node at (0.75+\foo*0.5, 2.25) {\the\foo};
                \ifnum \foo>0
                \repeat

                \foo=3
                \loop
                \advance \foo -1
                \node at (0.25, 1.75-\foo*0.5) {\the\foo};
                \ifnum \foo>0
                \repeat

                \node at (0.75, 1.75-\foo*0.5) {8};
                \node at (0.75, 1.25-\foo*0.5) {0};
                \node at (0.75, 0.75-\foo*0.5) {5};

                \node at (1.25, 1.75-\foo*0.5) {2};
                \node at (1.25, 1.25-\foo*0.5) {1};
                \node at (1.25, 0.75-\foo*0.5) {4};

                \newcount\X
                \newcount\Y
                \loop
                \Y = 0
                \advance \X by 1
                {%
                    \loop
                    \advance \Y by 1

                    \draw[draw=., ultra thick] 	(\X*0.5,\Y*0.5) 	rectangle ++(0.5,0.5);

                    \ifnum \Y <3
                    \repeat
                }%
                \ifnum \X < 2
                \repeat
            \end{tikzpicture}
        \end{column}
        \begin{column}{7cm}
            \begin{itemize}
                \item Definition: \pythoninline{square =[[8, 0, 5], [2, 1, 4]]}
                \item Jede Dimension stellt man durch einen eigenen Index dar
                \item Adressierung: \pythoninline{square[Spalte][Zeile]}
                \item Länge:\\
                    1. Dimension: \pythoninline{len(square)}\\
                    2. Dimension: \pythoninline{len(square[0])} oder \pythoninline{len(square[1])}, \dots
                \item Nicht jede Spalte muss gleich viele Zeilen haben
            \end{itemize}
        \end{column}
    \end{columns}
\end{frame}

\subsection{Initialisieren}
\begin{frame}
    \slidehead
    \vskip -12pt
    \pythonfile{listings/mehrDimList_Initialisierung_Listing.py}
\end{frame}

\livecoding

\section{Quiz}
\begin{frame}
    \slidehead
    \begin{itemize}
        \item Wie erhält man Zugriff auf die Länge einer Liste?
            \pause
        \item Wie greife ich auf das 5. Element einer Liste zu?
            \pause
        \item Wie wird eine zweidimensionale Liste adressiert? (Stichwort: Zeile/Spalte)
            \pause
        \item Wofür benötigen wir \pythoninline{copy()} und warum erzielt eine einfache Zuweisung nicht den selben Effekt?
            \pause
        \item Wie schneide ich eine Liste hinter dem 3. Element ab?
    \end{itemize}
\end{frame}

%\nextvid{Funktionen}{Mehrdimensionale Listen Beispiele}

\end{document}
