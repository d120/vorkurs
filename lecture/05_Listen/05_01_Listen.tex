% !TeX document-id = {ce0a7a56-eb5e-4d6d-848f-46fb4968468e}
% !TeX TXS-program:compile = txs:///pdflatex/[--shell-escape]
\documentclass[accentcolor=3c,landscape,ngerman,presentation,t,usenames,dvipsnames,svgnames,table]{tudabeamer}

% Template-Modifikationen
\addtobeamertemplate{frametitle}{}{\vspace{-1em}} % mehr Platz vor dem Inhalt

% andere global gemeinsame definitionen
%Includes
\usepackage[ngerman]{babel} %Deutsche Silbentrennung
\usepackage[utf8]{inputenc} %Deutsche Umlaute
\usepackage{float}
\usepackage{graphicx}
\usepackage{minted}
\RequirePackage{csquotes}
\RequirePackage{fontawesome5}

\DeclareGraphicsExtensions{.pdf,.png,.jpg}

\makeatletter
\author{Vorkursteam der Fachschaft Informatik}
\let\Author\@author

% dark mode
\ExplSyntaxOn
\IfDarkModeT{
    \cs_if_exist:NT \setbeamercolor {
        \setbeamercolor*{smallrule}{bg=.}
        \setbeamercolor*{normal~text}{bg=\thepagecolor,fg=.}
        \setbeamercolor*{background~canvas}{parent=normal~text}
        \setbeamercolor*{section~in~toc}{parent=normal~text}
        \setbeamercolor*{subsection~in~toc}{parent=normal~text,fg=.}
        \setbeamercolor*{footline}{parent=normal~text}
        \setbeamercolor{block~title~alerted}{fg=white,bg=white!20!\thepagecolor}
        \setbeamercolor*{block~body}{bg=black!70!gray!98!blue}
        \setbeamercolor*{block~body~alerted}{bg=\thepagecolor}
    }
    \cs_if_exist:NT \setbeamertemplate {
        \setbeamertemplate{subsection~in~toc~shaded}[default][50]
    }
}
\ExplSyntaxOff

% macros
\renewcommand{\arraystretch}{1.2} % Höhe einer Tabellenspalte minimal erhöhen
\newcommand{\N}{{\mathbb N}}
\renewcommand{\code}{\inputminted[]{python}}

\IfDarkModeTF{
    \newmintedfile[pythonfile]{python}{
        fontsize=\small,
        style=native,
        linenos=true,
        numberblanklines=true,
        tabsize=4,
        obeytabs=false,
        breaklines=true,
        autogobble=true,
        encoding="utf8",
        showspaces=false,
        xleftmargin=20pt,
        frame=single,
        framesep=5pt,
    }
    \newmintinline{python}{
        style=native,
        encoding="utf8"
    }
    \newmintinline{kotlin}{
        style=native,
        encoding="utf8"
    }


    \definecolor{codegray}{HTML}{eaf1ff}
    \newminted[bashcode]{awk}{
        escapeinside=||,
        fontsize=\small,
        style=native,
        linenos=true,
        numberblanklines=true,
        tabsize=4,
        obeytabs=false,
        breaklines=true,
        autogobble=true,
        encoding="utf8",
        showspaces=false,
        xleftmargin=20pt,
        frame=single,
        framesep=5pt
    }
}{
    \newmintedfile[pythonfile]{python}{
        fontsize=\small,
        style=friendly,
        linenos=true,
        numberblanklines=true,
        tabsize=4,
        obeytabs=false,
        breaklines=true,
        autogobble=true,
        encoding="utf8",
        showspaces=false,
        xleftmargin=20pt,
        frame=single,
        framesep=5pt,
    }
    \newmintinline{python}{
        style=friendly,
        encoding="utf8"
    }
    \newmintinline{kotlin}{
        style=friendly,
        encoding="utf8"
    }

    \definecolor{codegray}{HTML}{eaf1ff}
    \newminted[bashcode]{awk}{
        escapeinside=||,
        fontsize=\small,
        style=friendly,
        linenos=true,
        numberblanklines=true,
        tabsize=4,
        obeytabs=false,
        breaklines=true,
        autogobble=true,
        encoding="utf8",
        showspaces=false,
        xleftmargin=20pt,
        frame=single,
        framesep=5pt
    }
}

\let\origpythonfile\pythonfile
\renewcommand{\pythonfile}[1]{\pythonfileh{#1}{}}
\newcommand{\pythonfileh}[2]{\origpythonfile[#2]{#1}}

\DeclareDocumentCommand{\kotlinfile}{O{} O{} m}{\inputCode[#1]{minted language=kotlin,#2}{#3}}

\newcommand*{\ditto}{\texttt{\char`\"}}

\newcommand{\shellprefix}{\textcolor{TUDa-3a}{\ttfamily\bfseries \$~}}
\DeclareTCBListing{commandshell}{ O{} O{} }{
    colback=\IfDarkModeTF{black}{black!80},
    colupper=white,
    colframe=TUDa-3a,
    listing only,
    % listing options={style=tcblatex,language=sh},
    listing engine=minted,
    minted style=dracula,
    minted options={
        % linenos=true,
        numbersep=3mm,
        texcl=true,
        autogobble,
        escapeinside=@@,
        breaklines,
        highlightcolor=yellow!50!black,
        #1
    },
    #2,
    % before upper={\textcolor{red}{\small\ttfamily\bfseries root \$> }},
    % every listing line={\textcolor{red}{\small\ttfamily\bfseries root \$> }}
}

%Includes
\usepackage{epstopdf}
\usepackage{wrapfig}
\usepackage{tipa}
\usepackage{tikz}
\usetikzlibrary{calc,shapes,arrows}
%tip: use http://l04.scarfboy.com/coding/phonetic-translation?from=ipa&fromtext=%CB%88pa%C9%AA%CE%B8n%CC%A9&to=tipa
%for converting ipa


\graphicspath{ {./media/} }

\def\shortyear#1{\expandafter\shortyearhelper#1}
\def\shortyearhelper#1#2#3#4{#3#4}

\newcount\NextYear
\NextYear = \year
\advance\NextYear by 1

\newcommand\NextYearShort{\shortyear{\the\NextYear}}

% notes
\usepackage{pgfpages}
\setbeamertemplate{note page}[plain]
%\setbeameroption{show notes on second screen}

% macro for change speaker sign
\newcommand{\changespeaker}{
	\begin{tikzpicture}[line width=.6mm, shorten >= 3pt, shorten <= 3pt]

	\coordinate (c1);
	\coordinate[right of=c1] (c2);

	\draw[rectangle, draw=red!80, fill=red!80, align=center, rounded corners] ($(c1.north west)+(0,-0.3)$) rectangle ($(c2.south east)+(0, 0.3)$) {};
	\draw[->,white] (c1)[bend left] to node[auto] {} (c2);
	\draw[->,white] (c2)[bend left] to node[auto] {} (c1);
	\end{tikzpicture}
}

%Listing-Style pyhon
\title[Programmiervorkurs]{Programmiervorkurs Wintersemester \the\year/\NextYearShort}
\subtitle{{\small der Fachschaft Informatik}}
\logo*{\includegraphics{../globalMedia/bildmarke_ohne_rand}}
\institute{Fachschaft Informatik}
\date{Wintersemester \the\year/\NextYearShort}


% macros
\newcommand{\livecoding}{\begin{frame}\frametitle{\insertsectionhead \\  {\small \insertsubsectionhead}}\centering \huge \vskip 2cm\textbf{\textcolor{red}{Live-Coding}}\end{frame}}

%\newcommand{\slidehead}{\frametitle{\insertsectionhead \\ {\small \insertsubsectionhead}}\vspace{3mm}}
\newcommand{\slidehead}{\frametitle{\insertsectionhead} \framesubtitle{\insertsubsectionhead}\vspace{3mm}}
\newcommand{\tocslide}{\begin{frame}[t]\frametitle{Inhaltsverzeichnis}\vspace{3mm}{\small\tableofcontents[subsectionstyle=shaded]}\end{frame}}


% colors
\definecolor{lightpetrol}{RGB}{0,223,194}


\begin{document}

%Deckblatt
\titlegraphic*{\includegraphics{media/cern.jpg}}
% Photo by Samuel Zeller on Unsplash.  Source: https://unsplash.com/photos/JuFcQxgCXwA Licence is completely free to do anything: "More precisely, Unsplash grants you an irrevocable, nonexclusive, worldwide copyright license to download, copy, modify, distribute, perform, and use photos from Unsplash for free, including for commercial purposes, without permission from or attributing the photographer or Unsplash."

\subtitle{Kapitel 5: ganz viele variablen in einer}
\titlegraphic{
	\begin{columns}
		\begin{column}{4cm}
			\vspace{1.5cm}
			\begin{center}
				{\huge Listen}
			\end{center}
		\end{column}
		\begin{column}{4cm}
		\vspace{-4mm}
			\begin{figure}
				\centering
			\includegraphics[scale=.28]{media/tattoo_ideas.png}
				\\	\sffamily \tiny Bild: \href{https://xkcd.com/2255/}{https://xkcd.com/2255/}
			\end{figure}
		\end{column}
	\end{columns}}
\maketitle


\section{Listen?}
\begin{frame}
	\slidehead
	\pythonfile{listings/no_list.py}
\end{frame}

\begin{frame}
	\slidehead
	\pythonfile{listings/no_list_for.py}
\end{frame}


\section{Adressierung}
\begin{frame}
	\slidehead

	\begin{itemize}
		\item Liste: Kann mehrere Werte aufnehmen
		\item Metapher: Schuhkarton mit vielen Fächern
	\end{itemize}

	%list graphic
	%its beautiful
	%best regard
	%Kevin O.
	\begin{figure}[!h]
		\centering

		\begin{tikzpicture}
			\newcount\foo
			\foo=10
			\loop
			\draw[draw=black, ultra thick] (11+\foo*0.5,0) rectangle ++(0.5,0.5);
			\advance \foo -1
			\node at (11.75+\foo*0.5, 0.75) {\the\foo};
			\ifnum \foo>0
			\repeat

			\draw [->, ultra thick] (12.5,-0.25) -- (11.5,-0.25);
			\node at (14, -0.25) {Listenlänge ist 10};
			\draw [->, ultra thick] (15.5,-0.25) -- (16.5,-0.25);

			\draw [-, ultra thick] (11.5,0.75) -- (11.2,0.75);
			\node at (10.5, 0.75) {Indizes};

			\draw [-, ultra thick] (16.25,0.25) -- (17.0,1);
			\node at (17.5, 1.25) {Element an Index 9};

		\end{tikzpicture}

	\end{figure}
	\pause
	Beispiel:\\
	\pythoninline{eineListe = [1, 2, 5, 9]}
	\begin{itemize}
		\item Der Wert 1 steht an Index 0
		\item Der Wert 2 steht an Index 1
		\item Der Wert 5 steht an Index 2 ...
		\item Die Liste hat eine Länge von 4
	\end{itemize}
\end{frame}

\subsection{Negative Adressierung}
\begin{frame}
	\slidehead
	\begin{itemize}
		\item Der Index -1 gibt das letzte Element zurück
		\item -2 das zweite Element von hinten
		\item ...
	\end{itemize}
\end{frame}

\section{Syntax}
\begin{frame}
	\slidehead
	\only<1>{\pythonfile{listings/list_syntax_no_solution.py}}
	\only<2>{\pythonfile{listings/list_syntax.py}}
\end{frame}

\section{Zugriffsfehler}
\begin{frame}
	\slidehead
	\pythonfile{listings/OutofBounce_Listing.py}
	Bedeutung:
	\begin{itemize}
		\item In Zeile 3 der Datei \texttt{dateiName.py} wurde ein Index angefordert, der nicht innerhalb der Liste liegt
	\end{itemize}

\end{frame}

\nextvid{Nützliche Listen-Funktionen}{Listen Beispiele}

\end{document}
