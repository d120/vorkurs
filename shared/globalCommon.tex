%Includes
\usepackage[ngerman]{babel} %Deutsche Silbentrennung
\usepackage[utf8]{inputenc} %Deutsche Umlaute
\usepackage{float}
\usepackage{graphicx}
\usepackage{minted}
\RequirePackage{csquotes}
\RequirePackage{fontawesome5}

\DeclareGraphicsExtensions{.pdf,.png,.jpg}

\makeatletter
\author{Vorkursteam der Fachschaft Informatik}
\let\Author\@author

% dark mode
\ExplSyntaxOn
\IfDarkModeT{
    \cs_if_exist:NT \setbeamercolor {
        \setbeamercolor*{smallrule}{bg=.}
        \setbeamercolor*{normal~text}{bg=\thepagecolor,fg=.}
        \setbeamercolor*{background~canvas}{parent=normal~text}
        \setbeamercolor*{section~in~toc}{parent=normal~text}
        \setbeamercolor*{subsection~in~toc}{parent=normal~text,fg=.}
        \setbeamercolor*{footline}{parent=normal~text}
        \setbeamercolor{block~title~alerted}{fg=white,bg=white!20!\thepagecolor}
        \setbeamercolor*{block~body}{bg=black!70!gray!98!blue}
        \setbeamercolor*{block~body~alerted}{bg=\thepagecolor}
    }
    \cs_if_exist:NT \setbeamertemplate {
        \setbeamertemplate{subsection~in~toc~shaded}[default][50]
    }
}
\ExplSyntaxOff

% macros
\renewcommand{\arraystretch}{1.2} % Höhe einer Tabellenspalte minimal erhöhen
\newcommand{\N}{{\mathbb N}}
\renewcommand{\code}{\inputminted[]{python}}

\IfDarkModeTF{
    \newmintedfile[pythonfile]{python}{
        fontsize=\small,
        style=native,
        linenos=true,
        numberblanklines=true,
        tabsize=4,
        obeytabs=false,
        breaklines=true,
        autogobble=true,
        encoding="utf8",
        showspaces=false,
        xleftmargin=20pt,
        frame=single,
        framesep=5pt,
    }
    \newmintinline{python}{
        style=native,
        encoding="utf8"
    }
    \newmintinline{kotlin}{
        style=native,
        encoding="utf8"
    }


    \definecolor{codegray}{HTML}{eaf1ff}
    \newminted[bashcode]{awk}{
        escapeinside=||,
        fontsize=\small,
        style=native,
        linenos=true,
        numberblanklines=true,
        tabsize=4,
        obeytabs=false,
        breaklines=true,
        autogobble=true,
        encoding="utf8",
        showspaces=false,
        xleftmargin=20pt,
        frame=single,
        framesep=5pt
    }
}{
    \newmintedfile[pythonfile]{python}{
        fontsize=\small,
        style=friendly,
        linenos=true,
        numberblanklines=true,
        tabsize=4,
        obeytabs=false,
        breaklines=true,
        autogobble=true,
        encoding="utf8",
        showspaces=false,
        xleftmargin=20pt,
        frame=single,
        framesep=5pt,
    }
    \newmintinline{python}{
        style=friendly,
        encoding="utf8"
    }
    \newmintinline{kotlin}{
        style=friendly,
        encoding="utf8"
    }

    \definecolor{codegray}{HTML}{eaf1ff}
    \newminted[bashcode]{awk}{
        escapeinside=||,
        fontsize=\small,
        style=friendly,
        linenos=true,
        numberblanklines=true,
        tabsize=4,
        obeytabs=false,
        breaklines=true,
        autogobble=true,
        encoding="utf8",
        showspaces=false,
        xleftmargin=20pt,
        frame=single,
        framesep=5pt
    }
}

\let\origpythonfile\pythonfile
\renewcommand{\pythonfile}[1]{\pythonfileh{#1}{}}
\newcommand{\pythonfileh}[2]{\origpythonfile[#2]{#1}}

\DeclareDocumentCommand{\kotlinfile}{O{} O{} m}{\inputCode[#1]{minted language=kotlin,minted style/.expanded=\IfDarkModeTF{native}{friendly},#2}{#3}}

\newcommand*{\ditto}{\texttt{\char`\"}}

\RequirePackage{caption,subcaption}

\newcommand{\shellprefix}{\textcolor{TUDa-3a}{\ttfamily\bfseries \$~}}
\newcommand{\shellcursor}{\tikz[baseline=-.5ex]{\node[fill,text height=1em, text width=1.5mm, inner sep=0pt,outer sep=0pt]{};}}
\DeclareTCBListing{commandshell}{ O{} O{} }{
    colback=\IfDarkModeTF{black}{black!80},
    colupper=white,
    colframe=TUDa-3a,
    listing only,
    % listing options={style=tcblatex,language=sh},
    listing engine=minted,
    minted style=native,
    minted options={
        % linenos=true,
        numbersep=3mm,
        texcl=true,
        autogobble,
        escapeinside=@@,
        breaklines,
        highlightcolor=yellow!50!black,
        #1
    },
    #2,
    % before upper={\textcolor{red}{\small\ttfamily\bfseries root \$> }},
    % every listing line={\textcolor{red}{\small\ttfamily\bfseries root \$> }}
}

\ExplSyntaxOn
\DeclareTCBListing{codeBlock}{ O{} m }{
    listing~engine=minted, % Minted verwenden
    colback=\IfDarkModeTF{codebg}{black!10!\thepagecolor}, %Hintergrundfarbe
    colframe=black!70, % Randfarbe
    coltext=.,
    listing~only,  % Sonst will er den Plain Text nach dem Minted Listing noch anfügen
    frame~hidden,
    boxrule=0pt,
    arc=\rubos@boxarc,
    %hbox, % This option could be used to limit the Length of Code Blocks automatically, but does not work with the minted Line Numbers
    title~style=\IfDarkModeTF{accentcolor!60!black}{accentcolor},
    minted~style/.expanded=\IfDarkModeTF{native}{tango}, % Highlighting Style anpassen
    minted~language=java, % Sprache setzen
    minted~options={ %Minted Optionen
        linenos=true,
        numbersep=3mm,
        texcl=true,
        autogobble,
        escapeinside=@@,
        breaklines,
        highlightcolor=\IfDarkModeTF{yellow!50!black}{yellow!80},
        #1 % weitere optionen für Minted zulassen
    },
    left=7.1mm, % Links Platz lassen
    enhanced, % Erlaubt uns, den ramen zu zeichnen
    fonttitle=\sffamily, % Titelschriftart auf
    overlay={ % Für Grauen Bereich links
        \begin{tcbclipinterior}
            \fill[black!70] (frame.south~west) rectangle ([xshift=5mm]frame.north~west); % Zeilennummernbereich färben
        \end{tcbclipinterior}
    },
    #2 % Weitere Argumente zulassen
}

\NewTCBInputListing{\inputErrorCode}{ O{} m }{
    listing~file={#2},
    coltext=red,
    listing~only,  % Sonst will er den Plain Text nach dem Minted Listing noch anfügen
    colback=\IfDarkModeTF{black}{black!80},
    colupper=white,
    colframe=TUDa-3a,
    before~upper=\color{TUDa-9b},
    #1 % Weitere Argumente zulassen
}
\ExplSyntaxOff
