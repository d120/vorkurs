%Includes
\usepackage[ngerman]{babel} %Deutsche Silbentrennung
\usepackage[utf8]{inputenc} %Deutsche Umlaute
\usepackage{float}
\usepackage{graphicx}
\usepackage{minted}

\DeclareGraphicsExtensions{.pdf,.png,.jpg}

\makeatletter
\author{Vorkursteam der Fachschaft Informatik}
\let\Author\@author

% dark Mode
\ExplSyntaxOn
\RequirePackage{pagecolor,xcolor, graphicx} % Used for dark Mode
\bool_gset_false:N \g_dark_mode_bool % Disable by default
\definecolor{anthrazitgrau}{HTML}{293133}
\newcommand{\enableDarkMode}[1][anthrazitgrau]{ %Command to enable Dark Mode (only works before \begin{document})
	\pagecolor{#1}
	\color{white}

	\cs_if_exist:NT \setbeamercolor {
		\setbeamercolor*{smallrule}{bg=.}
		\setbeamercolor*{normal~text}{bg=,fg=.}
		\setbeamercolor*{background canvas}{parent=normal~text}
		\setbeamercolor*{section~in~toc}{parent=normal~text}
		\setbeamercolor*{subsection~in~toc}{parent=normal~text,fg=\thepagecolor}
		\setbeamercolor*{footline}{parent=normal~text}
		\setbeamercolor{block~title~alerted}{fg=white,bg=white!20!\thepagecolor}
		\setbeamercolor*{block~body}{bg=black!70!gray!98!blue}
		\setbeamercolor*{block~body~alerted}{bg=\thepagecolor}
	}
	\cs_if_exist:NT \setbeamertemplate {
		\setbeamertemplate{subsection~in~toc~shaded}[default][50]
	}
	\bool_gset_true:N \g_dark_mode_bool
	% Prefer inverted Logo with dark Mode
	% \IfFileExists{tuda_logo_inverted.pdf}{\tl_gset:Nn \g_ptxcd_logofile_tl {tuda_logo_inverted.pdf}}{}
	% \hbox_gset:Nn \g__ptxcd_logo_box {% Update Logo Box
	% 	\makebox[2.2\c_ptxcd_logoheight_dim][l]{\includegraphics[height=\c_ptxcd_logoheight_dim]{\g_ptxcd_logofile_tl}}%
	% }
}

% dark Mode Makros
\prg_new_conditional:Nnn \__ptxcd_if_dark_mode: {T,F,TF} { % Conditional to check if dark Mode is active
	\bool_if:NTF \g_dark_mode_bool
	{\prg_return_true:}
	{\prg_return_false:}
}

\cs_set_eq:NN\IfDarkModeT \__ptxcd_if_dark_mode:T % Easy dark Mode check for use in document
\cs_set_eq:NN\IfDarkModeF \__ptxcd_if_dark_mode:F
\cs_set_eq:NN\IfDarkModeTF \__ptxcd_if_dark_mode:TF

\newcommand{\includeinvertablegraphics}[2][]{% Grafik wird beim Dark Mode automatisch Invertiert (rgb)
	\IfDarkModeTF{\includegraphics[decodearray={1.0~0.0~1.0~0.0~1.0~0.0},#1]{#2}}{\includegraphics[#1]{#2}}
}
\newcommand{\includeinvertablegrayscalegraphics}[2][]{% Grafik wird beim Dark Mode automatisch Invertiert (grayscale)
	\IfDarkModeTF{\includegraphics[decodearray={1.0~0.0},#1]{#2}}{\includegraphics[#1]{#2}}
}

% DARK_MODE environment check (enable if DARK_MODE=1)
\sys_get_shell:nnN { kpsewhich ~ --var-value ~ DARK_MODE } { } \l_dark_mode_env_var_tl
\tl_trim_spaces:N \l_dark_mode_env_var_tl
\tl_if_eq:NnT \l_dark_mode_env_var_tl {1} {\enableDarkMode{}}
\tl_if_eq:NnT \l_dark_mode_env_var_tl {2} {\enableDarkMode[black]{}}
\ExplSyntaxOff

% macros
\renewcommand{\arraystretch}{1.2} % Höhe einer Tabellenspalte minimal erhöhen
\newcommand{\N}{{\mathbb N}}
\newcommand{\code}{\inputminted[]{python}}
\IfDarkModeTF{
	\newmintedfile[pythonfile]{python}{
		fontsize=\small,
		style=native,
		linenos=true,
		numberblanklines=true,
		tabsize=4,
		obeytabs=false,
		breaklines=true,
		autogobble=true,
		encoding="utf8",
		showspaces=false,
		xleftmargin=20pt,
		frame=single,
		framesep=5pt,
	}
	\newmintinline{python}{
		style=native,
		encoding="utf8"
	}
	
	\definecolor{codegray}{HTML}{eaf1ff}
	\newminted[bashcode]{awk}{
		escapeinside=||,
		fontsize=\small,
		style=native,
		linenos=true,
		numberblanklines=true,
		tabsize=4,
		obeytabs=false,
		breaklines=true,
		autogobble=true,
		encoding="utf8",
		showspaces=false,
		xleftmargin=20pt,
		frame=single,
		framesep=5pt
	}
}{
	\newmintedfile[pythonfile]{python}{
		fontsize=\small,
		style=friendly,
		linenos=true,
		numberblanklines=true,
		tabsize=4,
		obeytabs=false,
		breaklines=true,
		autogobble=true,
		encoding="utf8",
		showspaces=false,
		xleftmargin=20pt,
		frame=single,
		framesep=5pt,
	}
	\newmintinline{python}{
		style=friendly,
		encoding="utf8"
	}
	
	\definecolor{codegray}{HTML}{eaf1ff}
	\newminted[bashcode]{awk}{
		escapeinside=||,
		fontsize=\small,
		style=friendly,
		linenos=true,
		numberblanklines=true,
		tabsize=4,
		obeytabs=false,
		breaklines=true,
		autogobble=true,
		encoding="utf8",
		showspaces=false,
		xleftmargin=20pt,
		frame=single,
		framesep=5pt
	}
}

\let\origpythonfile\pythonfile
\renewcommand{\pythonfile}[1]{\pythonfileh{#1}{}}
\newcommand{\pythonfileh}[2]{\origpythonfile[#2]{#1}}

\newcommand*{\ditto}{\texttt{\char`\"}}