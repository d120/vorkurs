% !TeX root = vorstellung.tex
%\PassOptionsToClass{handout}{beamer}
\documentclass[accentcolor=3c,landscape,ngerman,presentation,t,usenames,dvipsnames,svgnames,table]{tudabeamer}

% Template-Modifikationen
\addtobeamertemplate{frametitle}{}{\vspace{-1em}} % mehr Platz vor dem Inhalt

% andere global gemeinsame definitionen
%Includes
\usepackage[ngerman]{babel} %Deutsche Silbentrennung
\usepackage[utf8]{inputenc} %Deutsche Umlaute
\usepackage{float}
\usepackage{graphicx}
\usepackage{minted}
\RequirePackage{csquotes}
\RequirePackage{fontawesome5}

\DeclareGraphicsExtensions{.pdf,.png,.jpg}

\makeatletter
\author{Vorkursteam der Fachschaft Informatik}
\let\Author\@author

% dark mode
\ExplSyntaxOn
\IfDarkModeT{
    \cs_if_exist:NT \setbeamercolor {
        \setbeamercolor*{smallrule}{bg=.}
        \setbeamercolor*{normal~text}{bg=\thepagecolor,fg=.}
        \setbeamercolor*{background~canvas}{parent=normal~text}
        \setbeamercolor*{section~in~toc}{parent=normal~text}
        \setbeamercolor*{subsection~in~toc}{parent=normal~text,fg=.}
        \setbeamercolor*{footline}{parent=normal~text}
        \setbeamercolor{block~title~alerted}{fg=white,bg=white!20!\thepagecolor}
        \setbeamercolor*{block~body}{bg=black!70!gray!98!blue}
        \setbeamercolor*{block~body~alerted}{bg=\thepagecolor}
    }
    \cs_if_exist:NT \setbeamertemplate {
        \setbeamertemplate{subsection~in~toc~shaded}[default][50]
    }
}
\ExplSyntaxOff

% macros
\renewcommand{\arraystretch}{1.2} % Höhe einer Tabellenspalte minimal erhöhen
\newcommand{\N}{{\mathbb N}}
\renewcommand{\code}{\inputminted[]{python}}

\IfDarkModeTF{
    \newmintedfile[pythonfile]{python}{
        fontsize=\small,
        style=native,
        linenos=true,
        numberblanklines=true,
        tabsize=4,
        obeytabs=false,
        breaklines=true,
        autogobble=true,
        encoding="utf8",
        showspaces=false,
        xleftmargin=20pt,
        frame=single,
        framesep=5pt,
    }
    \newmintinline{python}{
        style=native,
        encoding="utf8"
    }
    \newmintinline{kotlin}{
        style=native,
        encoding="utf8"
    }


    \definecolor{codegray}{HTML}{eaf1ff}
    \newminted[bashcode]{awk}{
        escapeinside=||,
        fontsize=\small,
        style=native,
        linenos=true,
        numberblanklines=true,
        tabsize=4,
        obeytabs=false,
        breaklines=true,
        autogobble=true,
        encoding="utf8",
        showspaces=false,
        xleftmargin=20pt,
        frame=single,
        framesep=5pt
    }
}{
    \newmintedfile[pythonfile]{python}{
        fontsize=\small,
        style=friendly,
        linenos=true,
        numberblanklines=true,
        tabsize=4,
        obeytabs=false,
        breaklines=true,
        autogobble=true,
        encoding="utf8",
        showspaces=false,
        xleftmargin=20pt,
        frame=single,
        framesep=5pt,
    }
    \newmintinline{python}{
        style=friendly,
        encoding="utf8"
    }
    \newmintinline{kotlin}{
        style=friendly,
        encoding="utf8"
    }

    \definecolor{codegray}{HTML}{eaf1ff}
    \newminted[bashcode]{awk}{
        escapeinside=||,
        fontsize=\small,
        style=friendly,
        linenos=true,
        numberblanklines=true,
        tabsize=4,
        obeytabs=false,
        breaklines=true,
        autogobble=true,
        encoding="utf8",
        showspaces=false,
        xleftmargin=20pt,
        frame=single,
        framesep=5pt
    }
}

\let\origpythonfile\pythonfile
\renewcommand{\pythonfile}[1]{\pythonfileh{#1}{}}
\newcommand{\pythonfileh}[2]{\origpythonfile[#2]{#1}}

\DeclareDocumentCommand{\kotlinfile}{O{} O{} m}{\inputCode[#1]{minted language=kotlin,#2}{#3}}

\newcommand*{\ditto}{\texttt{\char`\"}}

\newcommand{\shellprefix}{\textcolor{TUDa-3a}{\ttfamily\bfseries \$~}}
\DeclareTCBListing{commandshell}{ O{} O{} }{
    colback=\IfDarkModeTF{black}{black!80},
    colupper=white,
    colframe=TUDa-3a,
    listing only,
    % listing options={style=tcblatex,language=sh},
    listing engine=minted,
    minted style=dracula,
    minted options={
        % linenos=true,
        numbersep=3mm,
        texcl=true,
        autogobble,
        escapeinside=@@,
        breaklines,
        highlightcolor=yellow!50!black,
        #1
    },
    #2,
    % before upper={\textcolor{red}{\small\ttfamily\bfseries root \$> }},
    % every listing line={\textcolor{red}{\small\ttfamily\bfseries root \$> }}
}

%Includes
\usepackage{epstopdf}
\usepackage{wrapfig}
\usepackage{tipa}
\usepackage{tikz}
\usetikzlibrary{calc,shapes,arrows}
%tip: use http://l04.scarfboy.com/coding/phonetic-translation?from=ipa&fromtext=%CB%88pa%C9%AA%CE%B8n%CC%A9&to=tipa
%for converting ipa


\graphicspath{ {./media/} }

\def\shortyear#1{\expandafter\shortyearhelper#1}
\def\shortyearhelper#1#2#3#4{#3#4}

\newcount\NextYear
\NextYear = \year
\advance\NextYear by 1

\newcommand\NextYearShort{\shortyear{\the\NextYear}}

% notes
\usepackage{pgfpages}
\setbeamertemplate{note page}[plain]
%\setbeameroption{show notes on second screen}

% macro for change speaker sign
\newcommand{\changespeaker}{
	\begin{tikzpicture}[line width=.6mm, shorten >= 3pt, shorten <= 3pt]

	\coordinate (c1);
	\coordinate[right of=c1] (c2);

	\draw[rectangle, draw=red!80, fill=red!80, align=center, rounded corners] ($(c1.north west)+(0,-0.3)$) rectangle ($(c2.south east)+(0, 0.3)$) {};
	\draw[->,white] (c1)[bend left] to node[auto] {} (c2);
	\draw[->,white] (c2)[bend left] to node[auto] {} (c1);
	\end{tikzpicture}
}

%Listing-Style pyhon
\title[Programmiervorkurs]{Programmiervorkurs Wintersemester \the\year/\NextYearShort}
\subtitle{{\small der Fachschaft Informatik}}
\logo*{\includegraphics{../globalMedia/bildmarke_ohne_rand}}
\institute{Fachschaft Informatik}
\date{Wintersemester \the\year/\NextYearShort}


% macros
\newcommand{\livecoding}{\begin{frame}\frametitle{\insertsectionhead \\  {\small \insertsubsectionhead}}\centering \huge \vskip 2cm\textbf{\textcolor{red}{Live-Coding}}\end{frame}}

%\newcommand{\slidehead}{\frametitle{\insertsectionhead \\ {\small \insertsubsectionhead}}\vspace{3mm}}
\newcommand{\slidehead}{\frametitle{\insertsectionhead} \framesubtitle{\insertsubsectionhead}\vspace{3mm}}
\newcommand{\tocslide}{\begin{frame}[t]\frametitle{Inhaltsverzeichnis}\vspace{3mm}{\small\tableofcontents[subsectionstyle=shaded]}\end{frame}}


% colors
\definecolor{lightpetrol}{RGB}{0,223,194}

\logo*{\includegraphics{../../lecture/globalMedia/logo\IfDarkModeT{-dark}.pdf}}
% Packages
\usepackage{hyperref}
\usepackage{csquotes}
\usepackage{tikz}
\usetikzlibrary{fit,backgrounds}

\usepackage[font=normalsize, labelfont=sf, position=bottom]{caption}
\usepackage[labelfont=normalfont, position=bottom]{subcaption}
% Stylistic Changes
\captionsetup[figure]{justification=centering}
\setlength{\parindent}{0pt}

\def\gamefont{\bfseries\sffamily}


\begin{document}

%Deckblatt
\subtitle{Programmierchallenge}

\begin{frame}[fragile]
    \titlepage
    \begin{columns}[c]
        \begin{column}{4cm}
            \begin{center}
                {\huge Hangman}
            \end{center}
        \end{column}
        \begin{column}{4cm}
            \begin{figure}
                % \begin{noindent}
                \begin{BVerbatim}
  +---+
  |   |
  O   |
 /|\  |
 / \  |
      |
=========
V O R K _ R S
A C E M P Y X
                \end{BVerbatim}
                % \end{noindent}
                \\	\sffamily \tiny Beispiel Mögliche Hang\-man-Im\-ple\-men\-tierung\\- obere Zeile zeigt Wort \enquote{VORKURS} \\- untere Zeile zeigt falsch geratene Buchstaben \enquote{ACEMPYX}
            \end{figure}
        \end{column}
    \end{columns}
\end{frame}
\section{Programmierchallenge}
\subsection{Was ist das?}
\begin{frame}
    \slidehead
    \pause
    \begin{itemize}
        \item Ein größeres freiwilliges Projekt, dass man nach oder alternativ zu den Aufgaben machen kann (empfohlen ist nach der Übung!)
        \item Ihr bekommt die Möglichkeit eure Spiele abzugeben
        \item Die besten Ideen werden am Montag gezeigt
    \end{itemize}
\end{frame}

\subsection{Warum?}
\begin{frame}
    \slidehead
    \pause
    \begin{itemize}
        \item Euch eine mögliche Anwendung eures gelerntes Wissens zeigen
        \item Ein Spiel programmieren
        \item Euch herausfordern
    \end{itemize}
    \pause
    \vspace{\fill}
    \begin{center}
        \textbf{\huge Und deswegen gibt es heute eine Programmierchallenge!}
    \end{center}
    \vspace{\fill}
\end{frame}

\subsection{Wie soll das gehen?}
\begin{frame}
    \slidehead
    \begin{itemize}
        \item Keine Vorerfahrung
        \item Nur vier Tage Programmiervorkurs-Vorlesungen
        \item Und dann ein komplettes Spiel programmieren?
    \end{itemize}
    \pause
    \vspace{\fill}
    \begin{center}
        \textbf{\huge Wie sollen wir da schon bereit für ein größeres Projekt sein?}
    \end{center}
    \vspace{\fill}
\end{frame}

\subsection{Ihr seid bereit.}
\begin{frame}
    \slidehead
    \begin{itemize}
        \item Ihr kennt jetzt die wichtigsten Grundkonzepte in Python
        \item Die Challenge lässt sich mit dem Wissen des Vorkurses lösen
        \item So riesig ist das Projekt dann doch nicht $\Rightarrow$ gut schaffbar in zwei Tagen
        \item Ihr dürft in Gruppen abgeben (beliebige Größe)
        \item Wir helfen euch gerne
    \end{itemize}
    \pause
    \vspace{\fill}
    \begin{center}
        \textbf{\huge Ihr Schafft das!}
    \end{center}
    \vspace{\fill}
\end{frame}

\subsection{Ablauf}
\begin{frame}
    \slidehead
    \begin{itemize}
        \item Freischaltung der genauen Aufgabenstellung nach dieser Vorstellung (Moodle)
        \item Sucht euch Gruppen
        \item Abgabe bis 09.10.2022 23:59 Uhr in Moodle
        \item Die besten Abgaben werden nach der Ophase vorgestellt (genauer Termin wird noch bekannt gegeben)
    \end{itemize}
\end{frame}

\subsection{Was muss man beachten?}
\begin{frame}
    \slidehead
    \begin{itemize}
        \item KISS
        \item Eine Datei, keine Projekte
        \item Kommentiert euren Code
        \item Ihr dürft \textbf{keine} Imports verwenden
        \item euer Programm \textbf{muss} ohne Fehler gestartet werden können
    \end{itemize}
\end{frame}

\section{Ansonsten...}
\begin{frame}
    \slidehead
    \vspace{\fill}
    \begin{center}
        \huge ...noch viel Spaß !
    \end{center}
    \vspace{\fill}
\end{frame}
\end{document}
