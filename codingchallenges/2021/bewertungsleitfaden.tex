\documentclass[ngerman,accentcolor=3c,colorbacktitle,12pt,T1,points=true, RGB]{tudaexercise}

% Packages
\usepackage{hyperref}

%\usepackage{ClearSans}
\usepackage{csquotes}
\usepackage{tikz}
\usetikzlibrary{fit,backgrounds}
\usepackage{tcolorbox}
\tcbuselibrary{skins}

%Includes
\usepackage[ngerman]{babel} %Deutsche Silbentrennung
\usepackage[utf8]{inputenc} %Deutsche Umlaute
\usepackage{float}
\usepackage{graphicx}
\usepackage{minted}
\RequirePackage{csquotes}
\RequirePackage{fontawesome5}

\DeclareGraphicsExtensions{.pdf,.png,.jpg}

\makeatletter
\author{Vorkursteam der Fachschaft Informatik}
\let\Author\@author

% dark mode
\ExplSyntaxOn
\IfDarkModeT{
    \cs_if_exist:NT \setbeamercolor {
        \setbeamercolor*{smallrule}{bg=.}
        \setbeamercolor*{normal~text}{bg=\thepagecolor,fg=.}
        \setbeamercolor*{background~canvas}{parent=normal~text}
        \setbeamercolor*{section~in~toc}{parent=normal~text}
        \setbeamercolor*{subsection~in~toc}{parent=normal~text,fg=.}
        \setbeamercolor*{footline}{parent=normal~text}
        \setbeamercolor{block~title~alerted}{fg=white,bg=white!20!\thepagecolor}
        \setbeamercolor*{block~body}{bg=black!70!gray!98!blue}
        \setbeamercolor*{block~body~alerted}{bg=\thepagecolor}
    }
    \cs_if_exist:NT \setbeamertemplate {
        \setbeamertemplate{subsection~in~toc~shaded}[default][50]
    }
}
\ExplSyntaxOff

% macros
\renewcommand{\arraystretch}{1.2} % Höhe einer Tabellenspalte minimal erhöhen
\newcommand{\N}{{\mathbb N}}
\renewcommand{\code}{\inputminted[]{python}}

\IfDarkModeTF{
    \newmintedfile[pythonfile]{python}{
        fontsize=\small,
        style=native,
        linenos=true,
        numberblanklines=true,
        tabsize=4,
        obeytabs=false,
        breaklines=true,
        autogobble=true,
        encoding="utf8",
        showspaces=false,
        xleftmargin=20pt,
        frame=single,
        framesep=5pt,
    }
    \newmintinline{python}{
        style=native,
        encoding="utf8"
    }
    \newmintinline{kotlin}{
        style=native,
        encoding="utf8"
    }


    \definecolor{codegray}{HTML}{eaf1ff}
    \newminted[bashcode]{awk}{
        escapeinside=||,
        fontsize=\small,
        style=native,
        linenos=true,
        numberblanklines=true,
        tabsize=4,
        obeytabs=false,
        breaklines=true,
        autogobble=true,
        encoding="utf8",
        showspaces=false,
        xleftmargin=20pt,
        frame=single,
        framesep=5pt
    }
}{
    \newmintedfile[pythonfile]{python}{
        fontsize=\small,
        style=friendly,
        linenos=true,
        numberblanklines=true,
        tabsize=4,
        obeytabs=false,
        breaklines=true,
        autogobble=true,
        encoding="utf8",
        showspaces=false,
        xleftmargin=20pt,
        frame=single,
        framesep=5pt,
    }
    \newmintinline{python}{
        style=friendly,
        encoding="utf8"
    }
    \newmintinline{kotlin}{
        style=friendly,
        encoding="utf8"
    }

    \definecolor{codegray}{HTML}{eaf1ff}
    \newminted[bashcode]{awk}{
        escapeinside=||,
        fontsize=\small,
        style=friendly,
        linenos=true,
        numberblanklines=true,
        tabsize=4,
        obeytabs=false,
        breaklines=true,
        autogobble=true,
        encoding="utf8",
        showspaces=false,
        xleftmargin=20pt,
        frame=single,
        framesep=5pt
    }
}

\let\origpythonfile\pythonfile
\renewcommand{\pythonfile}[1]{\pythonfileh{#1}{}}
\newcommand{\pythonfileh}[2]{\origpythonfile[#2]{#1}}

\DeclareDocumentCommand{\kotlinfile}{O{} O{} m}{\inputCode[#1]{minted language=kotlin,#2}{#3}}

\newcommand*{\ditto}{\texttt{\char`\"}}

\newcommand{\shellprefix}{\textcolor{TUDa-3a}{\ttfamily\bfseries \$~}}
\DeclareTCBListing{commandshell}{ O{} O{} }{
    colback=\IfDarkModeTF{black}{black!80},
    colupper=white,
    colframe=TUDa-3a,
    listing only,
    % listing options={style=tcblatex,language=sh},
    listing engine=minted,
    minted style=dracula,
    minted options={
        % linenos=true,
        numbersep=3mm,
        texcl=true,
        autogobble,
        escapeinside=@@,
        breaklines,
        highlightcolor=yellow!50!black,
        #1
    },
    #2,
    % before upper={\textcolor{red}{\small\ttfamily\bfseries root \$> }},
    % every listing line={\textcolor{red}{\small\ttfamily\bfseries root \$> }}
}


\usepackage[font=normalsize, labelfont=sf, position=bottom]{caption}
\usepackage[labelfont=normalfont, position=bottom]{subcaption}
% Stylistic Changes
\captionsetup[figure]{justification=centering}
\setlength{\parindent}{0pt}
\defcaptionname{ngerman, german}{\taskname}{Kriterie}
\renewcommand*{\subtaskformat}{\thesubtask\enskip} % This looks way cleaner
\renewcommand*{\creditformat}[1]{\hfill#1}
\renewcommand*{\creditformatsum}[1]{\creditformat{Gesamt: #1}}

\def\gamefont{\bfseries\sffamily}

\definecolor{grid color}{RGB}{187, 173, 160}
\definecolor{tile 0}{HTML}{CCC0B3}
\definecolor{tile 2}{HTML}{EEE4DA}
\definecolor{tile 4}{HTML}{EEE1C9}
\definecolor{tile 8}{HTML}{F3B27A}
\definecolor{tile 16}{HTML}{F69664}
\definecolor{tile 32}{HTML}{F77C5F}
\definecolor{tile 64}{HTML}{F75F3B}
\definecolor{tile 128}{HTML}{EDD073}
\definecolor{tile 256}{HTML}{EDCC62}
\definecolor{tile 512}{HTML}{EDC950}
\definecolor{tile 1024}{HTML}{EDC53F}
\definecolor{tile 2048}{HTML}{EDC22E}
\definecolor{tile 4096}{HTML}{3C3A33}
%
\definecolor{small color}{HTML}{776E65}
\definecolor{big color}{HTML}{F9F6F2}
%
\tikzset{
    case 2048 base/.style={
        minimum size=9mm,rounded corners=.3mm,text=#1,inner sep=0,line width=0,
    },
    %
    case 2048 Large/.style={font=\Large\gamefont,case 2048 base=#1},
    case 2048 large/.style={font=\large\gamefont,case 2048 base=#1},
    case 2048 normal/.style={font=\normalsize\gamefont,case 2048 base=#1},
    %
    case 2048 0/.style={case 2048 Large=black,fill=tile 0,node contents={}},
    case 2048 2/.style={case 2048 Large=small color,fill=tile 2,node contents={2}},
    case 2048 4/.style={case 2048 Large=small color,fill=tile 4,node contents={4}},
    case 2048 8/.style={case 2048 Large=big color,fill=tile 8,node contents={8}},
    case 2048 16/.style={case 2048 Large=big color,fill=tile 16,node contents={16}},
    case 2048 32/.style={case 2048 Large=big color,fill=tile 32,node contents={32}},
    case 2048 64/.style={case 2048 Large=big color,fill=tile 64,node contents={64}},
    case 2048 128/.style={case 2048 large=big color,fill=tile 128,node contents={128}},
    case 2048 256/.style={case 2048 large=big color,fill=tile 256,node contents={256}},
    case 2048 512/.style={case 2048 large=big color,fill=tile 512,node contents={512}},
    case 2048 1024/.style={case 2048 normal=big color,fill=tile 1024,node contents={1024}},
    case 2048 2048/.style={case 2048 normal=big color,fill=tile 2048,node contents={2048}},
    case 2048 4096/.style={case 2048 normal=big color,fill=tile 4096,node contents={4096}},
}

% Document

\begin{document}


\title{Programmierchallenge Wintersemester 2021/22}
\subtitle{der Fachschaft Informatik}
\author{Autor: Ruben Deisenroth}
\term{\textbf{\textsf{Bewertungsleitfaden}}}
\maketitle
%\selectcolormodel{RGB}
%\definecolor{testtt}{RGB}{0,136,119}
%\definecolor{TUDa-3c}{cmyk/RGB/HTML}{1,.2,.6,0/0,136,119/008877}
\begin{tcolorbox}[
        colback=\IfDarkModeTF{accentcolor!20!\thepagecolor}{accentcolor!10!\thepagecolor}, %Hintergrundfarbe
        %colframe=accentcolor, % Randfarbe
        coltext=.,
        colframe=gray, % Randfarbe
        frame hidden,
        title style=accentcolor,
        arc=3pt,
        boxrule=0pt,
        left=5pt, % Links Platz lassen
        enhanced, % Erlaubt uns, den ramen zu zeichnen
        fonttitle=\sffamily, % Titelschriftart auf 
        overlay={ % Für Grauen Bereich links
            \begin{tcbclipinterior}%
                \fill[TUDa-3c] (frame.south west) rectangle ([xshift=4pt]frame.north west); % Zeilennummernbereich färben
            \end{tcbclipinterior}%
        }]
    \textbf{\textsf{Disclaimer}}: Dieser Leitfaden ist nur dafür da um Ideen zu geben, was geprüft werden kann/was man ins Feedback schreiben kann.

    Die angegebenen Punktzahlen sind \textbf{Vorschläge}.
\end{tcolorbox}
Bitte seit nicht zu streng gerade bei den Abgaben ohne Vorerfahrung und gebt \textbf{konstruktives} Feedback.
\begin{task}[points=auto]{Spiellogik}
    \begin{subtask}[points=10,title={Startzustand}]
        \begin{itemize}
            \item Das Spielfeld hat die Richtige Größe (4x4)
            \item es sind genau zwei Felder gefüllt.
        \end{itemize}
    \end{subtask}
    \begin{subtask}[points=20,title={Verschieben}]
        \begin{itemize}
            \item Alle Kacheln werden weitestmöglich in die gewählte Richtung verschoben
            \item Die Reihenfolge der Kacheln ist richtig (kein Überspringen)
        \end{itemize}
        Testet aber bitte alle Schieberichtungen
    \end{subtask}
    \begin{subtask}[points=25,title={Verschmelzen}]
        Die wichtigsten Fälle werden durch die folgende Verschiebung dargestellt:
        \begin{center}
            \begin{tikzpicture}[baseline={([yshift=-.8ex]current bounding box.center)}]
                \def\tiles{
                    {2,0,4,0},
                    {16,0,0,16},
                    {4,2,2,4},
                    {8,8,8,8},
                }

                \foreach \line [count=\y] in \tiles {
                    \foreach \pix [count=\x] in \line {
                        \path (\x,-\y) node[name=c2048-\x-\y,case 2048 \pix];
                    }
                }

                \begin{scope}[on background layer]
                    \node[fill=grid color,fit=(c2048-1-1)(c2048-4-4),
                        inner sep=1mm,rounded corners=.3mm]{};
                \end{scope}
            \end{tikzpicture}
            $\stackrel{schieben(r)}{\Longrightarrow}$
            \begin{tikzpicture}[baseline={([yshift=-.8ex]current bounding box.center)}]
                \def\tiles{
                    {0,0,2,4},
                    {0,0,0,32},
                    {0,4,4,4},
                    {0,0,16,16},
                }

                \foreach \line [count=\y] in \tiles {
                    \foreach \pix [count=\x] in \line {
                        \path (\x,-\y) node[name=c2048-\x-\y,case 2048 \pix];
                    }
                }

                \begin{scope}[on background layer]
                    \node[fill=grid color,fit=(c2048-1-1)(c2048-4-4),
                        inner sep=1mm,rounded corners=.3mm]{};
                \end{scope}
            \end{tikzpicture}
        \end{center}
        Testet aber bitte alle Richtungen, da man je nach Schieberichtung in unterschiedlicher Reihenfolge durchitterieren muss
    \end{subtask}
    \begin{subtask}[points=15,title={Zufall}]
        Wenn \pythoninline{import random} verwendet wurde gibt es hier \textbf{0 Punkte}
        \begin{itemize}
            \item Verteilung der Zufallszahlen ist linear
            \item alle freien Felder haben die gleiche Wahrscheinlichkeit
            \item  Die Wahrscheinlichkeit, dass einem freien Feld eine 4 statt einer 2 zugewiesen wird,
                beträgt $10\%$
        \end{itemize}
    \end{subtask}
    \begin{subtask}[points=20,title={Spieleablauf (nach Startzustand)}]
        \begin{itemize}
            \item In jeder weiteren Runde wird die Verschieberichtung abgefragt
            \item In jeder weiteren Runde wird ein weiteres zufälliges, freies Feld mit 2 oder 4 gefüllt
            \item Ist kein Verschieben mehr möglich, wird ein \enquote{loose screen} angezeigt
            \item Wird 2048 erreicht, wird ein \enquote{win screen} angezeigt
        \end{itemize}
    \end{subtask}
    \begin{subtask}[points=10,title={Anzeige}]
        Hier können auch gerne Bonuspunkte für besonders hübsche Anzeigen (z.B. mit unicode) vergeben werden ;$)$
        \begin{itemize}
            \item Alle Felder des Spielfeldes werden jede Runde angezeigt
            \item Die Anzeige sieht auch bei größeren Zahlen gut aus
            \item Die möglichen Eingaben werden angezeigt
            \item Es gibt einen einfachen Weg, das Spiel jederzeit ohne Fehlermeldung zu beenden
            \item (Besonders schön wäre: Punkte+Rundencounter sowie highscore)
        \end{itemize}
    \end{subtask}
\end{task}
\begin{task}[points=auto]{Sonstiges}
    Andere Sachen, die ihr gerne im Feedback erwähnen könnt:
    \begin{itemize}
        \item Kreativität
        \item Code Stil (und Kommentarstil)
        \item loben für hat wahrscheinlich vorerfahrung/hat es mit Vorkurswissen gelöst
        \item Abgabezeitpunkt
    \end{itemize}
\end{task}
\end{document}
