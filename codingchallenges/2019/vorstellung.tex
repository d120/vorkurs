\documentclass[accentcolor=tud3c,colorbacktitle,inverttitle,landscape,german,presentation,t]{tudbeamer}

\usepackage[ngerman]{babel} %Deutsche Silbentrennung
\usepackage[utf8]{inputenc} %Deutsche Umlaute
\usepackage{float}
\usepackage{listings}
\usepackage{graphicx}
\usepackage{epstopdf}
\usepackage{wrapfig}

\lstset{language=python,
	basicstyle=\small,
	keywordstyle=\color{blue!80!black!100},
	identifierstyle=,
	commentstyle=\color{green!50!black!100},
	stringstyle=\ttfamily,
	breaklines=true,
	stepnumber=1,
	%numbers=left,
	numberstyle=\small,
	frame=single,
	backgroundcolor=\color{blue!3},
	literate= 
	{Ö}{{\"O}}1
	{Ä}{{\"A}}1
	{Ü}{{\"U}}1
	{ß}{{\ss}}1
	{ö}{{\"o}}1
	{ü}{{\"u}}1
	{ä}{{\"a}}1
}

\title[Programmiervorkurs]{Programmiervorkurs Wintersemester 2019/20}
\subtitle{{\small der Fachschaft Informatik}}
\logo[2]{\includegraphics[scale=5]{../../lecture/globalMedia/bildmarke_ohne_rand}}
\author{Vorkursteam der Fachschaft Informatik}
\institute{TU Darmstadt}
\date{Wintersemester 2019}


	
\begin{document}
\begin{titleframe}
	\begin{center}
		\vspace{2cm}
		{\huge Programmierchallenge \\ "`TicTacToe"'}
	\end{center}
\end{titleframe}


\begin{frame}
	\frametitle{Was wollen wir machen}
		\begin{itemize}
			\item Keine Vorerfahrung
			\item Nur drei Tage Programmiervorkurs
			\item Und dann ein komplettes Spiel programmieren?
		\end{itemize}
\end{frame}


\begin{frame}
	\frametitle{Was wollen wir machen}
		\begin{itemize}
			\item Euch eine mögliche Anwendung eures gelerntes Wissens zeigen
			\item Ein Spiel programmieren
			\item Euch herausfordern
		\end{itemize}
		\vspace{6mm}
	\begin{center}
	 \textbf{\huge Und deswegen gibt es heute eine Programmierchallenge!}	
	\end{center}
	
\end{frame}



\begin{frame}
	\frametitle{Was ist das?}
	\begin{itemize}
		\item Ein größeres Projekt, dass man nach oder alternativ zu den Aufgaben machen kann (empfohlen ist nach der Übung!)
		\item Ihr bekommt die Möglichkeit eure Spiele abzugeben
		\item Die besten Ideen werden am Freitag gezeigt
	\end{itemize}
\end{frame}

\begin{frame}
	\frametitle{Wie wird es ablaufen?}
	\begin{itemize}
		\item Arbeitsblatt mit der genauen Aufgabenstellung (Moodle)
		\item Gruppen von drei oder vier Leuten
		\item Abgabe mittels E-Mail oder in Moodle bis morgen 14:00
		\item Die besten Abgaben werden Freitag während der Vorlesung gezeigt
	\end{itemize}
\end{frame}

\begin{frame}
	\frametitle{Was sind die Regeln?}
	\begin{itemize}
		\item KISS
		\item Eine Datei, keine Projekte
		\item Kommentiert euren Code 
		\item Muss kompilieren
	\end{itemize}
\end{frame}

\begin{frame}
	\frametitle{Ansonsten...}
	\vspace{15mm}
	\begin{center}
		\huge ...noch viel Spaß !
	\end{center}
\end{frame}
\end{document} 