% !TeX document-id = {505cbeed-ac21-4fdd-a7f4-eefb28b91c1e}
% !TeX TXS-program:compile = txs:///pdflatex/[--shell-escape]
% !TeX root = PythonInstallieren.tex
% \documentclass[accentcolor=3c,colorbacktitle,12pt]{tudaexercise}
\RequirePackage{import}
\subimport{../../exercises}{preamble.tex}
\usepackage{hyperref}
\title{Installation von Python}
\subsubtitle{vorkurs@d120.de}

\begin{document}
\maketitle

Bitte schaue vorher, dass dein System auf dem aktuellen Stand ist.

In dieser Installationsanleitung installieren wir:
\begin{itemize}
    \item \href{https://www.python.org/}{Python Version 3.X.X}
    \item \href{https://ipython.org/}{ipython (\underline{I}nteractive \underline{P}ython)}
\end{itemize}

\section*{Windows}
\begin{hinweis}
    Die \verb+$+ Zeichen in den Befehlen sind Platzhalter für den Terminal-Prompt. Sie dürfen \textbf{nicht} mitkopiert werden.
\end{hinweis}
\subsection*{Automatische Installation durch scoop}

\subsubsection*{Terminal öffnen (Powershell)}
Siehe Terminal Guide.

Darauf achten, dass \textbf{nicht} als Admin

\subsubsection*{Scoop installieren}
Anhand der Anleitung auf \url{https://scoop.sh/} folgendes ins Terminal eingeben.
%\begin{noindent}
\begin{commandshell}
@\shellprefix{}@Set-ExecutionPolicy -ExecutionPolicy RemoteSigned -Scope CurrentUser
\end{commandshell}
Das muss mit \verb#A + Enter# bestätigt werden.

Dann das hier eingeben und Enter drücken:
\begin{commandshell}
@\shellprefix{}@irm get.scoop.sh | iex
\end{commandshell}

\subsubsection*{Python installieren}
\begin{commandshell}
    @\shellprefix{}@scoop install python
\end{commandshell}
\section*{ipython installieren}
\begin{commandshell}
@\shellprefix{}@python -m pip install ipython
\end{commandshell}
Nun kannst du die Installation überprüfen (siehe Abschnitt \nameref{sec:check}).

\newpage
\section*{Mac OS}

\subsection*{Homebrew Installieren}
\begin{hinweis}
    Die \verb+$+ Zeichen in den Befehlen sind Platzhalter für den Terminal-Prompt. Sie dürfen \textbf{nicht} mitkopiert werden.
\end{hinweis}
\begin{itemize}
    \item Auf der seite \url{https://brew.sh} den Befehl kopieren
    \item Im terminal diesen Befehl einfügen
    \item Installations-Anweisungen folgen (passwort eingeben), die Installation dauert ca. 2 Minuten
    \item Sobald der homebrew install Befehl fertig ist, diese Befehle ausführen
        %\begin{noindent}
    \begin{commandshell}
@\shellprefix{}@(echo; echo 'eval "$(/opt/homebrew/bin/brew shellenv)"') @\texttt{>>}@ ~/.profile
    \end{commandshell}
    \begin{commandshell}
@\shellprefix{}@eval "$(/opt/homebrew/bin/brew shellenv)"
    \end{commandshell}
    %\end{noindent}
\end{itemize}

\subsection*{Automatische Installation durch Homebrew}
\begin{itemize}
    \item Gebe in der Kommandozeile den folgenden Befehl ein und drücke Enter\footnote{Befehle generiert mit \url{https://formulae.brew.sh/formula/ipython}}:
        %\begin{noindent}
\begin{commandshell}
@\shellprefix{}@brew install ipython
\end{commandshell}
    %\end{noindent}
        Python wird automatisch mit installiert
    \item Warte, bis die Installation abgeschlossen ist
    \item Überprüfe die Installation (siehe Abschnitt \nameref{sec:check})
\end{itemize}

Anmerkung: bei einer Installation über Brew bei Mac OS der python-Befehl \texttt{python3} und nicht \texttt{python}. Der \texttt{ipython}-Befehl bleibt gleich.

\newpage
\section*{Linux}
Auf dem meisten Distros heißt das zu installierende Paket einfach {\ttfamily python}.
\begin{hinweis}
    Die \verb+$+ Zeichen in den Befehlen sind Platzhalter für den Terminal-Prompt. Sie dürfen \textbf{nicht} mitkopiert werden.
\end{hinweis}
\begin{commandshell}[][minted language=text]
    @\shellprefix{}@pacman -S python @\PYG{c+c1}{# Arch Linux}@
\end{commandshell}

\begin{commandshell}[][minted language=text]
    @\shellprefix{}@apt install python @\PYG{c+c1}{# Debian, Ubuntu, Mint, ...}@
\end{commandshell}

\subsection*{ipython}
Jeweils mit pip installieren:
\begin{commandshell}[][minted language=text]
    @\shellprefix{}@python -m pip install ipython
\end{commandshell}

\section*{Überprüfung der Installation}\label{sec:check}
Um zu testen ob alles geklappt hat, öffne eine Kommandozeile und gebe den folgenden Befehl ein:
\begin{commandshell}
    @\shellprefix{}@python --version
\end{commandshell}

Wenn alles geklappt hat, sollte die Ausgabe in etwa so aussehen (die Versionsnummern können abweichen):
\begin{commandshell}[][]
    @\shellprefix{}@ python --version
    Python 3.12.6
\end{commandshell}

Ob \texttt{ipython} funktioniert, kannst du mit dem folgenden Befehl testen:
\begin{commandshell}
    @\shellprefix{}@ipython
\end{commandshell}
Wenn alles geklappt hat, sollte die Ausgabe in etwa so aussehen (die Versionsnummern können abweichen):
\begin{commandshell}[][minted language=text]
    @\shellprefix@ipython
    Python 3.12.6 (main, Sep  8 2024, 13:18:56) [GCC 14.2.1 20240805]
    Type 'copyright', 'credits' or 'license' for more information
    IPython 8.27.0 -- An enhanced Interactive Python. Type '?' for help.

    In [1]: @\shellcursor@
\end{commandshell}

\end{document}
