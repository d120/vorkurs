% !TeX document-id = {505cbeed-ac21-4fdd-a7f4-eefb28b91c1e}
% !TeX TXS-program:compile = txs:///pdflatex/[--shell-escape]
% \documentclass[accentcolor=3c,colorbacktitle,12pt]{tudaexercise}
\RequirePackage{import}
\subimport{../../exercises}{preamble.tex}
\usepackage{hyperref}
\title{Installation von Kotlin}
\subsubtitle{vorkurs@d120.de}

\begin{document}
\maketitle

Bitte schaue vorher, dass dein System auf dem aktuellen Stand ist.

In dieser Installationsanleitung installieren wir:
\begin{itemize}
    \item Kotlin version 1.9
    \item Java JDK 11+ (wird für Kotlin benötigt)
    \item \href{https://github.com/Kotlin/kotlin-interactive-shell}{ki (\underline{K}otlin \underline{I}nteractive-Shell)}
\end{itemize}
\section*{Windows}
\subsection*{Automatische Installation durch scoop}
\subsubsection*{Terminal öffnen (Powershell)}
Siehe Terminal Guide
\subsubsection*{Scoop installieren}
Anhand der Anleitung auf \url{https://scoop.sh/} folgendes ins Terminal eingeben und nach jeder Zeile Enter drücken:
%\begin{noindent}
\begin{commandshell}
@\shellprefix{}@Set-ExecutionPolicy -ExecutionPolicy RemoteSigned -Scope CurrentUser
@\shellprefix{}@irm get.scoop.sh | iex
@\shellprefix{}@scoop install git
\end{commandshell}
%\end{noindent}
\subsubsection*{Java installieren}
Als Erstes kannst du mit folgendem Befehl überprüfen, ob du Java installiert hast:
\begin{commandshell}
    @\shellprefix{}@javac -version
\end{commandshell}
Wenn mindestens Version 11 angezeigt wird, hast du eine ausreichend aktuelle Version. Wir empfehlen aber in allen Fällen die neuste Version mit den folgenden Befehlen zu installieren:
%\begin{noindent}
\begin{commandshell}
@\shellprefix{}@scoop bucket add java
@\shellprefix{}@scoop install java/openjdk
\end{commandshell}
%\end{noindent}
\subsubsection*{Kotlin installieren}
\begin{commandshell}
    @\shellprefix{}@scoop install main/kotlin
\end{commandshell}
\subsubsection*{Kotlin Interactive Shell installieren}
%\begin{noindent}
\begin{commandshell}
@\shellprefix{}@scoop bucket add extras
@\shellprefix{}@scoop install extras/kotlin-interactive-shell
\end{commandshell}
%\end{noindent}
Nun kannst du die Installation überprüfen (siehe Abschnitt \nameref{sec:check}).
% \subsection*{Kommandozeile öffnen (Powershell)}
% Siehe Terminal guide
% \subsection*{Automatische Installation durch Winget}
% \begin{hinweis}
%     Winget gibt es erst ab Windows 10 2004. Wenn du eine ältere Version hast, musst du die manuelle Installation verwenden.
% \end{hinweis}
% \begin{itemize}
%     \item Als erstes musst du die JDK installieren. Ob du bereits eine JDK installiert hast, kannst du mit dem folgenden Befehl feststellen:
%         \begin{commandshell}
%             @\shellprefix{}@java -version
%         \end{commandshell}
%         Wenn die Rückgabe sowas wie \verb+java version "17.0.1"+ ist, hast du bereits eine JDK installiert. In diesem Fall kannst du die JDK-Installation überspringen. Für die aktuelle Kotlin Version wird mindestens JDK-Version 17 benötigt.
%     \item Gebe in der Powershell den folgenden Befehl ein und drücke Enter\footnote{Befehl generiert mit \url{https://winstall.app/apps/JetBrains.Kotlin.Compiler}}:
%         \begin{commandshell}
%             @\shellprefix{}@winget install --id=EclipseAdoptium.Temurin.17.JDK -e
%         \end{commandshell}
%     \item Wenn verlangt wird, dass du die Nutzungsbedingungen akzeptierst, drücke \texttt{y} und Enter
%     \item Warte, bis die Installation abgeschlossen ist
%     \item Als nächstes installieren wir den Kotlin Compiler. Gebe dazu den folgenden Befehl ein und drücke Enter\footnote{Befehl generiert mit \url{https://winstall.app/apps/JetBrains.Kotlin.Compiler}}:
%         \begin{commandshell}
%             @\shellprefix{}@winget install --id=JetBrains.Kotlin.Compiler -e
%         \end{commandshell}
%     \item Damit die neuen Befehle erkannnt werden, musst du die Kommandozeile neu starten, also einmal schließen und wieder öffnen.
%     \item Überprüfe die Installation (siehe Abschnitt \nameref{sec:check})
% \end{itemize}
% \subsection*{Automatische Installation durch Chocolatey}
% \subsubsection*{Chocolatey installieren}
% \begin{itemize}
%     \item Falls chocolatey schon installiert ist, kannst du diesen Schritt überspringen
%     \item Gebe in der Powershell den folgenden Befehl ein und drücke Enter\footnote{Befehl generiert mit \url{https://chocolatey.org/install}} (da der Befehl sich nicht gut kopieren lässt am Besten aus dem Link der Fußnote kopieren):
%         \begin{commandshell}
%             @\shellprefix{}@Set-ExecutionPolicy Bypass -Scope Process -Force;
%             [System.Net.ServicePointManager]::SecurityProtocol =
%             [System.Net.ServicePointManager]::SecurityProtocol -bor 3072;
%             iex ((New-Object System.Net.WebClient)
%             .DownloadString('https://community.chocolatey.org/install.ps1'))
%         \end{commandshell}
%     \item Warte, bis die Installation abgeschlossen ist
% \end{itemize}
% \begin{hinweis}
%     Ab Windows 10 kann chocolatey auch über winget per \texttt{winget install chocolatey} installiert werden.
% \end{hinweis}
% \clearpage
% \subsubsection*{JVM und Kotlin Compiler durch Chocolatey installieren}
% \begin{itemize}
%     \item Gebe in der Powershell den folgenden Befehl ein und drücke Enter\footnote{siehe auch \url{https://community.chocolatey.org/packages/kotlinc} und \url{https://community.chocolatey.org/packages/Temurin17/}}:
%         \begin{commandshell}
%             @\shellprefix{}@choco install temurin17 kotlinc -y
%         \end{commandshell}
%     \item Warte, bis die Installation abgeschlossen ist
%     \item Damit die neuen Befehle erkannnt werden, musst du die Kommandozeile neu starten, also einmal schließen und wieder öffnen.
%     \item Überprüfe die Installation (siehe Abschnitt \nameref{sec:check})
% \end{itemize}
% \subsubsection{ki}
% Aktuell gibt es leider keinen \enquote{schönen} Weg, \texttt{ki} über einen Paketmanager zu installieren. Stattdessen kannst du \texttt{ki} lokal installieren. Dazu musst du folgende Schritte ausführen:
% \begin{itemize}
%     \item Lade KI von hier herunter \url{https://github.com/Kotlin/kotlin-interactive-shell/releases/tag/v0.5.2}
%     \item Entpacke die Datei in einen Ordner deiner Wahl.
%     \item Gehe nun in den bin-Ordner der entpackten Datei und öffne dort ein Terminal. (Siehe Terminal-Anleitung)
%     \item Es kann sein, dass du statt dem Befehl \texttt{ki} den Befehl \mintinline{text}{.\ki.bat} ausführen musst.
%     \item Überprüfe die Installation (siehe Abschnitt \nameref{sec:check})
% \end{itemize}
% \begin{hinweis}
%     Diese Installation funktioniert nur, solange sich deine Kommandozeile im bin-Ordner befindet. Wenn du die Kommandozeile schließt, musst du beim nächsten Öffnen erneut in den bin-Ordner wechseln.
% \end{hinweis}
% \subsubsection*{ki durch Chocolatey installieren}
% \begin{itemize}
%     \item Gebe in der Powershell den folgenden Befehl ein und drücke Enter\footnote{siehe auch \url{https://community.chocolatey.org/packages/ki}}:
%         \begin{commandshell}
%             @\shellprefix{}@choco install ki -y
%         \end{commandshell}
%     \item Selbes Spiel wie oben
% \end{itemize}


% \subsection*{Manuell (nicht empfohlen!!!, nur wenn Winget und Chocolatey nicht funktionieren)}
% \subsubsection*{Download}
% \begin{itemize}
%     \item JDK 17 oder höher wird vorausgesetzt
%     \item Gehe auf die Seite \url{https://github.com/JetBrains/kotlin/releases} und lade die neuste Version herunter (die neuste Version ist die oberste in der Liste). Die Datei sollte den Namen \texttt{kotlin-compiler-<version>.zip} haben.
%     \item Entpacke die Datei in einen Ordner deiner Wahl.
%     \item Gehe nun in den bin-Ordner der entpackten Datei und öffne dort ein Terminal. (Shift + Rechtsklick $\rightarrow$ \enquote{Eingabeaufforderung hier öffnen} oder \enquote{PowerShell-Fenster hier öffnen}).
%     \item Überprüfe die Installation (siehe Abschnitt \nameref{sec:check})
% \end{itemize}
% \begin{hinweis}
%     Diese Installation funktioniert nur, solange sich deine Kommandozeile im bin-Ordner befindet. Wenn du die Kommandozeile schließt, musst du sie erneut öffnen und in den bin-Ordner wechseln.
% \end{hinweis}
\section*{Mac OS}
% Kaufe jetzt das Kotlin Starter Kit für nur 99,99€ im App Store und erhalte eine kostenlose Kotlin-Tasse im Wert von 9,99€ dazu! Nur solange der Vorrat reicht! (Nein, Spaß, aber wir haben leider keine Macs zum Testen.)

\subsection*{Homebrew Installieren}
\begin{itemize}
    \item Auf der seite \url{https://brew.sh} den Befehl kopieren
    \item Im terminal diesen Befehl einfügen
    \item Installations-Anweisungen folgen (passwort eingeben), die Installation dauert ca. 2 Minuten
    \item Sobald der homebrew install Befehl fertig ist, diese Befehle ausführen
    %\begin{noindent}
    \begin{commandshell}
@\shellprefix{}@(echo; echo 'eval "$(/usr/local/bin/brew shellenv)"') @\texttt{>>}@ ~/.profile
@\shellprefix{}@eval "$(/usr/local/bin/brew shellenv)"
    \end{commandshell}
    %\end{noindent}
\end{itemize}

\subsection*{Automatische Installation durch Homebrew}
\begin{itemize}
    \item Gebe in der Kommandozeile den folgenden Befehl ein und drücke Enter\footnote{Befehle generiert mit \url{https://formulae.brew.sh/formula/kotlin} und \url{https://formulae.brew.sh/formula/ki}}:
        %\begin{noindent}
\begin{commandshell}
@\shellprefix{}@brew install kotlin
@\shellprefix{}@brew install ki
\end{commandshell}
    %\end{noindent}
    \item Warte, bis die Installation abgeschlossen ist
    \item Überprüfe die Installation (siehe Abschnitt \nameref{sec:check})
\end{itemize}

\section*{Linux}
\subsection*{JVM und Kotlin}
Auf dem meisten Distros heißt das zu installierende Paket einfach {\ttfamily kotlin} und hat automatisch die korrekte java-Version als Abhängigkeit.
\begin{commandshell}[][minted language=text]
    @\shellprefix{}@pacman -S kotlin @\PYG{c+c1}{# Arch Linux}@
\end{commandshell}

\begin{commandshell}[][minted language=text]
    @\shellprefix{}@apt install kotlin @\PYG{c+c1}{# Debian, Ubuntu, Mint, ...}@
\end{commandshell}

\subsection*{ki}
Für Arch-Linux gibt es ein AUR-Paket: \url{https://aur.archlinux.org/packages/ki/}
Für alle anderen haben wir \href{https://raw.githubusercontent.com/d120/vorkurs/91e562678d871e52162ac98ee7175842ba628a96/misc/scripts/install-ki-locally.sh}{ein script} bereitgestellt:
\begin{commandshell}
    wget -O - https://raw.githubusercontent.com\
    /d120/vorkurs/91e562678d871e52162ac98ee7175842ba628a96\
    /misc/scripts/install-ki-locally.sh | bash
\end{commandshell}

\section*{Überprüfung der Installation}\label{sec:check}
Um zu testen ob alles geklappt hat, öffne eine Kommandozeile und gebe den folgenden Befehl ein:
\begin{commandshell}
    @\shellprefix{}@kotlinc -version
\end{commandshell}

Wenn alles geklappt hat, sollte die Ausgabe in etwa so aussehen (die Versionsnummern können abweichen):
\begin{commandshell}[][]
    @\shellprefix{}@kotlinc -version
    info: kotlinc-jvm 1.9.10 (JRE 17.0.8.1+1)
\end{commandshell}

Ob \texttt{ki} funktioniert, kannst du mit dem folgenden Befehl testen:
\begin{commandshell}
    @\shellprefix{}@ki
\end{commandshell}
Wenn alles geklappt hat, sollte die Ausgabe in etwa so aussehen (die Versionsnummern können abweichen):
\begin{commandshell}[][minted language=text]
    @\shellprefix@ki
    ki-shell 0.5.2/1.7.0
    type :h for help
    [0] @\shellcursor@
\end{commandshell}

\end{document}
