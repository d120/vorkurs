% !TeX document-id = {505cbeed-ac21-4fdd-a7f4-eefb28b91c1e}
% !TeX TXS-program:compile = txs:///pdflatex/[--shell-escape]
\documentclass[accentcolor=3c,colorbacktitle,12pt]{tudaexercise}
%Includes
\usepackage[ngerman]{babel} %Deutsche Silbentrennung
\usepackage[utf8]{inputenc} %Deutsche Umlaute
\usepackage{float}
\usepackage{graphicx}
\usepackage{minted}
\RequirePackage{csquotes}
\RequirePackage{fontawesome5}

\DeclareGraphicsExtensions{.pdf,.png,.jpg}

\makeatletter
\author{Vorkursteam der Fachschaft Informatik}
\let\Author\@author

% dark mode
\ExplSyntaxOn
\IfDarkModeT{
    \cs_if_exist:NT \setbeamercolor {
        \setbeamercolor*{smallrule}{bg=.}
        \setbeamercolor*{normal~text}{bg=\thepagecolor,fg=.}
        \setbeamercolor*{background~canvas}{parent=normal~text}
        \setbeamercolor*{section~in~toc}{parent=normal~text}
        \setbeamercolor*{subsection~in~toc}{parent=normal~text,fg=.}
        \setbeamercolor*{footline}{parent=normal~text}
        \setbeamercolor{block~title~alerted}{fg=white,bg=white!20!\thepagecolor}
        \setbeamercolor*{block~body}{bg=black!70!gray!98!blue}
        \setbeamercolor*{block~body~alerted}{bg=\thepagecolor}
    }
    \cs_if_exist:NT \setbeamertemplate {
        \setbeamertemplate{subsection~in~toc~shaded}[default][50]
    }
}
\ExplSyntaxOff

% macros
\renewcommand{\arraystretch}{1.2} % Höhe einer Tabellenspalte minimal erhöhen
\newcommand{\N}{{\mathbb N}}
\renewcommand{\code}{\inputminted[]{python}}

\IfDarkModeTF{
    \newmintedfile[pythonfile]{python}{
        fontsize=\small,
        style=native,
        linenos=true,
        numberblanklines=true,
        tabsize=4,
        obeytabs=false,
        breaklines=true,
        autogobble=true,
        encoding="utf8",
        showspaces=false,
        xleftmargin=20pt,
        frame=single,
        framesep=5pt,
    }
    \newmintinline{python}{
        style=native,
        encoding="utf8"
    }
    \newmintinline{kotlin}{
        style=native,
        encoding="utf8"
    }


    \definecolor{codegray}{HTML}{eaf1ff}
    \newminted[bashcode]{awk}{
        escapeinside=||,
        fontsize=\small,
        style=native,
        linenos=true,
        numberblanklines=true,
        tabsize=4,
        obeytabs=false,
        breaklines=true,
        autogobble=true,
        encoding="utf8",
        showspaces=false,
        xleftmargin=20pt,
        frame=single,
        framesep=5pt
    }
}{
    \newmintedfile[pythonfile]{python}{
        fontsize=\small,
        style=friendly,
        linenos=true,
        numberblanklines=true,
        tabsize=4,
        obeytabs=false,
        breaklines=true,
        autogobble=true,
        encoding="utf8",
        showspaces=false,
        xleftmargin=20pt,
        frame=single,
        framesep=5pt,
    }
    \newmintinline{python}{
        style=friendly,
        encoding="utf8"
    }
    \newmintinline{kotlin}{
        style=friendly,
        encoding="utf8"
    }

    \definecolor{codegray}{HTML}{eaf1ff}
    \newminted[bashcode]{awk}{
        escapeinside=||,
        fontsize=\small,
        style=friendly,
        linenos=true,
        numberblanklines=true,
        tabsize=4,
        obeytabs=false,
        breaklines=true,
        autogobble=true,
        encoding="utf8",
        showspaces=false,
        xleftmargin=20pt,
        frame=single,
        framesep=5pt
    }
}

\let\origpythonfile\pythonfile
\renewcommand{\pythonfile}[1]{\pythonfileh{#1}{}}
\newcommand{\pythonfileh}[2]{\origpythonfile[#2]{#1}}

\DeclareDocumentCommand{\kotlinfile}{O{} O{} m}{\inputCode[#1]{minted language=kotlin,#2}{#3}}

\newcommand*{\ditto}{\texttt{\char`\"}}

\newcommand{\shellprefix}{\textcolor{TUDa-3a}{\ttfamily\bfseries \$~}}
\DeclareTCBListing{commandshell}{ O{} O{} }{
    colback=\IfDarkModeTF{black}{black!80},
    colupper=white,
    colframe=TUDa-3a,
    listing only,
    % listing options={style=tcblatex,language=sh},
    listing engine=minted,
    minted style=dracula,
    minted options={
        % linenos=true,
        numbersep=3mm,
        texcl=true,
        autogobble,
        escapeinside=@@,
        breaklines,
        highlightcolor=yellow!50!black,
        #1
    },
    #2,
    % before upper={\textcolor{red}{\small\ttfamily\bfseries root \$> }},
    % every listing line={\textcolor{red}{\small\ttfamily\bfseries root \$> }}
}

\title{Programmiervorkurs Linux-Befehle}
\subsubtitle{vorkurs@d120.de}

\begin{document}
\maketitle

\section{Dateisystem}

\subsection{cd}
In ein anderes Verzeichnis wechseln.
\begin{bashcode}
    cd |\textit{Pfad zum Verzeichnis}|
\end{bashcode}


\subsection{ls}
Listet alle Dateien im aktuellen Verzeichnis auf.
\begin{bashcode}
    ls
\end{bashcode}
Listet alle Dateien im aktuellen Verzeichnis in Listenform auf.
\begin{bashcode}
    ls -l
\end{bashcode}

\subsection{mv}
Verschiebt oder benennt eine Datei oder ein Verzeichnis um.
\begin{bashcode}
    mv "Name" "neuer Name"
    mv "Pfad/Name" "neuer Pfad/neuer Name"
\end{bashcode}

\subsection{mkdir}
Erstellt ein neues Verzeichnis.
\begin{bashcode}
    mkdir |\textit{Ordnername}|
\end{bashcode}

\subsection{cp}
Kopiert eine Datei.\\
Kopiert einen Ordner mit Option \textit{-R}.
\begin{bashcode}
    cp |\textit{Datei}| |\textit{Kopie}|
    cp -R |\textit{Ordner}| |\textit{KopierterOrdner}|
\end{bashcode}

\subsection{rm}
Löscht eine Datei oder mit Option -R einen Ordner.
\begin{bashcode}
    rm |\textit{Datei}|
    rm -R |\textit{Ordner}|
\end{bashcode}

\section{Programmieren}
\subsection{nano und gedit}
Nano ist ein Konsolen-Texteditor. Gibt man als Parameter eine existierende Datei an, wird diese geöffnet; existiert die Datei nicht, wird sie beim Speichern erstellt. Gedit ist quasi das gleiche mit graphischer Oberfläche.
\begin{bashcode}
    nano |\textit{Datei}|
    gedit |\textit{Datei}|
\end{bashcode}
Innerhalb des Editors kann man mit den Pfeiltasten navigieren und wie gewohnt schreiben.\\
Mit Strg+O kann man die Datei speichern, man wird nach dem Pfad zum Speichern gefragt und kann diesen mit Enter bestätigen.\\
Mit Strg+X verlässt man den Editor.

\subsection{Python}
\subsubsection*{Linux/macOS}
Mit \textit{python3} kann man py-Skripte ausführen, wobei Python 3.x.x. verwendet wird. \textit{python} wiederum verwendet 2.x.x und ist veraltet, weshalb es nicht mehr verwendet werden sollte.
\begin{bashcode}
    python3 |\textit{Datei.py}|
\end{bashcode}

\subsubsection*{Windows}
Mit \textit{python} kann man py-Skripte ausführen.
\begin{bashcode}
    python |\textit{Datei.py}|
\end{bashcode}

\end{document}
