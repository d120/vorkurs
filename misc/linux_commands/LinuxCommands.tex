% !TeX document-id = {505cbeed-ac21-4fdd-a7f4-eefb28b91c1e}
% !TeX TXS-program:compile = txs:///pdflatex/[--shell-escape]
\RequirePackage{import}
\subimport{../../exercises}{preamble.tex}
\usepackage{hyperref}

\title{Programmiervorkurs Linux-Befehle}
\subsubtitle{vorkurs@d120.de}

\begin{document}
\maketitle

\section{Dateisystem}

\subsection{cd}
In ein anderes Verzeichnis wechseln.
\begin{bashcode}
    cd |\textit{Pfad zum Verzeichnis}|
\end{bashcode}


\subsection{ls}
Listet alle Dateien im aktuellen Verzeichnis auf.
\begin{bashcode}
    ls
\end{bashcode}
Listet alle Dateien im aktuellen Verzeichnis in Listenform auf.
\begin{bashcode}
    ls -l
\end{bashcode}

\subsection{mv}
Verschiebt oder benennt eine Datei oder ein Verzeichnis um.
\begin{bashcode}
    mv "Name" "neuer Name"
    mv "Pfad/Name" "neuer Pfad/neuer Name"
\end{bashcode}

\subsection{mkdir}
Erstellt ein neues Verzeichnis.
\begin{bashcode}
    mkdir |\textit{Ordnername}|
\end{bashcode}

\subsection{cp}
Kopiert eine Datei.\\
Kopiert einen Ordner mit Option \textit{-r}.
\begin{bashcode}
    cp |\textit{Datei}| |\textit{Kopie}|
    cp -r |\textit{Ordner}| |\textit{KopierterOrdner}|
\end{bashcode}

\subsection{rm}
Löscht eine Datei oder mit Option -r einen Ordner.
\begin{bashcode}
    rm |\textit{Datei}|
    rm -r |\textit{Ordner}|
\end{bashcode}

\section{Programmieren}
\subsection{nano und gedit}
Nano ist ein Konsolen-Texteditor. Gibt man als Parameter eine existierende Datei an, wird diese geöffnet; existiert die Datei nicht, wird sie beim Speichern erstellt. Gedit ist quasi das gleiche mit graphischer Oberfläche.
\begin{bashcode}
    nano |\textit{Datei}|
    gedit |\textit{Datei}|
\end{bashcode}
Innerhalb des Editors kann man mit den Pfeiltasten navigieren und wie gewohnt schreiben.\\
Mit Strg+O kann man die Datei speichern, man wird nach dem Pfad zum Speichern gefragt und kann diesen mit Enter bestätigen.\\
Mit Strg+X verlässt man den Editor.

\subsection{Kotlin}
Mit \textit{kotlin} kann man kts-Skripte ausführen.
\emph{Achting: Die Dateiendung ist hierbei entscheidend.}
\begin{bashcode}
    kotlin |\textit{Datei.kts}|
\end{bashcode}

Mit \textit{:load} kann man kts-Skripte
auch direkt in eine Repl laden.
\begin{bashcode}
    kotlin
    Welcome to Kotlin version 1.9.10 (JRE 11.0.20.1+1)
    Type :help for help, :quit for quit
    >>> :load |\textit{Datei.kts}|

    ki
    ki-shell 0.5.2/1.7.0
    type :h for help
    [0] :load |\textit{Datei.kts}|
\end{bashcode}

\end{document}
