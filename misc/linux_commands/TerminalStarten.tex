% !TeX document-id = {505cbeed-ac21-4fdd-a7f4-eefb28b91c1e}
% !TeX TXS-program:compile = txs:///pdflatex/[--shell-escape]
% \documentclass[accentcolor=3c,colorbacktitle,12pt]{tudaexercise}
\RequirePackage{import}
\subimport{../../exercises}{preamble.tex}
\usepackage{hyperref}
\title{Starten der Kommandozeile}
\subsubtitle{vorkurs@d120.de}

\begin{document}
\maketitle

\section*{Windows Kommandozeile öffnen (Powershell)}
\begin{enumerate}
    \item Drücke die Windows-Taste + R
    \item Gebe \texttt{powershell} ein und drücke Strg + Enter
\end{enumerate}
Für einen bestimmten Order:
\begin{enumerate}
    \item Shift + Rechtsklick im Explorer
    \item \emph{Powershell hier öffnen} auswählen
\end{enumerate}
\begin{hinweis}
    Falls die \enquote{Terminal App} installiert ist, kann auch diese verwendet werden.
\end{hinweis}

\section*{Mac OS}
Die gesuchte Anwendung heißt \emph{Terminal}.

\section*{Linux}
Auf dem meisten Distros heißt die Anwendung einfach Terminal.
Oft kann man diese mit Strg + Alt + T  öffnen.
So insbesondere auch auf den Poolrechnern.

Je nach Datei Explorer könnt ihr auch einen Order.
\enquote{Im Terminal Öffnen}.

\end{document}
