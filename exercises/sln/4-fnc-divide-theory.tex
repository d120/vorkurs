Bei Gleitkommazahlen werden oft nicht die angegebene Zahlen gespeichert sondern eine Approximation, eine Annäherung. Das hängt unter anderem mit der Art der Speicherung einer Gleitkommazahl zusammen. Während im Dezimalsystem eine 0,1 ganz einfach darzustellen ist, arbeitet der Rechner im Binärsystem (also auf Zweierbasis) und im Binärsystem ist die Dezimalzahl 0,1 periodisch ($0.1_2$=0,00011001100110011...). Da der Speicher einer Gleitkommazahl beschränkt ist, schneidet der Rechner nach einer bestimmten Länge einfach ab.\\
Eine andere Fehlerquelle birgt die Verrechnung - hier beispielhaft die Addition - von sehr großen mit sehr kleinen Zahlen. Um ein konkretes Beispiel zu geben: Wenn man die beiden floats 100.000.000 und 0,000000001 addiert erhält man als Ergebnis 100.000.000. Auch hier reicht die Art der Speicherung für float-Zahlen nicht aus, um eine 100.000.000,000000001 zu speichern. Die genaue Erklärung ersparen wir uns hier, da sie zu technisch wäre.
