\RequirePackage[minted_workaround,caption_workaround,boxarc]{algo-common}
\ExplSyntaxOn
% Solution environment check (enable if SOLUTION=1)
\sys_get_shell:nnN { kpsewhich ~ --var-value ~ SOLUTION } { } \l_solution_env_var_tl
\tl_trim_spaces:N \l_solution_env_var_tl
\tl_if_eq:NnT \l_solution_env_var_tl {1} {\PassOptionsToClass{solution=true}{tudaexercise}}
\ExplSyntaxOff
\documentclass[accentcolor=3c,colorbacktitle,12pt]{tudaexercise}
% \usepackage[T1]{fontenc}
%\usepackage[utf8]{inputenc}
\ifPDFTeX
    \RequirePackage[utf8]{inputenc}
\fi
%\usepackage[ngerman]{babel}
%Includes
\usepackage[ngerman]{babel} %Deutsche Silbentrennung
\usepackage[utf8]{inputenc} %Deutsche Umlaute
\usepackage{float}
\usepackage{graphicx}
\usepackage{minted}
\RequirePackage{csquotes}
\RequirePackage{fontawesome5}

\DeclareGraphicsExtensions{.pdf,.png,.jpg}

\makeatletter
\author{Vorkursteam der Fachschaft Informatik}
\let\Author\@author

% dark mode
\ExplSyntaxOn
\IfDarkModeT{
    \cs_if_exist:NT \setbeamercolor {
        \setbeamercolor*{smallrule}{bg=.}
        \setbeamercolor*{normal~text}{bg=\thepagecolor,fg=.}
        \setbeamercolor*{background~canvas}{parent=normal~text}
        \setbeamercolor*{section~in~toc}{parent=normal~text}
        \setbeamercolor*{subsection~in~toc}{parent=normal~text,fg=.}
        \setbeamercolor*{footline}{parent=normal~text}
        \setbeamercolor{block~title~alerted}{fg=white,bg=white!20!\thepagecolor}
        \setbeamercolor*{block~body}{bg=black!70!gray!98!blue}
        \setbeamercolor*{block~body~alerted}{bg=\thepagecolor}
    }
    \cs_if_exist:NT \setbeamertemplate {
        \setbeamertemplate{subsection~in~toc~shaded}[default][50]
    }
}
\ExplSyntaxOff

% macros
\renewcommand{\arraystretch}{1.2} % Höhe einer Tabellenspalte minimal erhöhen
\newcommand{\N}{{\mathbb N}}
\renewcommand{\code}{\inputminted[]{python}}

\IfDarkModeTF{
    \newmintedfile[pythonfile]{python}{
        fontsize=\small,
        style=native,
        linenos=true,
        numberblanklines=true,
        tabsize=4,
        obeytabs=false,
        breaklines=true,
        autogobble=true,
        encoding="utf8",
        showspaces=false,
        xleftmargin=20pt,
        frame=single,
        framesep=5pt,
    }
    \newmintinline{python}{
        style=native,
        encoding="utf8"
    }
    \newmintinline{kotlin}{
        style=native,
        encoding="utf8"
    }


    \definecolor{codegray}{HTML}{eaf1ff}
    \newminted[bashcode]{awk}{
        escapeinside=||,
        fontsize=\small,
        style=native,
        linenos=true,
        numberblanklines=true,
        tabsize=4,
        obeytabs=false,
        breaklines=true,
        autogobble=true,
        encoding="utf8",
        showspaces=false,
        xleftmargin=20pt,
        frame=single,
        framesep=5pt
    }
}{
    \newmintedfile[pythonfile]{python}{
        fontsize=\small,
        style=friendly,
        linenos=true,
        numberblanklines=true,
        tabsize=4,
        obeytabs=false,
        breaklines=true,
        autogobble=true,
        encoding="utf8",
        showspaces=false,
        xleftmargin=20pt,
        frame=single,
        framesep=5pt,
    }
    \newmintinline{python}{
        style=friendly,
        encoding="utf8"
    }
    \newmintinline{kotlin}{
        style=friendly,
        encoding="utf8"
    }

    \definecolor{codegray}{HTML}{eaf1ff}
    \newminted[bashcode]{awk}{
        escapeinside=||,
        fontsize=\small,
        style=friendly,
        linenos=true,
        numberblanklines=true,
        tabsize=4,
        obeytabs=false,
        breaklines=true,
        autogobble=true,
        encoding="utf8",
        showspaces=false,
        xleftmargin=20pt,
        frame=single,
        framesep=5pt
    }
}

\let\origpythonfile\pythonfile
\renewcommand{\pythonfile}[1]{\pythonfileh{#1}{}}
\newcommand{\pythonfileh}[2]{\origpythonfile[#2]{#1}}

\DeclareDocumentCommand{\kotlinfile}{O{} O{} m}{\inputCode[#1]{minted language=kotlin,#2}{#3}}

\newcommand*{\ditto}{\texttt{\char`\"}}

\newcommand{\shellprefix}{\textcolor{TUDa-3a}{\ttfamily\bfseries \$~}}
\DeclareTCBListing{commandshell}{ O{} O{} }{
    colback=\IfDarkModeTF{black}{black!80},
    colupper=white,
    colframe=TUDa-3a,
    listing only,
    % listing options={style=tcblatex,language=sh},
    listing engine=minted,
    minted style=dracula,
    minted options={
        % linenos=true,
        numbersep=3mm,
        texcl=true,
        autogobble,
        escapeinside=@@,
        breaklines,
        highlightcolor=yellow!50!black,
        #1
    },
    #2,
    % before upper={\textcolor{red}{\small\ttfamily\bfseries root \$> }},
    % every listing line={\textcolor{red}{\small\ttfamily\bfseries root \$> }}
}
\usepackage{hyperref}
\usepackage{ifthen}
\usepackage{listings}
%\usepackage{graphicx}
\usepackage{multicol}
\usepackage{multirow}
\usepackage{amssymb}
\RequirePackage{silence}
\WarningFilter[sillyfonterror]{latex}{Font~shape~declaration~has}
\ActivateWarningFilters[sillyfonterror]
\ifLuaTeX
    % Fix Font warnings for mathdesign
    \DeclareFontFamily{TU}{mdbch}{}
    \DeclareFontShape{TU}{mdbch}{m}{n}{
        <-> \UnicodeFontFile{lmroman10-regular}{\UnicodeFontTeXLigatures}
    }{}
    \RequirePackage[utf8]{luainputenc} % if problems with ß exist
\fi

\definecolor{darkblue}{rgb}{0,0,.5}
\hypersetup{colorlinks=true, breaklinks=true, linkcolor=\IfDarkModeTF{cyan}{darkblue}, menucolor=darkblue, urlcolor=\IfDarkModeTF{cyan}{darkblue}}

% configuration
\makeatletter
\@ifundefined{c@ex}{
    \newcounter{ex}\setcounter{ex}{1}
}{}
\makeatother
\IfDarkModeT{
    \selectcolormodel{RGB}
}
\newboolean{sln}\setboolean{sln}{false}
\newboolean{SoSe}\setboolean{SoSe}{false}
\newcommand{\thenextyear}{\the\numexpr\year+1\relax}
%

\newcommand{\ext}{py}
\newcommand{\sln}[1]{\ifthenelse{\boolean{sln}}{\subsubsection*{Antwort}{\itshape #1}}{}}
\newcommand{\slnformat}[1]{\ifthenelse{\boolean{sln}}{#1}{}}
\newcommand{\vorkurstaskformat}[1]{\ifthenelse{\boolean{sln}}{}{#1}}
\newcommand{\vorkurstask}[1]{\input{task/#1}\IfFileExists{./sln/#1.tex}{\sln{\input{sln/#1}}}{\IfFileExists{./sln/#1.\ext}{\sln{\pythonfile{sln/#1.\ext}}}{\ClassError{Vorkurs-TeX}{No solution specified for task #1}{Add solution file #1.tex or #1.\ext}}}}
\newcommand{\mccmd}{Kreuze zu jeder Antwort an, ob sie zutrifft (\textbf{w}) oder nicht (\textbf{f}).}
\newcommand{\mchead}{\item[\textbf{w} \textbf{f} ]}
\newcommand{\mcitem}[1]{\item[$\square\ \square$] #1}
\newcommand{\mcitemt}[1]{\item[$\ifthenelse{\boolean{sln}}{\blacksquare}{\square}\ \square$] #1}
\newcommand{\mcitemf}[1]{\item[$\square\ \ifthenelse{\boolean{sln}}{\blacksquare}{\square}$] #1}
\newcommand{\ptitle}{\ifthenelse{\boolean{SoSe}}{Sommersemester \the\year}{Wintersemester \the\year/\thenextyear}}

\newcommand{\lstinlinenoit}[1]{\upshape{\lstinline|#1|}\itshape}
\lstset{language=Python, basicstyle=\ttfamily\small, keywordstyle=\color{\IfDarkModeTF{cyan!60!black}{blue!80!black}}, identifierstyle=, commentstyle=\color{green!50!.}, stringstyle=\ttfamily,
    tabsize=4, breaklines=true, numbers=left, numberstyle=\small, frame=single, backgroundcolor=\color{blue!3!\thepagecolor}}
\author{Fachschaft Informatik}

\newcommand{\stage}[1]{(\ifcase#1\or{Einstieg}\or{Vertiefung}\or{Herausforderung}\else\fi)}
\newcommand{\bonus}[1]{\textit{BonusFact: }#1}

\sheetnumber{2}
\title{Aufgaben Programmiervorkurs}
\subtitle{von der Fachschaft Informatik\hfill\ptitle}

\usepackage{enumitem}

\begin{document}
\maketitle{}

\begin{task}[points=auto]{Variablen \stage1}
    \begin{subtask*}[points=0]{Fehlersuche}
        Gegeben ist folgendes Listing:
        \begin{codeBlock}[]{minted language=python}
            meinAlter : int = 21
            meinalter = meinalter + 1
            print(meinAlter)
        \end{codeBlock}
        Beim Ausführen der zweiten Zeile wirft Python einen Fehler. Warum? Verbessert das Listing.
        \begin{solution}
            Die Variable in Zeile 2 müsste {\ttfamily meinAlter} heißen und nicht {\ttfamily meinalter}.
        \end{solution}
    \end{subtask*}
    \begin{subtask*}[points=0]{Namenskonvention}
        \begin{multicols}{3}
            \begin{enumerate}[label=(\alph*)]
                \item \mintinline{text}{matrikelNummer}
                \item \mintinline{text}{ÄhmKeineAhnung}
                \item \mintinline{text}{2fancy4kotlin}
                \item \mintinline{text}{_meinAlter}
                \item \mintinline{text}{richtig&gut}
                \item \mintinline{text}{_2Euro}
                \item \mintinline{text}{Variable2}
                \item \mintinline{text}{nochBesser}
                \item \mintinline{text}{#Falsch}
            \end{enumerate}
        \end{multicols}
        (i) Welche der aufgelisteten Variablennamen sind \textbf{gültige} Variablennamen?

        (ii) Und welche davon entsprechen der Namenskonvention \enquote{camelCase}?
        \begin{solution}
            (i) Die Antworten c), e) und i) sind nicht gültig.\\Gültig sind somit:
            \mintinline{text}{Matrikelnummer},
            \mintinline{text}{ÄhmKeineAhnung},
            \mintinline{text}{_meinAlter},
            \mintinline{text}{_2Euro},
            \mintinline{text}{Variable2},
            \mintinline{text}{noch\_$besser}

            (ii) Nur die Antworten a) und h) entsprechen der Namenskonvention \enquote{camelCase} und sind gültig.
        \end{solution}
    \end{subtask*}
    \begin{subtask*}[points=0]{Assignment}
        Ergänzt im folgenden Listing die fehlenden Variablenzuweisungen, so dass keine Fehler auftreten und am Ende {\ttfamily variable7} zu {\ttfamily True}, {\ttfamily variable2} zu {\ttfamily 5} und {\ttfamily variable5} zu {\ttfamily \verb+"abc"+} auswertet.
        \begin{codeBlock}[]{minted language=python}
            variable1 = # hier Zuweisung 1 einsetzen
            variable2 = 3
            variable3 = # hier Zuweisung 2 einsetzen
            variable2 = variable2 + variable1
            if variable3 > 5:
                variable2 = 0
            else:
                variable2 *= variable3
            variable4 = # hier Zuweisung 3 einsetzen
            variable5 = # hier Zuweisung 4 einsetzen
            variable5 = variable5 + variable4
            variable6 = # hier Zuweisung 5 einsetzen
            variable7 = variable6 > variable2 * 2
            variable5 = variable5 + "c"
            print(variable7)
            print(variable2)
            print(variable5)
        \end{codeBlock}

        \textit{Hinweis: Hier habt ihr 5 Zeilen, wo ihr eure Zuweisungen einsetzen könnt. Es kann hilfreich sein, mit den Ergebnissen anzufangen, die am Ende rauskommen sollen, und euch dann von hinten nach vorne durchzuarbeiten. In der realistischem Code würde man natürlich bezeichnendere Namen wählen.}

        \begin{solution}
            \begin{itemize}
                \item \pythoninline{variable1 = 2}
                \item \pythoninline{variable3 = 1}
                \item \pythoninline{variable4 = "b"}
                \item \pythoninline{variable5 = "a"}
                \item \pythoninline{variable7 = 11}
            \end{itemize}
            Es gibt natürlich auch noch andere Möglichkeiten
        \end{solution}
    \end{subtask*}
\end{task}
\begin{task}[points=auto]{Logische Operationen \stage1}
    \begin{subtask*}[points=0]{Operatoren}
        Es seien die folgenden Variablen deklariert und initialisiert:
        \begin{codeBlock}[]{title=\codeBlockTitle{asdf},minted language=python}
            x: int = 6
            y: int = 7
            z: int = 0
            a: int = False
        \end{codeBlock}
        Welchen Wert enthält die Variable \pythoninline{b} jeweils nach Ausführung der folgenden Anweisungen?
        \begin{enumerate}
            \item \pythoninline{b: bool = x > 5 or y < 7 and z != 0}
            \item \pythoninline{b: bool = x * y != y * x and x / z == 0}
            \item \pythoninline{b: bool = not (x == y) and z <= 0}
            \item \pythoninline{b: bool = x >= 11 or x < 9 and not(y == 2) and x + y * z > 0 or a}
            \item \pythoninline{b: bool = z != z or not a and x - y * z <= 0}
            \item \pythoninline{b: bool = not a and z < y - x}
        \end{enumerate}

        \begin{solution}
            \begin{multicols}{3}
                \begin{enumerate}
                    \item \pythoninline{b == True}
                    \item \pythoninline{b == False}
                    \item \pythoninline{b == True}
                    \item \pythoninline{b == True}
                    \item \pythoninline{b == False}
                    \item \pythoninline{b == True}
                \end{enumerate}
            \end{multicols}
        \end{solution}
    \end{subtask*}
\end{task}
\begin{task}[points=auto]{Eingabe/Ausgabe \stage2}
    \begin{subtask*}[points=0]{Variablen einlesen}
        Im Rahmen der Vorlesung habt ihr die Funktion \pythoninline{input()} kennengelernt, welche es euch erlaubt, auf einfache Art und Weise Benutzereingaben von der Konsole einzulesen. Schreibt jetzt ein Programm, welches mit Hilfe dieser Funktion nacheinander folgende Eingaben einliest und in Variablen speichert:
        \begin{itemize}
            \item euren Namen als Zeichenkette
            \item euer Geburtsdatum als drei ganze Zahlen: Tag, Monat, Jahr
        \end{itemize}
        Insgesamt sollen also vier Eingaben verarbeiten werden. Achtet dabei darauf, dass euer Programm die erwarteten Eingaben textuell klar strukturiert (Stichwort: Bedienerfreundlichkeit).
        Nachdem die letzte Eingabe verarbeitet ist, soll folgende Zeichenkette ausgegeben werden (wobei natürlich die Platzhalter zu ersetzen sind):

        Ausgabe: \pythoninline{"<euer Name> hat am <tt.mm.jjjj> Geburtstag."}

        \begin{hinweise}
            \begin{itemize}
                \item Die Verarbeitung von Eingaben wird im Foliensatz \verb+03_Variablen.pdf+ erläutert.
                \item Variablen sollten in Strings nicht mit \enquote{\pythoninline{+}} eingefügt werden, stattdessen an die entsprechenden Stellen im String mit String-Templates einsetzen.

                    Beispiel: \pythoninline{f"{variable0} ist kleiner {variable1}"} ergibt wenn \pythoninline{variable0 == 4} und\\
                    \pythoninline{variable1 == 42} ist den String \pythoninline{"4 ist kleiner 42"}
            \end{itemize}
        \end{hinweise}

        \begin{solution}
            \begin{codeBlock}[]{minted language=python}
                name : str = input("Bitte gib deinen Namen ein: ")
                tag : int = int(input("Gib den Tag deiner Geburt ein: "))
                monat : int = int(input("Gib den Monat deiner Geburt ein: "))
                jahr : int = int(input("Gib das Jahr deiner Geburt ein: "))

                print(f"{name} hat am {tag}.{monat}.{jahr} Geburtstag.")
            \end{codeBlock}
        \end{solution}
    \end{subtask*}
\end{task}
\end{document}
