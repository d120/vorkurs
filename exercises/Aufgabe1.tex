\RequirePackage[minted_workaround,caption_workaround,boxarc]{algo-common}
\ExplSyntaxOn
% Solution environment check (enable if SOLUTION=1)
\sys_get_shell:nnN { kpsewhich ~ --var-value ~ SOLUTION } { } \l_solution_env_var_tl
\tl_trim_spaces:N \l_solution_env_var_tl
\tl_if_eq:NnT \l_solution_env_var_tl {1} {\PassOptionsToClass{solution=true}{tudaexercise}}
\ExplSyntaxOff
\documentclass[accentcolor=3c,colorbacktitle,12pt]{tudaexercise}
% \usepackage[T1]{fontenc}
%\usepackage[utf8]{inputenc}
\ifPDFTeX
    \RequirePackage[utf8]{inputenc}
\fi
%\usepackage[ngerman]{babel}
%Includes
\usepackage[ngerman]{babel} %Deutsche Silbentrennung
\usepackage[utf8]{inputenc} %Deutsche Umlaute
\usepackage{float}
\usepackage{graphicx}
\usepackage{minted}
\RequirePackage{csquotes}
\RequirePackage{fontawesome5}

\DeclareGraphicsExtensions{.pdf,.png,.jpg}

\makeatletter
\author{Vorkursteam der Fachschaft Informatik}
\let\Author\@author

% dark mode
\ExplSyntaxOn
\IfDarkModeT{
    \cs_if_exist:NT \setbeamercolor {
        \setbeamercolor*{smallrule}{bg=.}
        \setbeamercolor*{normal~text}{bg=\thepagecolor,fg=.}
        \setbeamercolor*{background~canvas}{parent=normal~text}
        \setbeamercolor*{section~in~toc}{parent=normal~text}
        \setbeamercolor*{subsection~in~toc}{parent=normal~text,fg=.}
        \setbeamercolor*{footline}{parent=normal~text}
        \setbeamercolor{block~title~alerted}{fg=white,bg=white!20!\thepagecolor}
        \setbeamercolor*{block~body}{bg=black!70!gray!98!blue}
        \setbeamercolor*{block~body~alerted}{bg=\thepagecolor}
    }
    \cs_if_exist:NT \setbeamertemplate {
        \setbeamertemplate{subsection~in~toc~shaded}[default][50]
    }
}
\ExplSyntaxOff

% macros
\renewcommand{\arraystretch}{1.2} % Höhe einer Tabellenspalte minimal erhöhen
\newcommand{\N}{{\mathbb N}}
\renewcommand{\code}{\inputminted[]{python}}

\IfDarkModeTF{
    \newmintedfile[pythonfile]{python}{
        fontsize=\small,
        style=native,
        linenos=true,
        numberblanklines=true,
        tabsize=4,
        obeytabs=false,
        breaklines=true,
        autogobble=true,
        encoding="utf8",
        showspaces=false,
        xleftmargin=20pt,
        frame=single,
        framesep=5pt,
    }
    \newmintinline{python}{
        style=native,
        encoding="utf8"
    }
    \newmintinline{kotlin}{
        style=native,
        encoding="utf8"
    }


    \definecolor{codegray}{HTML}{eaf1ff}
    \newminted[bashcode]{awk}{
        escapeinside=||,
        fontsize=\small,
        style=native,
        linenos=true,
        numberblanklines=true,
        tabsize=4,
        obeytabs=false,
        breaklines=true,
        autogobble=true,
        encoding="utf8",
        showspaces=false,
        xleftmargin=20pt,
        frame=single,
        framesep=5pt
    }
}{
    \newmintedfile[pythonfile]{python}{
        fontsize=\small,
        style=friendly,
        linenos=true,
        numberblanklines=true,
        tabsize=4,
        obeytabs=false,
        breaklines=true,
        autogobble=true,
        encoding="utf8",
        showspaces=false,
        xleftmargin=20pt,
        frame=single,
        framesep=5pt,
    }
    \newmintinline{python}{
        style=friendly,
        encoding="utf8"
    }
    \newmintinline{kotlin}{
        style=friendly,
        encoding="utf8"
    }

    \definecolor{codegray}{HTML}{eaf1ff}
    \newminted[bashcode]{awk}{
        escapeinside=||,
        fontsize=\small,
        style=friendly,
        linenos=true,
        numberblanklines=true,
        tabsize=4,
        obeytabs=false,
        breaklines=true,
        autogobble=true,
        encoding="utf8",
        showspaces=false,
        xleftmargin=20pt,
        frame=single,
        framesep=5pt
    }
}

\let\origpythonfile\pythonfile
\renewcommand{\pythonfile}[1]{\pythonfileh{#1}{}}
\newcommand{\pythonfileh}[2]{\origpythonfile[#2]{#1}}

\DeclareDocumentCommand{\kotlinfile}{O{} O{} m}{\inputCode[#1]{minted language=kotlin,#2}{#3}}

\newcommand*{\ditto}{\texttt{\char`\"}}

\newcommand{\shellprefix}{\textcolor{TUDa-3a}{\ttfamily\bfseries \$~}}
\DeclareTCBListing{commandshell}{ O{} O{} }{
    colback=\IfDarkModeTF{black}{black!80},
    colupper=white,
    colframe=TUDa-3a,
    listing only,
    % listing options={style=tcblatex,language=sh},
    listing engine=minted,
    minted style=dracula,
    minted options={
        % linenos=true,
        numbersep=3mm,
        texcl=true,
        autogobble,
        escapeinside=@@,
        breaklines,
        highlightcolor=yellow!50!black,
        #1
    },
    #2,
    % before upper={\textcolor{red}{\small\ttfamily\bfseries root \$> }},
    % every listing line={\textcolor{red}{\small\ttfamily\bfseries root \$> }}
}
\usepackage{hyperref}
\usepackage{ifthen}
\usepackage{listings}
%\usepackage{graphicx}
\usepackage{multicol}
\usepackage{multirow}
\usepackage{amssymb}
\RequirePackage{silence}
\WarningFilter[sillyfonterror]{latex}{Font~shape~declaration~has}
\ActivateWarningFilters[sillyfonterror]
\ifLuaTeX
    % Fix Font warnings for mathdesign
    \DeclareFontFamily{TU}{mdbch}{}
    \DeclareFontShape{TU}{mdbch}{m}{n}{
        <-> \UnicodeFontFile{lmroman10-regular}{\UnicodeFontTeXLigatures}
    }{}
    \RequirePackage[utf8]{luainputenc} % if problems with ß exist
\fi

\definecolor{darkblue}{rgb}{0,0,.5}
\hypersetup{colorlinks=true, breaklinks=true, linkcolor=\IfDarkModeTF{cyan}{darkblue}, menucolor=darkblue, urlcolor=\IfDarkModeTF{cyan}{darkblue}}

% configuration
\makeatletter
\@ifundefined{c@ex}{
    \newcounter{ex}\setcounter{ex}{1}
}{}
\makeatother
\IfDarkModeT{
    \selectcolormodel{RGB}
}
\newboolean{sln}\setboolean{sln}{false}
\newboolean{SoSe}\setboolean{SoSe}{false}
\newcommand{\thenextyear}{\the\numexpr\year+1\relax}
%

\newcommand{\ext}{py}
\newcommand{\sln}[1]{\ifthenelse{\boolean{sln}}{\subsubsection*{Antwort}{\itshape #1}}{}}
\newcommand{\slnformat}[1]{\ifthenelse{\boolean{sln}}{#1}{}}
\newcommand{\vorkurstaskformat}[1]{\ifthenelse{\boolean{sln}}{}{#1}}
\newcommand{\vorkurstask}[1]{\input{task/#1}\IfFileExists{./sln/#1.tex}{\sln{\input{sln/#1}}}{\IfFileExists{./sln/#1.\ext}{\sln{\pythonfile{sln/#1.\ext}}}{\ClassError{Vorkurs-TeX}{No solution specified for task #1}{Add solution file #1.tex or #1.\ext}}}}
\newcommand{\mccmd}{Kreuze zu jeder Antwort an, ob sie zutrifft (\textbf{w}) oder nicht (\textbf{f}).}
\newcommand{\mchead}{\item[\textbf{w} \textbf{f} ]}
\newcommand{\mcitem}[1]{\item[$\square\ \square$] #1}
\newcommand{\mcitemt}[1]{\item[$\ifthenelse{\boolean{sln}}{\blacksquare}{\square}\ \square$] #1}
\newcommand{\mcitemf}[1]{\item[$\square\ \ifthenelse{\boolean{sln}}{\blacksquare}{\square}$] #1}
\newcommand{\ptitle}{\ifthenelse{\boolean{SoSe}}{Sommersemester \the\year}{Wintersemester \the\year/\thenextyear}}

\newcommand{\lstinlinenoit}[1]{\upshape{\lstinline|#1|}\itshape}
\lstset{language=Python, basicstyle=\ttfamily\small, keywordstyle=\color{\IfDarkModeTF{cyan!60!black}{blue!80!black}}, identifierstyle=, commentstyle=\color{green!50!.}, stringstyle=\ttfamily,
    tabsize=4, breaklines=true, numbers=left, numberstyle=\small, frame=single, backgroundcolor=\color{blue!3!\thepagecolor}}
\author{Fachschaft Informatik}

\newcommand{\stage}[1]{(\ifcase#1\or{Einstieg}\or{Vertiefung}\or{Herausforderung}\else\fi)}
\newcommand{\bonus}[1]{\textit{BonusFact: }#1}

\sheetnumber{1}
\title{Aufgaben Programmiervorkurs}
\subtitle{von der Fachschaft Informatik\hfill\ptitle}

\usepackage{hyperref}
\usepackage{wrapfig}
\usepackage{footnote}

\begin{document}
\maketitle{}

\begin{task}[points=auto]{Einleitung}
    \begin{subtask*}[points=0]{Python Installieren}
        Hey, willkommen bei den Übungen zum Programmiervorkurs im \ptitle. Wie in der
        Vorlesung verwenden wir für die Übungen die Programmiersprache Python. \\\\
        %
        Für das Installieren und Starten von Python haben wir eigene Guides in Moodle verlinkt.
        Folgt denen, sodass ihr Python ausführen könnt.
        \begin{solution}
            Python ist installiert und dem Pfad hinzugefügt. Der Befehl
            \mintinline{text}{python --version} gibt eine Versionsnummer von mindestens \mintinline{text}{3.12.6} aus.
            % Der Python Interpreter mit \mintinline{text}{python} wurde gestartet.

            Die REPL wurde korrekt installiert und mit \mintinline{text}{ipython} gestartet. Der Befehl
            \mintinline{text}{ipython} gibt eine Versionsnummer von mindestens \mintinline{text}{8.27.0} aus.
        \end{solution}
    \end{subtask*}
    \begin{subtask*}[points=0]{Theoriefragen}
        \begin{itemize}
            \mchead
            \mcitemt{Programmcode steht üblicherweise in Textdateien}
            \mcitemf{Um Python auszuführen, muss ich einen neuen Computer kaufen}
            \mcitemt{Um Python-Code auszuführen kann eine Konsole verwendet werden}
            \mcitemf{Es gibt genau drei Programmiersprachen}
            \mcitemf{Es ist niemals sinnvoll seinen Code mit Kommentaren zu überladen}
            \mcitemt{Eine if-Anweisung beeinflusst den Programmablauf.}
            \mcitemf{Eine if-Anweisung benötigt immer eine else-Klausel.}
            \mcitemt{Ein \lstinline{if (0 < 3)} führt immer den eingeschlossenen Codeblock aus.}
            \mcitemt{Man kann if-Anweisungen verschachteln}
        \end{itemize}
        \begin{solution}
            Die zutreffenden Antworten sind die Aussagen 1, 3, 6 und 9.
        \end{solution}
    \end{subtask*}
\end{task}
\begin{task}[points=auto]{Ausdrücke \stage1}
    \begin{subtask*}[points=0]{Frage \& Antwort}
        Gebt nacheinander die folgenden Ausdrücke in \pythoninline{Python} ein.
        Drückt nach jedem Eintrag \textit{ENTER}, um den Ausdruck auszuwerten.
        Was passiert? Versucht, zu erklären, warum.
        Probiert auch noch weitere Ausdrücke,
        die euch einfallen.

        \begin{multicols}{3}
            \begin{itemize}
                \item \mintinline{text}{2}
                \item \mintinline{text}{10.0}
                \item \mintinline{text}{"1"}
                \item \mintinline{text}{Test}
                \item \mintinline{text}{"Test"}
                \item \mintinline{text}{true}
                \item \mintinline{text}{True}
                \item \mintinline{text}{false}
                \item \mintinline{text}{False}
            \end{itemize}
        \end{multicols}

        Zusatz: Lasse dir jeweils auch mit \pythoninline{type(...)} den Typ angeben.

        \begin{solution}
            Zahlen ohne Anführungszeichen ergeben die Zahl. Rationale Zahlen brauchen einen
            Dezimal\fatsf{punkt}.
            Doppelte Anführungszeichen (\mintinline{text}{"}) ergeben Strings,
            Wahrheitswerte müssen groß geschrieben werden (\pythoninline{True},
            \pythoninline{False}).
        \end{solution}
    \end{subtask*}
    \begin{subtask*}[points=0]{Mathe 0 für Informatiker*innen}
        Wenn euch das Programm aber nur das zurückgeben könnte, was ihr hinschreibt,
        dann wäre das ja noch lange kein \textit{Computer}. Darum rechnen wir nun etwas.
        Überlegt euch was die folgenden Ausdrücke ergeben, und was die verwendeten
        Symbole für Operationen bezeichnen.
        Beachtet dabei, die Operatorenpräzedenz.
        Lasse dir auch immer den Typ ausgeben.

        \begin{multicols}{3}
            \begin{itemize}
                \item \pythoninline{16 + 26}
                \item \pythoninline{13.75 + 28.67}
                \item \pythoninline{2 * 3 + 6 * 6}
                \item \pythoninline{12 * 133 // 28 - 15}
                \item \pythoninline{1.3 * 5.6}
                \item \pythoninline{2**5 + 2**3 + 2**1}
            \end{itemize}
        \end{multicols}

        \begin{solution}
            Das Ergebnis ist meistens eine Variante von \pythoninline{42}.
            Wichtig ist, dass hier Punkt- vor Strichrechnung gilt.
            Beachtet die unterschiedlichen Datentypen.
            Die letzte Aufgabe betont, dass manche Operationen auf Kommazahlen keine genauen Ergebnisse liefern.
        \end{solution}
    \end{subtask*}
\end{task}
\begin{task}[points=auto]{Konvertierung \stage2}
    \begin{subtask*}[points=0]{Konvertierung}
        Nehmen wir mal an, ihr habt den Text \pythoninline{"1000"} zur Verfügung. Vielleicht
        errechnet; vielleicht aus einer API (\textit{application programming interface}, allgemeine Bezeichnung für Schnittstellen zwischen Programmen)? Es kommt immer wieder vor, dass die Daten noch den falschen Typ haben, wenn ihr sie bekommt.

        Gebt die folgenden Ausdrücke ein, und findet das Ergebnis heraus.

        \textit{Hinweis: es kann sein, dass einige Konvertierungen fehlschlagen. In diesem Fall überlegt euch, warum. Die Fehlermeldung kann dabei hilfreich sein.}

        \begin{multicols}{3}
            \begin{enumerate}
                \item \pythoninline{int("100")}
                \item \pythoninline{float("1.3")}
                \item \pythoninline{str(123)}
                \item \pythoninline{str(2) + "5"}
                \item \pythoninline{str(2 + 5)}
                \item \pythoninline{float(5)}
                \item \pythoninline{int(5.5)}
                \item \pythoninline{int("4.2")}
                \item \pythoninline{float("5")}
            \end{enumerate}
        \end{multicols}

        \begin{solution}
            \begin{multicols}{3}
                \begin{enumerate}
                    \item \pythoninline{int = 100}
                    \item \pythoninline{float = 1.3}
                    \item \pythoninline{str = "123"}
                    \item \pythoninline{str = 25}
                    \item \pythoninline{str = 7}
                    \item \pythoninline{float = 5.0}
                    \item \pythoninline{int = 5}
                    \item Format Error
                    \item \pythoninline{float = 5.0}
                \end{enumerate}
            \end{multicols}
        \end{solution}
    \end{subtask*}
    \begin{subtask*}[points=0]{Verrückte Eingabe \stage3}
        Versucht einen Ausdruck zu finden, der nur \pythoninline{"1"}, \pythoninline{0},
        \pythoninline{int}, \pythoninline{str} und Operationen verwendet, der
        insgesamt zu \pythoninline{3000} auswertet. Beachtet, dass solche Ausdrücke
        wie ihr hier schreiben sollt, so \textbf{niemals} in echtem Code auftauchen
        sollten.

        \textit{Hinweis: Zwar soll am Ende eine Zahl rauskommen, ihr werdet aber in dieser Aufgabe oft zwischen Zahl und Text hin- und herkonvertieren müssen.}

        \begin{solution}
            \begin{codeBlock}[]{minted language=python}
                In [1]: int(str(int("1") + int("1") + int("1")) + str(0) + str(0) + str(0))
                Out[1]: 3000
            \end{codeBlock}
            Es gibt natürlich auch noch andere Möglichkeiten.
        \end{solution}
    \end{subtask*}
\end{task}
\begin{task}[points=auto]{Fehler}
    \begin{subtask*}[points=0]{Fehlertypen \stage1}

        Ordnet die folgenden drei Fehlertypen den folgenden Ausdrücken zu.
        \begin{multicols}{3}
            \begin{itemize}
                \item[] \textbf{SyntaxError}
                \item[] \textbf{ZeroDivisionError}
                \item[] \textbf{NameError}
            \end{itemize}
            \begin{itemize}
                \item[] \pythoninline{true}
                \item[] \pythoninline{int)3(}
                \item[] \pythoninline{2 // 0}
            \end{itemize}
            \begin{itemize}
                \item[] \textit{Syntaktischer Fehler}
                \item[] \textit{Lexikalischer Fehler}
                \item[] \textit{Semantischer Fehler}
            \end{itemize}
        \end{multicols}

        \begin{solution}
            \begin{enumerate}
                \item \pythoninline{true} ist ein falsch geschriebenes Schlüsselwort und somit ein \\
                    lexikalischer Fehler (NameError)
                \item \pythoninline{int)3(} hat die Klammern vertauscht, somit ein \\
                    syntaktischer Fehler (SyntaxError)
                \item \pythoninline{2 // 0} ist eine undefinierte Rechnung, also ein
                    \\ semantischer Fehler (ZeroDivisionError)
            \end{enumerate}

        \end{solution}
    \end{subtask*}
    \begin{subtask*}[points=0]{Finde den Fehler \stage2}
        Gegeben sind die folgenden Listings.
        Führt die Programme aus und beschreibt, warum jeweils ein Fehler auftritt.
        \begin{codeBlock}[]{minted language=python}
            >>> prinnt("Hello World")  // a)
            >>> print(Hello World!)    // b)
            >>> print("Alter: " + 18)  // c)
            >>> print("Hello World!)   // d)
        \end{codeBlock}
        \begin{solution}
            \begin{enumerate}
                \item \mintinline{text}{NameError}: Ein \textcolor{red}{Lexikalischer Fehler}, da es keine Funktion mit dem angegebenen Namen
                    \mintinline{text}{prinnt} gibt.
                    Gültig wäre \mintinline{text}{print}.
                    Ein klassische Tippfehler.
                \item Ein \textcolor{red}{Syntaktischer Fehler},
                    da Strings immer in einfachen oder doppelten Anführungszeichen stehen müssen.
                \item \mintinline{text}{TypeError}: Ein \textcolor{red}{Semantischer Fehler}, da Texte und Zahlen nicht addierbar sind.
                    \textbf{Bonus Fact}: Hierfür kann man z.B. \pythoninline{"Alter: " + str(18)} nutzen.
                \item Ein \textcolor{red}{Syntaktischer Fehler}, da der Rest der Zeile nun auch als String gelesen wird, der aber nie endet.
            \end{enumerate}

        \end{solution}
    \end{subtask*}
\end{task}

\begin{task}{Das erste Programm}
    Für die folgenden Aufgaben solltet ihr jeweils eine Python-Programmdatei
    anlegen.
    Öffnet dazu einen beliebigen Ordner in eurem Dateiexplorer.
    Öffnet dann ein Terminal im selben Verzeichnis.

    \begin{itemize}
        \item Unter Linux \& Mac (u.a. den Poolrechnern): Rechtsklick $\to$ In Terminal öffnen
        \item Unter Windows: Shift+Rechtsklick $\to$ Powershell hier öffnen (bei Windows 11: Terminal hier öffnen)
    \end{itemize}

    Öffnet dann einen Texteditor eurer Wahl. Unter Linux z.B.
    {\ttfamily gedit, kate, usw.}, unter Windows kann Notepad verwendet werden.
    Erstellt eine Datei mit eurem Programm und speichert sie mit der Endung
    {\ttfamily.py}, beispielsweise {\ttfamily programm.py}
    in dem vorher ausgewählten Ordner.

    Als Programm könnt ihr das beliebte Hello World Programm
    \footnote{\url{https://en.wikipedia.org/wiki/\%22Hello,_World!\%22_program}}
    implementieren.
    Gebt dafür \pythoninline{"Hello, World!"} aus.

    Ihr könnt diese Datei dann starten,
    indem ihr {\ttfamily python programm.py} in dem geöffneten Terminal ausführt.

    \begin{solution}
        Eine Python Datei kann erstellt und ausgeführt werden.
    \end{solution}
\end{task}

\begin{task}{Komplexere Programme}
    Schreibe nun ein etwas komplexere Python-Programme:

    \begin{subtask*}{Teilbarkeit}
        \begin{enumerate}
            \item Lasse dir von dem Nutzer zwei ganze Zahlen eingeben.
            \item Ist eine Zahl ein Vielfaches der anderen, gib aus,
                welche ein Vielfaches von welcher ist.
            \item Andernfalls gib aus, dass es keine Vielfachen gibt.
        \end{enumerate}

        Beispiel:
        \begin{itemize}
            \item $5, 15 \to$ 15 ist ein Vielfaches von 5
            \item $5, 17 \to$ Keine Vielfaches
        \end{itemize}

        \begin{hinweise}
            \begin{itemize}
                \item Überlege dir vorab alle Fälle, die eintreffen können
                \item Teilbarkeit kann man mit \pythoninline{%} prüfen
            \end{itemize}
        \end{hinweise}

        \begin{solution}
            \begin{codeBlock}[]{minted language=python}
                print("Gebe zwei Zahlen ein")
                a = int(input())
                b = int(input())

                if a > b:
                    if a % b == 0:
                        print(f"{a} ist ein vielfachens von {b}")
                    else:
                        print("Kein Vielfaches")
                else:
                    if b % a == 0:
                        print(f"{b} ist ein vielfachens von {a}")
                    else:
                        print("Kein Vielfaches")
            \end{codeBlock}
        \end{solution}
    \end{subtask*}

    \begin{subtask*}{Vergleiche}
        \begin{enumerate}
            \item Lasse dir drei Zahlen eingeben
            \item Gebe aus, wie viele Werte gleich sind
        \end{enumerate}

        Beispiel:
        \begin{itemize}
            \item $1, 1, 1 \to$ Drei gleiche Werte
            \item $1, 2, 3 \to$ Alles verschieden
            \item $1, 2, 1 \to$ zwei gleiche Werte
        \end{itemize}

        \begin{hinweise}
            \begin{itemize}
                \item Prüfe mehrere Bedingung mit logischen Operatoren
                \item Benutze \pythoninline{elif} um überflüssiges
                    Verschachteln zu vermeiden
            \end{itemize}
        \end{hinweise}

        \begin{solution}
            \begin{codeBlock}[]{minted language=python}
                print("Gebe drei Zahlen ein")
                a = int(input())
                b = int(input())
                c = int(input())

                if a == b and b == c:
                    print("Alle drei Zahlen sind gleich")
                elif a == b or a == c or b == c:
                    print("Zwei von drei Zahlen sind gleich")
                else:
                    print("Alle Zahlen sind unterschiedlich")
            \end{codeBlock}
        \end{solution}
    \end{subtask*}
\end{task}

\end{document}
