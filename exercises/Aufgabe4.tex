\RequirePackage[minted_workaround,caption_workaround,boxarc]{algo-common}
\ExplSyntaxOn
% Solution environment check (enable if SOLUTION=1)
\sys_get_shell:nnN { kpsewhich ~ --var-value ~ SOLUTION } { } \l_solution_env_var_tl
\tl_trim_spaces:N \l_solution_env_var_tl
\tl_if_eq:NnT \l_solution_env_var_tl {1} {\PassOptionsToClass{solution=true}{tudaexercise}}
\ExplSyntaxOff
\documentclass[accentcolor=3c,colorbacktitle,12pt]{tudaexercise}
% \usepackage[T1]{fontenc}
%\usepackage[utf8]{inputenc}
\ifPDFTeX
    \RequirePackage[utf8]{inputenc}
\fi
%\usepackage[ngerman]{babel}
%Includes
\usepackage[ngerman]{babel} %Deutsche Silbentrennung
\usepackage[utf8]{inputenc} %Deutsche Umlaute
\usepackage{float}
\usepackage{graphicx}
\usepackage{minted}
\RequirePackage{csquotes}
\RequirePackage{fontawesome5}

\DeclareGraphicsExtensions{.pdf,.png,.jpg}

\makeatletter
\author{Vorkursteam der Fachschaft Informatik}
\let\Author\@author

% dark mode
\ExplSyntaxOn
\IfDarkModeT{
    \cs_if_exist:NT \setbeamercolor {
        \setbeamercolor*{smallrule}{bg=.}
        \setbeamercolor*{normal~text}{bg=\thepagecolor,fg=.}
        \setbeamercolor*{background~canvas}{parent=normal~text}
        \setbeamercolor*{section~in~toc}{parent=normal~text}
        \setbeamercolor*{subsection~in~toc}{parent=normal~text,fg=.}
        \setbeamercolor*{footline}{parent=normal~text}
        \setbeamercolor{block~title~alerted}{fg=white,bg=white!20!\thepagecolor}
        \setbeamercolor*{block~body}{bg=black!70!gray!98!blue}
        \setbeamercolor*{block~body~alerted}{bg=\thepagecolor}
    }
    \cs_if_exist:NT \setbeamertemplate {
        \setbeamertemplate{subsection~in~toc~shaded}[default][50]
    }
}
\ExplSyntaxOff

% macros
\renewcommand{\arraystretch}{1.2} % Höhe einer Tabellenspalte minimal erhöhen
\newcommand{\N}{{\mathbb N}}
\renewcommand{\code}{\inputminted[]{python}}

\IfDarkModeTF{
    \newmintedfile[pythonfile]{python}{
        fontsize=\small,
        style=native,
        linenos=true,
        numberblanklines=true,
        tabsize=4,
        obeytabs=false,
        breaklines=true,
        autogobble=true,
        encoding="utf8",
        showspaces=false,
        xleftmargin=20pt,
        frame=single,
        framesep=5pt,
    }
    \newmintinline{python}{
        style=native,
        encoding="utf8"
    }
    \newmintinline{kotlin}{
        style=native,
        encoding="utf8"
    }


    \definecolor{codegray}{HTML}{eaf1ff}
    \newminted[bashcode]{awk}{
        escapeinside=||,
        fontsize=\small,
        style=native,
        linenos=true,
        numberblanklines=true,
        tabsize=4,
        obeytabs=false,
        breaklines=true,
        autogobble=true,
        encoding="utf8",
        showspaces=false,
        xleftmargin=20pt,
        frame=single,
        framesep=5pt
    }
}{
    \newmintedfile[pythonfile]{python}{
        fontsize=\small,
        style=friendly,
        linenos=true,
        numberblanklines=true,
        tabsize=4,
        obeytabs=false,
        breaklines=true,
        autogobble=true,
        encoding="utf8",
        showspaces=false,
        xleftmargin=20pt,
        frame=single,
        framesep=5pt,
    }
    \newmintinline{python}{
        style=friendly,
        encoding="utf8"
    }
    \newmintinline{kotlin}{
        style=friendly,
        encoding="utf8"
    }

    \definecolor{codegray}{HTML}{eaf1ff}
    \newminted[bashcode]{awk}{
        escapeinside=||,
        fontsize=\small,
        style=friendly,
        linenos=true,
        numberblanklines=true,
        tabsize=4,
        obeytabs=false,
        breaklines=true,
        autogobble=true,
        encoding="utf8",
        showspaces=false,
        xleftmargin=20pt,
        frame=single,
        framesep=5pt
    }
}

\let\origpythonfile\pythonfile
\renewcommand{\pythonfile}[1]{\pythonfileh{#1}{}}
\newcommand{\pythonfileh}[2]{\origpythonfile[#2]{#1}}

\DeclareDocumentCommand{\kotlinfile}{O{} O{} m}{\inputCode[#1]{minted language=kotlin,#2}{#3}}

\newcommand*{\ditto}{\texttt{\char`\"}}

\newcommand{\shellprefix}{\textcolor{TUDa-3a}{\ttfamily\bfseries \$~}}
\DeclareTCBListing{commandshell}{ O{} O{} }{
    colback=\IfDarkModeTF{black}{black!80},
    colupper=white,
    colframe=TUDa-3a,
    listing only,
    % listing options={style=tcblatex,language=sh},
    listing engine=minted,
    minted style=dracula,
    minted options={
        % linenos=true,
        numbersep=3mm,
        texcl=true,
        autogobble,
        escapeinside=@@,
        breaklines,
        highlightcolor=yellow!50!black,
        #1
    },
    #2,
    % before upper={\textcolor{red}{\small\ttfamily\bfseries root \$> }},
    % every listing line={\textcolor{red}{\small\ttfamily\bfseries root \$> }}
}

\usepackage{hyperref}
\usepackage{ifthen}
\usepackage{listings}
%\usepackage{graphicx}
\usepackage{multicol}
\usepackage{multirow}
\usepackage{amssymb}
\RequirePackage{silence}
\WarningFilter[sillyfonterror]{latex}{Font~shape~declaration~has}
\ActivateWarningFilters[sillyfonterror]
\ifLuaTeX
    % Fix Font warnings for mathdesign
    \DeclareFontFamily{TU}{mdbch}{}
    \DeclareFontShape{TU}{mdbch}{m}{n}{
        <-> \UnicodeFontFile{lmroman10-regular}{\UnicodeFontTeXLigatures}
    }{}
    \RequirePackage[utf8]{luainputenc} % if problems with ß exist
\fi

\definecolor{darkblue}{rgb}{0,0,.5}
\hypersetup{colorlinks=true, breaklinks=true, linkcolor=\IfDarkModeTF{cyan}{darkblue}, menucolor=darkblue, urlcolor=\IfDarkModeTF{cyan}{darkblue}}

% configuration
\makeatletter
\@ifundefined{c@ex}{
    \newcounter{ex}\setcounter{ex}{1}
}{}
\makeatother
\IfDarkModeT{
    \selectcolormodel{RGB}
}
\newboolean{sln}\setboolean{sln}{false}
\newboolean{SoSe}\setboolean{SoSe}{false}
\newcommand{\thenextyear}{\the\numexpr\year+1\relax}
%

\newcommand{\ext}{py}
\newcommand{\sln}[1]{\ifthenelse{\boolean{sln}}{\subsubsection*{Antwort}{\itshape #1}}{}}
\newcommand{\slnformat}[1]{\ifthenelse{\boolean{sln}}{#1}{}}
\newcommand{\vorkurstaskformat}[1]{\ifthenelse{\boolean{sln}}{}{#1}}
\newcommand{\vorkurstask}[1]{\input{task/#1}\IfFileExists{./sln/#1.tex}{\sln{\input{sln/#1}}}{\IfFileExists{./sln/#1.\ext}{\sln{\pythonfile{sln/#1.\ext}}}{\ClassError{Vorkurs-TeX}{No solution specified for task #1}{Add solution file #1.tex or #1.\ext}}}}
\newcommand{\mccmd}{Kreuze zu jeder Antwort an, ob sie zutrifft (\textbf{w}) oder nicht (\textbf{f}).}
\newcommand{\mchead}{\item[\textbf{w} \textbf{f} ]}
\newcommand{\mcitem}[1]{\item[$\square\ \square$] #1}
\newcommand{\mcitemt}[1]{\item[$\ifthenelse{\boolean{sln}}{\blacksquare}{\square}\ \square$] #1}
\newcommand{\mcitemf}[1]{\item[$\square\ \ifthenelse{\boolean{sln}}{\blacksquare}{\square}$] #1}
\newcommand{\ptitle}{\ifthenelse{\boolean{SoSe}}{Sommersemester \the\year}{Wintersemester \the\year/\thenextyear}}

\newcommand{\lstinlinenoit}[1]{\upshape{\lstinline|#1|}\itshape}
\lstset{language=Python, basicstyle=\ttfamily\small, keywordstyle=\color{\IfDarkModeTF{cyan!60!black}{blue!80!black}}, identifierstyle=, commentstyle=\color{green!50!.}, stringstyle=\ttfamily,
    tabsize=4, breaklines=true, numbers=left, numberstyle=\small, frame=single, backgroundcolor=\color{blue!3!\thepagecolor}}
\author{Fachschaft Informatik}

\newcommand{\stage}[1]{(\ifcase#1\or{Einstieg}\or{Vertiefung}\or{Herausforderung}\else\fi)}
\newcommand{\bonus}[1]{\textit{BonusFact: }#1}

\sheetnumber{4}
\title{Aufgaben Programmiervorkurs}
\subtitle{von der Fachschaft Informatik\hfill\ptitle}
\begin{document}
\maketitle{}

\begin{task}[points=auto]{Funktionen}
    \begin{subtask*}[points=0]{Taschenrechner Reloaded \stage1}
        In der vorigen Übung sollte ein simpler Taschenrechner programmiert werden. Nun wollen wir diesen Taschenrechner erweitern. Schreibt zuerst jeweils eine Funktion für jede der angebotenen Grundrechenarten.

        \pythoninline{def plus(a: int, b: int) -> int:} soll beispielsweise \pythoninline{plus(3, 5) == 8} liefern.

        Im Anschluss sollen die Rechenanweisungen in der if-Struktur durch Funktionsaufrufe ersetzt werden. Vergesst nicht, das Ergebnis auszugeben.

        \begin{solution}
            siehe nächste Teilaufgabe
        \end{solution}
    \end{subtask*}
    \begin{subtask*}[points=0]{Taschenrechner Reloaded II \stage1}
        Erweitert den Taschenrechner nun noch um einen weiteren Operator. Dabei soll auch das Potenzieren möglich sein. Dabei gilt $a^n=a\cdot a\cdot a \cdot\ldots\cdot a$ (d.h. a wird n-mal mit sich selbst miltipliziert). Zum Aufruf der Potenzier-Funktion soll der Benutzer \^{} (das kleine "`Dach"' links oben auf der Tastatur) eingeben.

        \begin{solution}
            \begin{codeBlock}[]{minted language=python}
                def plus(op1: float, op2: float) -> float:
                    return op1 + op2

                def minus(op1: float, op2: float) -> float:
                    return op1 - op2

                def mal(op1: float, op2: float) -> float:
                    return op1 * op2

                def teilen(op1: float, op2: float) -> float:
                    return op1 / op2

                def potenz(op1: float, op2: float) -> float:
                    return op1 ** op2

                op1 : float = float(input("Wie lautet der erste Operand: "))
                op2 : float = float(input("Wie lautet der zweite Operand: "))

                op = input("Wie lautet der Operator: ")

                if op == "+":
                    print(f"Das Ergebnis lautet: {plus(op1, op2)}")
                elif (op == "-"):
                    print(f"Das Ergebnis lautet: {minus(op1, op2)}")
                elif op == "*":
                    print(f"Das Ergebnis lautet: {mal(op1, op2)}")
                elif op == "/":
                    if op2 != 0.0:
                        print(f"Das Ergebnis lautet: {teilen(op1, op2)}")
                    else:
                        print("Division durch Null ist nicht erlaubt.")
                elif op == "^":
                    print(f"Das Ergebnis lautet: {potenz(op1, op2)}")
                else:
                    print("Keine gültige Eingabe!")
            \end{codeBlock}
        \end{solution}
    \end{subtask*}
\end{task}
\begin{task}[points=auto]{Schriftliches Dividieren}
    \begin{subtask*}[points=0]{Schriftliches Dividieren - Theorie \stage2}
        Gleitkommaberechnungen sind in Python, sowie in den meisten anderen Programmiersprachen, nicht exakt. Könnt ihr euch denken, wieso?\\
        Ihr könnt das Problem an dem folgenden Beispiel betrachten.
        \begin{codeBlock}[]{minted language=python}
            a : float = 5.6
            b : float = 5.8
            c : float = a + b
            print(f"{a} + {b} = {c}")
        \end{codeBlock}

        \begin{solution}
            Bei Gleitkommazahlen werden oft nicht die angegebene Zahlen gespeichert sondern eine Approximation, deutsch \enquote{Annäherung}. Das hängt unter anderem mit der Art der Speicherung einer Gleitkommazahl zusammen. Während im Dezimalsystem eine 0,1 ganz einfach darzustellen ist, arbeitet der Rechner im Binärsystem (also auf Zweierbasis) und im Binärsystem ist die Dezimalzahl 0,1 periodisch ($0,1_2=0,00011001100110011 \dots$) Da der Speicher einer Gleitkommazahl beschränkt ist, schneidet der Rechner nach einer bestimmten Länge einfach ab.
            Das ist nicht neues für uns. Schon in der Schule hat man ohne
            große Bedenken gerundet. Etwa bei $\frac13 \approx 0.3333$.

            Eine andere Fehlerquelle birgt die Verrechnung - hier beispielhaft die Addition - von sehr großen mit sehr kleinen Zahlen. Um ein konkretes Beispiel zu geben: Wenn man die beiden floats 100.000.000 und 0,000000001 addiert erhält man als Ergebnis 100.000.000. Auch hier reicht die Art der Speicherung für float-Zahlen nicht aus, um eine 100.000.000,000000001 zu speichern. Die genaue Erklärung ersparen wir uns hier, da sie zu technisch wäre.
        \end{solution}
    \end{subtask*}
    \begin{subtask*}[points=0]{Schriftliches Dividieren \stage2}
        Um das Problem der ungenauen Gleitkommadivision zumindest zum Teil zu lösen, ist eure Aufgabe, eine Funktion zu schreiben, die zwei (positive) Ganzzahlen dividiert und das Ergebnis als String zurück gibt. Das Ergebnis soll anschließend (in der main-Funktion) auf der Konsole ausgegeben werden. Dabei wollen wir uns das Vorgehen des schriftliches Dividieren zunutze machen. Den ganzzahligen Anteil des Ergebnisses könnt ihr mit der Ganzzahldivision berechnen, den Rest mit dem Modulo-Operator.

        Wenn die Division nicht aufgeht, also eine periodische Zahl liefern würde (zum Beispiel $\frac13$), braucht euer Programm nicht zu terminieren, es muss also kein Ergebnis liefern und darf endlos vor sich hin rechnen.
        Ihr dürft aber auch nach jedem Schritt ein Teil des Ergebnisses ausgeben,
        im obigen Beispiel also endlos viele Dreien.

        Schafft ihr es (außer bei Nutzereingaben natürlich), ohne die Funktionen
        \pythoninline{str} und \pythoninline{int} auszukommen?

        \begin{hinweis}
            Die Ganzzahldivision in Python wird mit \pythoninline{//} durchgeführt. Beispiel:
            \begin{codeBlock}[]{minted language=python}
                5 // 2 == 2
            \end{codeBlock}
            Mit \enquote{\%} bekommt ihr den Rest der Division $\frac{a}{b}$.
        \end{hinweis}

        \begin{solution}
            \begin{codeBlock}[]{minted language=python}
                def longdivide(a: int, b: int):
                    div = a // b
                    rest = a % b
                    result=div
                    if rest != 0:
                        result+=","
                        while rest != 0:
                            div = (rest * 10) // b
                            rest = (rest * 10) % b
                            result+=div
                    return result


                z1 = int(input("Geben Sie die erste Zahl ein:"))
                z2 = int(input("Geben Sie die zweite Zahl ein:"))
                print(longdivide(z1, z2))
            \end{codeBlock}
        \end{solution}
    \end{subtask*}
    \begin{subtask*}[points=0]{Schriftliches Dividieren Reloaded \stage3}
        Da viele Divisionen nicht aufgehen, wollen wir die definierte Funktion dahingehend überarbeiten, damit wir im Anschluss jedes (positive) Ganzzahlenpaar dividieren können.
        Dafür müsst ihr zuerst überlegen, woran periodische Ergebnisse während der Division erkannt werden können und eine entsprechende Detektion in eure Funktion einbauen.
        Außerdem soll euer Programm ein periodisches Ergebnis auch korrekt ausgeben. Es gibt international verschiedene Schreibweisen für periodische Zahlen, am einfachsten für uns ist, wenn wir die wiederkehrenden Ziffern einfach einklammern:
        \begin{codeBlock}[]{minted language=python}
            1/3 = 0,(3)
            40/3 = 13,(3)
            17/130 = 0,1(307692)
            35/24 = 1,458(3)
        \end{codeBlock}

        \begin{solution}
            \begin{codeBlock}[]{minted language=python}
                def longdivide(a: int, b: int):
                    div = a / b
                    rem = a \% b

                    seenRemainders = [rem]
                    afterKomma = ""

                    print(div)
                    if rem == 0:
                        return

                    print(",")

                    while rem != 0:
                        div = rem * 10

                        rem = div \% b
                        div = div / b
                        afterKomma += div

                        index = seenRemainders.index(rem)
                        if index >= 0:
                            print(afterKomma[0:index+1])

                            print("(")
                            print(afterKomma[index:len(afterKomma)])
                            print(")")

                            return

                        seenRemainders.append(rem)

                    print(afterKomma)

                longdivide(9, 2) // = 4,5
                longdivide(1, 3) // = 0,(3)
                longdivide(40, 3) // = 13,(3)
                longdivide(17, 130) // = 0,1(307692)
                longdivide(35, 24) // = 1,458(3)
            \end{codeBlock}
        \end{solution}
    \end{subtask*}
\end{task}
\begin{task}[points=auto]{Zusatz}
    Wenn du die Aufgaben erledigt hast, kannst du dich gerne noch an den Schleifen-Varianten der gegebenen Algorithmen versuchen oder für Aufgaben der vergangenen Tage eine rekursive Lösung suchen. Die Tutoren stehen dir dabei gerne bei, allerdings werden wir dafür keine Lösungen bereit stellen. Du kannst die jeweiligen Versionen der Algorithmen in den Lösungen als Kontrolle für deine Versuche nutzen.
\end{task}
\begin{task}[points=auto]{Challenge}
    \large \textbf{Hinweis:} Nun ist Zeit für die Programmierchallenge! Viel Erfolg! :-)
\end{task}
\end{document}
