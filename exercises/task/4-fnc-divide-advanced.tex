\subsection{Schriftliches Dividieren Reloaded \stage3}
Da viele Divisionen nicht aufgehen, wollen wir die definierte Funktion dahingehend überarbeiten, damit wir im Anschluss jedes (positive) Ganzzahlenpaar dividieren können. Dafür müsst ihr zuerst überlegen, woran periodische Ergebnisse während der Division erkannt werden können und eine entsprechende Detektion in eure Funktion einbauen. Außerdem soll euer Programm ein periodisches Ergebnis auch korrekt ausgeben. Es gibt international verschiedene Schreibweisen für periodische Zahlen, am einfachsten für uns ist, wenn wir die wiederkehrenden Ziffern einfach einklammern:
\begin{lstlisting}
1/3 = 0,(3)
40/3 = 13,(3)
17/130 = 0,1(307692)
35/24 = 1,458(3)
\end{lstlisting}
