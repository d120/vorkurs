\subsection{Definition und Indices \stage1}

Beginnt ein neues Programm, und geht dabei davon aus, dass die Liste
\pythoninline{list} genau $8$ Elemente enthält. Euer Programm soll
nun die Ergebnisse der folgenden Rechnungen ausgeben:

\begin{enumerate}
    \item Die Summe der Zahlen an Index $1$, $2$ und $7$.
    \item Das Produkt jeder zweiten Zahl.
    \item Die erste Zahl mal 300.
\end{enumerate}

Für diese Aufgabe müsst ihr weder Schleifen noch Slicing verwenden. Probiert
euer Programm aus, indem ihr die folgenden Listen einsetzt und die Ausgaben
mit denen von eurem Programm vergleicht.

\begin{itemize}
    \item Für \pythoninline{liste = [1,1,2,3,5,8,13,21]} ist das $24$, $504$ und $300$
    \item Für \pythoninline{liste = [-1,1,-1,1,-1,1,-1,1]} ist das $1$, $1$ und $-300$
\end{itemize}
