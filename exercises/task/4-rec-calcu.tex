\subsection{Ultimate Taschenrechner \stage3}
Wir wollen den Taschenrechner nun auf Rekursion umstellen. Du kannst zur Vereinfachung
davon ausgehen, dass beide Operanden positiv und ganze Zahlen sind (für Testeingaben
musst du dich selbst dann aber auch daran halten).
\\Im Folgenden werden an die verschiedenen Operatoren Bedingungen gestellt (die
Funktionsnamen sind so gewählt wie in der Musterlösung von Tag 3, wenn deine
Addierfunktion beispielsweise "`add"' heißt, musst du das \texttt{plus} in der Aufgabenstellung duch \texttt{add} ersetzen):
\begin{itemize}
    \item \textbf{\textcolor[rgb]{0,0.5,1}{plus:}} verwende "`+1"', "`-1"' und plus
    \item \textbf{\textcolor[rgb]{0,0.5,1}{minus:}} verwende "`+1"', "`-1"' und minus
    \item \textbf{\textcolor[rgb]{0,0.5,1}{mal:}} verwende plus, minus und mal
    \item \textbf{\textcolor[rgb]{0,0.5,1}{teilen:}} verwende plus, minus und teilen (schreibe nur die Ganzzahldivision)
    \item \textbf{\textcolor[rgb]{0,0.5,1}{potenz:}} verwende plus, minus, mal, teilen und potenz
\end{itemize}
Du darfst nur die genannten Funktionen (wenn plus erlaubt ist, darfst du "`plus(x,y)"' aufrufen),
konstante Zahlen, die Variablen zahl1 und zahl2 (also die Parameter) und das Prinzip
der Rekursion verwenden. Du darfst keine Schleifen und keine arithmetischen Operatoren
(+, -, *, /, ... ; nur +1 und -1) verwenden. Du darf die if-Anweisung verwenden.
Wenn du einen Operator nicht umsetzen kannst, dann versuch dich an einem anderen.
Du kannst beispielsweise die Multiplikation rekursiv umschreiben, ohne eine rekursive
Addition zu schreiben; nutze dann einfach die bestehende Additionsfunktion.
