\subsection{Ultimate Taschenrechner \stage3}
Wir wollen den Taschenrechner nun auf Rekursion umstellen. Ihr könnt zur Vereinfachung
davon ausgehen, dass beide Operanden positiv und ganze Zahlen sind (für Testeingaben
müsst ihr euch selbst dann aber auch daran halten).\\
Im Folgenden werden an die verschiedenen Operatoren Bedingungen gestellt (die
Funktionsnamen sind so gewählt wie in der Musterlösung von Tag 3, wenn eure
Addierfunktion beispielsweise "`add"' heißt, müsst ihr das \texttt{plus} in der Aufgabenstellung duch \texttt{add} ersetzen):
\begin{itemize}
    \item \textbf{\textcolor[rgb]{0,0.5,1}{plus:}} verwendet "`+1"', "`-1"' und plus
    \item \textbf{\textcolor[rgb]{0,0.5,1}{minus:}} verwendet "`+1"', "`-1"' und minus
    \item \textbf{\textcolor[rgb]{0,0.5,1}{mal:}} verwendet plus, minus und mal
    \item \textbf{\textcolor[rgb]{0,0.5,1}{teilen:}} verwendet plus, minus und teilen (schreibe nur die Ganzzahldivision)
    \item \textbf{\textcolor[rgb]{0,0.5,1}{potenz:}} verwendet plus, minus, mal, teilen und potenz
\end{itemize}
Ihr dürft nur die genannten Funktionen (wenn plus erlaubt ist, dürft ihr "`plus(x,y)"' aufrufen),
konstante Zahlen, die Variablen zahl1 und zahl2 (also die Parameter) und das Prinzip
der Rekursion verwenden. Ihr dürft keine Schleifen und keine arithmetischen Operatoren
(+, -, *, /, ... ; nur +1 und -1) verwenden. Ihr dürft die if-Anweisung verwenden.

\textit{Hinweis: Wenn ihr einen Operator nicht umsetzen könnt, dann versucht euch an einem anderen.
Ihr könnt beispielsweise die Multiplikation rekursiv umschreiben, ohne eine rekursive
Addition zu schreiben; nutzt dann einfach die bestehende Additionsfunktion.}
