\subsection{Entscheidungsbaum}
Wenn große Mengen von Daten verarbeitet werden müssen, geht es oft darum, diese in bestimmte Gruppen bzw. Klassen einzuteilen, zum Beispiel sollen häufig fehlerhafte Daten von korrekten getrennt werden. Eine mögliche Klassifikationsmethode, die für Tabellendaten genutzt werden kann, ist ein Entscheidungsbaum.\\
\\
Entscheidungsbäume funktionieren so, dass man für einen Datenpunkt (also eine Tabellenzeile) nach und nach die Spalteneinträge überprüft, bis man an einem Punkt ankommt, wo man sicher die Klasse festlegen kann. Das kann je nach dem, in welcher Reihenfolge man die Spalten betrachtet, unterschiedlich kompliziert sein (manchmal reicht sogar eine einzelne Spalte, um sicher zu wissen, welche Klasse es ist).\\
\\
Deine Aufgabe ist es, mit den Daten in der folgenden Tabelle einen Entscheidungsbaum zu konstruieren, der dir ausgibt, wann es sich lohnt, ins Freibad zu gehen (so, wie es in der Tabellenspalte ganz rechts festgelegt ist). Lege dazu am Besten pro Spalte eine Variable an, wo du dann Zeile für Zeile die Werte nacheinander durchprobieren kannst, um deine Implementierung zu testen.
\begin{table}[ht]
    \centering
    \begin{tabular}{|c|c|c|c|c|}
        \hline
        Wettervorhersage & Temperatur & Luftfeuchtigkeit & Wind  & Freibad \\\hline
        sonnig           & 29         & hoch             & False & Ja      \\\hline
        sonnig           & 29         & mittel           & True  & Ja      \\\hline
        regen            & 13         & mittel           & True  & Nein    \\\hline
        bewölkt          & 22         & mittel           & False & Ja      \\\hline
        regen            & 13         & hoch             & False & Nein    \\\hline
        bewölkt          & 29         & mittel           & False & Ja      \\\hline
        sonnig           & 14         & mittel           & True  & Nein    \\\hline
        bewölkt          & 27         & hoch             & True  & Ja      \\\hline
        bewölkt          & 14         & hoch             & True  & Nein    \\\hline
        sonnig           & 20         & mittel           & False & Ja      \\\hline
        regen            & 27         & mittel           & False & Nein    \\\hline
        bewölkt          & 14         & mittel           & False & Nein    \\\hline
        regen            & 20         & mittel           & True  & Nein    \\\hline
        sonnig           & 22         & hoch             & True  & Ja      \\\hline
        sonnig           & 14         & hoch             & False & Ja      \\\hline
        bewölkt          & 22         & mittel           & True  & Nein    \\\hline
        regen            & 29         & mittel           & True  & Nein    \\\hline
        regen            & 22         & hoch             & False & Nein    \\\hline
    \end{tabular}
\end{table}
\newpage
Wenn du überprüfen möchtest, ob dein Entscheidungsbaum gut funktioniert, kannst du auch noch diese Datenpunkte ausprobieren:
\begin{table}[ht]
    \centering
    \begin{tabular}{|c|c|c|c|c|}
        \hline
        Wettervorhersage & Temperatur & Luftfeuchtigkeit & Wind  & Freibad \\\hline
        bewölkt          & 15         & hoch             & False & Nein    \\\hline
        sonnig           & 28         & mittel           & False & Ja      \\\hline
        sonnig           & 21         & hoch             & False & Ja      \\\hline
        regen            & 27         & hoch             & False & Nein    \\\hline
        regen            & 20         & mittel           & False & Nein    \\\hline
        bewölkt          & 28         & hoch             & False & Ja      \\\hline
        sonnig           & 13         & hoch             & True  & Nein    \\\hline
        bewölkt          & 21         & hoch             & False & Ja      \\\hline
        regen            & 27         & hoch             & True  & Nein    \\\hline
    \end{tabular}
\end{table}

\textit{Hinweis: Euer Programm soll am Ende so funktionieren, dass ihr die Eckdaten (also Wettervorhersage, Temperatur, Luftfeuchtigkeit und Wind) eingebt, und das Programm dann Antwortet, ob ihr ins Freibad gehen solltet oder nicht.}