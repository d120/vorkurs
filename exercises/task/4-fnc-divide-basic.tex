\subsection{Schriftliches Dividieren \stage2}
Um das Problem der ungenauen Gleitkommadivision zumindest zum Teil zu lösen, ist eure Aufgabe, eine Funktion zu schreiben, die zwei (positive) Ganzzahlen dividiert und das Ergebnis als String zurück gibt. Das Ergebnis soll anschließend (in der main-Funktion) auf der Konsole ausgegeben werden. Dabei wollen wir uns das Vorgehen des schriftliches Dividierens zunutze machen. Den ganzzahligen Anteil des Ergebnisses dürft ihr direkt berechnen und braucht das Vorgehen erst zur Berechnung der Nachkommastellen anwenden. Wenn die Division nicht aufgeht, also eine periodische Zahl liefern würde (zum Beispiel 1/3), braucht euer Programm nicht zu terminieren, es muss also kein Ergebnis liefern und darf endlos vor sich hin rechnen.

\textit{Hinweis: Ganzzahldivision könnt ihr mit // durchführen}
