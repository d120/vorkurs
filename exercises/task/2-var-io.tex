\subsection{Variablen einlesen}
Im Rahmen der Vorlesung habt ihr die Funktion \texttt{input} kennengelernt, welche es euch erlaubt, auf einfache Art und Weise Benutzereingaben von der Konsole einzulesen. Schreibt jetzt ein Programm, welches mit Hilfe dieser Funktion nacheinander folgende Eingaben einliest und in Variablen speichert.
\begin{itemize}
    \item euren Namen als Zeichenkette
    \item euer Geburtsdatum als drei ganze Zahlen: Tag, Monat, Jahr
\end{itemize}
Insgesamt sollen also vier Eingaben verarbeiten werden. Achtet dabei darauf, dass euer Programm die erwarteten Eingaben textuell klar strukturiert (Stichwort: Bedienerfreundlichkeit).\\
Nachdem die letzte Eingabe verarbeitet ist, soll folgende Zeichenkette ausgegeben werden (wobei natürlich die Platzhalter zu ersetzen sind):\\
Ausgabe: {\ttfamily{\dq}}<euer Name> hat am <tt.mm.jjjj> Geburtstag.{\ttfamily{\dq}}\\
\\
\textit{Hinweis: Die Verarbeitung von Eingaben wird im Foliensatz 'Variablen.pdf' erläutert.}\\
\\
\textit{Hinweis: Variablen sollten in Strings nicht mit '\pythoninline{+}' eingefügt werden, stattdessen vor den String ein '\pythoninline{f}' setzen und an die entsprechenden Stellen im String \pythoninline{{variable0}}, \pythoninline{{variable1}}, ... einsetzen. \\
Beispiel: \pythoninline{f"{variable0} ist kleiner {variable1}"} ergibt wenn variable0 = 4 und variable1 = 42 {\ttfamily{\dq}}4 ist kleiner 42{\ttfamily{\dq}}\\
