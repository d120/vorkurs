\subsection{Variablen einlesen}
Im Rahmen der Vorlesung hast du die Funktion \texttt{input} kennengelernt, welche es dir erlaubt, auf einfache Art und Weise Benutzereingaben von der Konsole einzulesen. Schreibe jetzt ein Programm, welches mit Hilfe dieser Funktion nacheinander folgende Eingaben einließt und in Variablen speichert.
\begin{itemize}
    \item deinen Namen als Zeichenkette
    \item dein Geburtsdatum als drei ganze Zahlen: Tag, Monat, Jahr
\end{itemize}
Insgesamt sollen also vier Eingaben verarbeiten werden. Achte dabei darauf, dass dein Programm die erwarteten Eingaben textuell klar strukturiert (Stichwort: Bedienerfreundlichkeit).\\
Nachdem die letzte Eingabe verarbeitet ist, soll folgende Zeichenkette ausgegeben werden (wobei natürlich die Platzhalter zu ersetzen sind):\\
Ausgabe: {\ttfamily{\dq}}<dein Name> hat am <tt.mm.jjjj> Geburtstag.{\ttfamily{\dq}}\\
\\
\textit{Hinweis:} Die Verarbeitung von Eingaben wird im Foliensatz 'Variablen.pdf' erläutert.\\
\\
\textit{Hinweis:} Variablen sollten in Strings nicht mit '\pythoninline{+}' eingefügt werden, stattdessen vor den String ein '\pythoninline{f}' setzen und an die entsprechenden Stellen im String \pythoninline{{variable0}}, \pythoninline{{variable1}}, ... einsetzen. \\
Beispiel: \pythoninline{f"{variable0} ist kleiner {variable1}"} ergibt wenn variable0 = 4 und variable1 = 42 {\ttfamily{\dq}}4 ist kleiner 42{\ttfamily{\dq}}\\
Alternativ kannst du die ältere format Methode verwenden. Dazu setzt du an die entsprechenden Stellen im String \pythoninline{{0}}, \pythoninline{{1}}, ... ein und trägst dahinter \pythoninline{.format(variable0, variable1, ...)} ein.\\
Beispiel: \pythoninline{"{0} ist kleiner {1}".format(4, 42)} ergibt ebenfalls {\ttfamily{\dq}}4 ist kleiner 42{\ttfamily{\dq}}
