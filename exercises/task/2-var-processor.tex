\subsection{Ein einfacher Prozessor}
Prozessoren bestehen im allgemeinen aus einer kleinen Anzahl von Registern, mit denen die eigentlichen Berechnungen durchgeführt werden. Im Folgenden sollt ihr das Verhalten eines kleinen Prozessors in Python simulieren. \\
Der Prozessor soll aus 4 Registern bestehen, die beliebige Werte speichern können (Diese könnt ihr als Variablen modellieren). Außerdem kann der Prozessor grundlegende Berechnungen (+, -, *, /, \%) auf seinen Registern durchführen und Werte in Registern vergleichen (\textbf{Achtung:} Ihr müsst also nach jeder einzelnen Rechung oder einem Vergleich das Ergebnis in eines der Register speichern, und dürft nicht mehrere Rechnungen nacheinander ausführen (Also statt \pythoninline{a = 2 + 3 * 4} müsstet ihr \pythoninline{b = 3 * 4} gefolgt von \pythoninline{a = 2 + b} schreiben. If-Abfragen sind auf diese Art möglich: \pythoninline{if a:}, also kein \pythoninline{else:} und ihr müsst den Wahrheitswert, den ihr mit if überpfürt, zuerst in ein Register schreiben.)) \\
In den Berechnungen dürf ihr außer den Registern auch Zahlenwerte verwenden, jedoch nicht das Register in das das Ergebnis geschrieben wird.\\
Erstellt nun mit diesen Randbedinungen ein Programm, dass das folgende tut: \\
\begin{lstlisting}
lese eine Ganzzahl a ein
lese eine Ganzzahl b ein
lese eine Ganzzahl c ein
verdoppele a und addiere 3 hinzu
lese eine Ganzzahl d ein
gib b und c aus
wenn a > d gilt, setze d = a
wenn a < d gilt, speichere Wert c = d - a, sonst speichere Wert c = 0
setze a = c * d + a * a
setze Wert b = a + d
setze c = b + d
gib a,b,c und d aus
\end{lstlisting}
