\subsection{Ultimate Taschenrechner II \stage3}
Erweitert den Taschenrechner im selben Stil noch um weitere Funktionen (in Klammern ist angegeben, wie die Funktion aufgerufen werden sollte):
\begin{itemize}
    \item \textbf{\textcolor[rgb]{0,0.5,1}{modulo (\%):}} verwende plus, minus und modulo
    \item \textbf{\textcolor[rgb]{0,0.5,1}{ggT (T):}} verwende plus, minus, modulo und ggT
    \item \textbf{\textcolor[rgb]{0,0.5,1}{min ( \underline{\ \ } ):}} verwende plus, minus und min
    \item \textbf{\textcolor[rgb]{0,0.5,1}{max (|):}} verwende plus, minus und max
\end{itemize}
Für den ggT (größten gemeinsamen Teiler) gilt:
$$ggT(a,b)=\left\{\begin{tabular}{cc}b&wenn a=0,\\a&wenn b=0,\\ggt(b, a \% b)&sonst\end{tabular}\right.$$
