\subsection{Python Installieren}
Hey, willkommen bei den Übungen zum Programmiervorkurs im \ptitle. Wie in der
Vorlesung verwenden wir für die Übungen die Programmiersprache Python. \\\\
%
Auf den Poolrechnern und auf den meisten Linux-Distributionen sollte die
benötigte Version 3 schon installiert sein. Prüft das, indem ihr (etwa mit
Rechtsklick -> \textit{Terminal öffnen}) auf dem Desktop) ein Terminal öffnet,
und führt den Befehl \texttt{python ----version} aus. Dieser sollte so etwas wie
\textit{Python 3.7.9} ausgeben. Wenn ihr stattdessen \textit{Python 2.x.x} (x = irgendwelche Zahlen) bekommt, versucht es mit \texttt{python3} statt \texttt{python}, wenn das klappt, ersetzt im Folgenden einfach immer \texttt{python} durch \texttt{python3}.

Sollte das nicht funktionieren, installiert Python 3 über einen Paketmanager
eurer Wahl (etwa mit \texttt{apt install python}). \\\\
%
Für Windows oder MacOS müsst ihr Python von der offiziellen Seite
\url{https://python.org/downloads} herunterladen. Klickt dort einfach die
Schaltfläche \textit{Download Python 3.10.7} (höhere Versionsnummer möglich) an.
Öffnet nun auch hier ein Terminal (Auf Windows die Windows-Taste drücken und \texttt{PowerShell} eingeben und das Programm auswählen.). \\\\
%
Führt nun den Befehl \texttt{python} oder \texttt{python3} aus um eine
interaktive Eingabezeile zu starten.

Wenn ihr bei der Installation oder den Aufgaben nicht weiterkommt, fragt uns einfach! Es mag vielleicht nach einer guten Idee aussehen, im Internet nach der Antwort zu suchen, aber da habt ihr dann keine Garantie, dass es eine sinnvolle Erklärung dazu gibt. Außerdem wisst ihr dann nicht, ob ihr vielleicht etwas nicht richtig verstanden habt. Zu guter Letzt gibt es für viele der Probleme, die wir euch hier Präsentieren, sehr schnelle Lösungen in Python, welche aber höhere Programmierkenntnisse voraussetzen. Ziel ist es nicht, möglichst alle Aufgaben richtig zu beantworten, sondern dass ihr versteht, was ihr dabei tun müsst.
