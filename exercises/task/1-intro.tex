\subsection{Python Installieren}
Hey, willkommen bei den Übungen zum Programmiervorkurs im \ptitle. Wie in der
Vorlesung verwenden wir für die Übungen die Programmiersprache Python. \\\\
%
Auf den Poolrechnern und auf den meisten Linux-Distributionen sollte die
benötigte Version 3 schon installiert sein. Prüft das, indem ihr (etwa mit
Rechtsklick -> \textit{Terminal öffnen}) auf dem Desktop) ein Terminal öffnet,
und führt den Befehl \texttt{python3 ----version} aus. Dieser sollte so etwas wie
\textit{Python 3.5.3} ausgeben.

Sollte das nicht funktionieren, installiert Python 3 über einen Paketmanager
eurer Wahl (etwa mit \texttt{apt install python}). \\\\
%
Für Windows oder MacOS müsst ihr Python von der offiziellen Seite
\url{https://python.org/downloads} herunterladen. Klickt dort einfach die
Schaltfläche \textit{Download Python 3.7.0} (höhere Versionsnummer möglich) an.
Öffnet nun auch hier ein Terminal (Win+R -> \texttt{cmd}). \\\\
%
Führt nun den Befehl \texttt{python} oder \texttt{python3} aus um eine
interaktive Eingabezeile zu starten.
