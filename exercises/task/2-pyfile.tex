Für die folgenden Aufgaben solltet ihr jeweils eine Python-Programmdatei
anlegen. Öffnet dazu einen beliebigen Ordner in eurem Dateiexplorer. Öffnet
dann ein Terminal im selben Verzeichnis.

\begin{itemize}
    \item Unter Linux \& Mac (u.a. den Poolrechnern): Rechtsklick -> In Terminal öffnen
    \item Unter Windows: Shift+Rechtsklick -> Befehlseingabe hier öffnen
    \item Oder auch: Shift+Rechtsklick -> Powershell hier öffnen
\end{itemize}

Öffnet dann einen Texteditor eurer Wahl. Unter Linux ist das etwa gedit, unter
Windows kann Notepad verwendet werden. Erstellt eine Datei mit eurem Programm
und speichert sie mit der Endung \texttt{.py}, beispielsweise \texttt{datei.py}
in dem vorher ausgewählten Ordner.

Du kannst diese Datei dann starten, indem du \texttt{python3 datei.py} (Linux, Mac)
bzw. \texttt{python datei.py} (Windows) in dem geöffneten Terminal ausführst.
