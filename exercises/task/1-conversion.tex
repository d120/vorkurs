\subsection{Konvertierung}

Nehmen wir mal an, ihr habt den Text \texttt{"1000"} zur Verfügung. Vielleicht
errechnet; vielleicht aus einer API (\textit{application programming interface}, allgemeine Bezeichnung für Schnittstellen zwischen Programmen)? Es kommt immer wieder vor, dass die Daten noch den falschen Typ haben, wenn ihr sie bekommt. In Python schreibt ihr dafür
den Typ hin, den ihr erhalten wollt, sowie den Ausdruck in Klammern dahinter.
Python versucht, den am Besten passenden Wert des gewünschten Typs zu finden.

Gebt die folgenden Ausdrücke ein, und findet das Ergebnis heraus.

\textit{Hinweis: es kann sein, dass einige Konvertierungen fehlschlagen. In diesem Fall überlegt euch, warum. Die Fehlermeldung kann dabei hilfreich sein.}

\begin{multicols}{3}
    \begin{enumerate}
        \item \pythoninline{int("100")}
        \item \pythoninline{float("1.3")}
        \item \pythoninline{float("NaN")}
        \item \pythoninline{str(234)}
        \item \pythoninline{int("0x11")}
        \item \pythoninline{str(2) + '5'}
        \item \pythoninline{str(2 + 5)}
        \item \pythoninline{float(5)}
        \item \pythoninline{int(5.5)}
        \item \pythoninline{int("4.2")}
    \end{enumerate}
\end{multicols}
