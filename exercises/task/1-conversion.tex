\subsection{Konvertierung}

Nehmen wir mal an, du hast den Text \texttt{"1000"} zur Verfügung. Vielleicht
errechnet; vielleicht aus einer API (\textit{application programming interface}, allgemeine Bezeichnung für Schnittstellen zwischen Programmen)? Es kommt immer wieder vor, dass die Daten
noch den falschen Typ haben, wenn du sie bekommst. In Python schreibst du dafür
den Typ hin, den du erhalten willst, sowie den Ausdruck in Klammern dahinter.
Python versucht, den am Besten passenden Wert des gewünschten Typs zu finden.

Gib die folgenden Ausdrücke ein, und finde das Ergebnis heraus.

\textit{Hinweis: es kann sein, dass einige Konvertierungen fehlschlagen. In diesem Fall überlege dir, warum. Die Fehlermeldung kann dabei hilfreich sein.}

\begin{multicols}{3}
    \begin{enumerate}
        \item \pythoninline{int("100")}
        \item \pythoninline{float("1.3")}
        \item \pythoninline{float("NaN")}
        \item \pythoninline{str(234)}
        \item \pythoninline{int("0x11")}
        \item \pythoninline{str(2) + '5'}
        \item \pythoninline{str(2 + 5)}
        \item \pythoninline{float(5)}
        \item \pythoninline{int(5.5)}
        \item \pythoninline{int("4.2")}
    \end{enumerate}
\end{multicols}
