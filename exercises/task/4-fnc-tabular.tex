\subsection{Tabellen Ausgabe (Stufe 2)}
Schreibt ein Programm, welches eine Tabelle von Potenzen ausgibt.
\\\begin{tabular}{|c|c|c|c}\cline{1-3}
    n & $n^2$ & $n^3$          & \multirow{4}{*}{$\cdots$} \\\cline{1-3}
    1 & 1     & 1              &                           \\\cline{1-3}
    2 & 4     & 8              &                           \\\cline{1-3}
    3 & 9     & 27             &                           \\\cline{1-3}
    \multicolumn{3}{c}{\vdots}
\end{tabular}
\\Dabei soll die Größe (also die höchste Potenz und die höchste Basis) variabel,
also von einer Benutzereingabe abhängig, sein. Zum Potenzieren kann die Funktion
des Taschenrechners wiederverwendet werden. Wenn beide Klassen im selben Ordner
liegen, kann die Potenz-Funktion ähnlich der Input-Funktionen direkt aus dem
neuen Tabellen-Programm genutzt werden. Je nach vergebenen Namen kann der
Funktionsaufruf etwa \lstinline{Taschenrechner.potenz(basis, exponent)} lauten.
